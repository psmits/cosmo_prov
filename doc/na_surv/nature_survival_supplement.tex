\documentclass{article}

\usepackage{amsmath, amsthm, amsfonts, amssymb}
\usepackage{multirow, microtype}
\usepackage{longtable}

\let\oldthebibliography=\thebibliography
\let\oldendthebibliography=\endthebibliography
\renewenvironment{thebibliography}[1]{%
  \oldthebibliography{#1}%
  \setcounter{enumiv}{50}%
}{\oldendthebibliography}

\title{Supplementary information}
\author{Peter D Smits}

\begin{document}
\maketitle

\section{Supplementary methods}
\subsection{Species occurrence and covariate information}
Fossil occurrence information was downloaded from the Paleobiology Database (PBDB; http://paleodb.org/). Occurrence, taxonomic, stratigraphic, and biological information was downloaded for all North American mammals. This data set was filtered so that only occurrences identified to the species-level, excluding all ``sp.''-s. All aquatic and volant taxa were also excluded. Additionally, all occurrences without latitude and longitude information were excluded.

Species dietary and locomotor category assignments were done using the assignments in initial the PBDB which were then reassigned into coarser categories (Table \ref{tab:trait_cats}). This was done to improve interpretability, increase sample size per category, and make results comparable to previous studies \cite{Jernvall2004,Price2012}.

All individual fossil occurrences were assigned to 2 My bins ranging through the entire Cenozoic. Taxon duration was measured as the number of bins from the first occurrence to the last occurrence, inclusive. This bin size was chosen because it approximately reflects the resolution of the North American Cenozoic mammal fossil record \cite{Alroy2009,Alroy2000g,Marcot2014}. The youngest cohort, 0-2 My, was excluded from analysis.

\begin{table}
  \centering
  \caption{Species trait assignments in this study are a coarser version of the information available in the PBDB. Information was coarsened to improve per category sample size and uniformity and followed this table.}
  \begin{tabular}[ht]{ l | l | l }
    \hline
    \multicolumn{2}{ c |}{This study} & PBDB categories \\
    \hline \hline
    \multirow{4}{*}{Diet} & Carnivore & Carnivore \\
    & Herbivore & Browser, folivore, granivore, grazer, herbivore. \\
    & Insectivore & Insectivore. \\
    & Omnivore & Frugivore, omnivore. \\ 
    \hline
    \multirow{3}{*}{Locomotor} & Arboreal & Arboreal.\\
    & Ground dwelling & Fossorial, ground dwelling, semifossorial, saltatorial. \\
    & Scansorial & Scansorial. \\
    \hline
  \end{tabular}
  \label{tab:trait_cats}
\end{table}


\subsubsection{Body size estimates and estimation}
Species body size estimates were sourced from a large selection of primary literature and data base compilations.

Databases used include the PBDB, PanTHERIA \cite{Jones2009c}, and the Neogene Old World Mammal database (NOW; http://www.helsinki.fi/science/now/). Major sources of additional compiled body size estimates include \cite{Brook2004a,Freudenthal2013,McKenna2011,Raia2012f,Smith2004c,Tomiya2013}. These were then supplemented with an additional literature search to try and fill in the remaining gaps. In many cases, species body mass was estimated using various published regression equations based on tooth or skull measurements (Tab. \ref{tab:mass_est}). See Supplementary Tables XXX for a complete list of individual measures and sources.


\begin{table}
  \centering
  \begin{tabular}{l | l | l | l}
    Group & Equation & log(Measurement) & Source \\
    \hline
    General & \(\log(m) = 1.827x + 1.81\) & lower m1 area &  \cite{Legendre1986} \\
    General & \(\log(m) = 2.9677x - 5.6712\) & mandible length & \cite{Legendre1986} \\
    General & \(\log(m) = 3.68x - 3.83\) & skull length & \cite{Legendre1986} \\
    Carnivores & \(\log(m) = 2.97x + 1.681\) & lower m1 length & \\
    Insectivores & \(\log(m) = 1.628x + 1.726\) & lower m1 area & \\
    Insectivores & \(\log(m) = 1.714x + 0.886\) & upper M1 area & \\
    Lagomorph & \(\log(m) = 2.671x - 2.671\) & lower toothrow area & \\
    Lagomorph & \(\log(m) = 4.468x - 3.002\) & lower m1 length & \\
    Marsupials & \(\log(m) = 3.284x + 1.83\) & upper M1 length & \\
    Marsupials & \(\log(m) = 1.733x + 1.571\) & upper M1 area & \\
    Rodentia & \(\log(m) = 1.767x + 2.172\) & lower m1 area & \\
    Ungulates & \(\log(m) = 1.516x + 3.757\) & lower m1 area & \\
    Ungulates & \(\log(m) = 3.076x + 2.366\) & lower m2 length & \\
    Ungulates & \(\log(m) = 1.518x + 2.792\) & lower m2 area & \\
    Ungulates & \(\log(m) = 3.113x - 1.374\) & lower toothrow length & \\
    \hline
  \end{tabular}
  \caption{CAPTION}
  \label{tab:mass_est}
\end{table}



\subsubsection{Biogeographic network}
Species geographic extent was characterized using a network-theoretic approach that has previously been applied to paleontological data \cite{Sidor2013,Vilhena2013}. This approach relies on defining a biogeographic bipartite network of taxa and localities. In this study, taxa were defined as species and localities were grid cells from a regular lattice layed over a global equal-area cylinder map projection. The regular lattice was defined as a 70 x 34 global grid where each cell corresponds to approximately 25 km\(^{2}\). This network is considered bipartite because taxa are connected to localities based on their occurrence but taxa are not connected to taxa nor are localities connected to localities.

A biogeographic network was constructed for each of the 2 My bins used in this study. Emergent bioprovinces were then identified using the map equation \cite{Rosvall2008,Rosvall2009a} as has been done before \cite{Sidor2013,Vilhena2013b,Vilhena2013}. These bioprovinces correspond to taxa and localities that are more interconnected with each other than with other nodes.

The map projection and regular lattice were made using shape files from http://www.naturalearthdata.com/ and the \texttt{raster} package for R \cite{raster}. The map equation and other network related analysis was done using the \texttt{igraph} package for R \cite{csardi2006igraph}.


\subsubsection{Supertree}

As there is no single, combined formal phylogenetic hypothesis of all Cenozoic fossils mammals from North America, it was necessary to construct a semi-formal supertree. This was done by combining taxonomic information for all the observed species and a few published phylogenies.

The initial taxonomic classification of the observed species was based on the associated taxonomic information from the PBDB. This information was then updated using the Encyclopedia of Life (http://eol.org/) which collects and collates taxonomic information in a single database. This was done programatically using the \texttt{taxize} package for R \cite{2013taxize}. Finally, this taxonomic information was further updated using a published taxonomy of fossil mammals \cite{Janis2008,Janis1998}. 

This taxonomy serves as in initial phylogenetic hypothesis which was then combined with a selection of species-level phylogenies \cite{Bininda-Emonds2007,Raia2012f} in order to better constraint a minimum estimate of the actual phylogenetic relationships of the species. The supertree was inferred via matrix representation parsimony was implemented in the \texttt{phytools} package for R \cite{revell2012phytools}. While four most parsimonious trees were found, I selected a single of these for use in analysis.

Polytomies were resolved in order of species first appearance. The resulting tree was then time scaled using the \texttt{paleotree} package via the ``minimum branch length'' approach with a minimum length of 0.1 My \cite{Bapst2012a}. The minimum length is necessary to avoid zero-length branches which cause the phylogenetic covariance matrix not be positive definite, which is important for computation (see below). While other time scaling approaches are possible \cite{Bapst2013a,Hedman2010} this method was chosen for it's simplicity and not requiring additional information about diversification rates which are the interest of this study. 


\subsection{Survival model}
Presented here is the model development process used here to formulate the two survival models used in this study. 

First, define \(y\) as a vector of length \(n\) where the \(i\)th element is the durations of species \(i\) where \(i = 1,\cdots,n\).

The simplest survival model where durations are assumed to follow an exponential distribution with a single ``rate'' or inverse-scale parameter \(\lambda\) \cite{Klein2003}. This is written out
\begin{align}
  p(y | \lambda) &= \lambda \exp(-\lambda y) \nonumber \\
  y &\sim \mathrm{Exp}(\lambda).
  \label{eq:exp}
\end{align}
The exponential distribution corresponds to situations where extinction risk is independent of age. To understand this, we need to define two functions: the survival function \(S(t)\) and the hazard function \(h(t)\). 

\(S(t)\) corresponds to the probability that a species having existed for \(t\) My will not have gone extinct while \(h(t)\) corresponds to the instantaneous extinction rate for some taxon age \(t\) \cite{Klein2003}. For an exponential model, \(S(t)\) is defined
\begin{equation}
  S(t) = \exp(-\lambda t)
  \label{eq:exp_surv}
\end{equation}
and \(h(t)\) is defined
\begin{equation}
  h(t) = \lambda
  \label{eq:exp_haz}
\end{equation}
The choice of the exponential distribution corresponds directly to the Law of Constant Extinction \cite{VanValen1973} as the right side of Eq. \ref{eq:exp_haz} does not depend on species age \(t\). 

The current sampling statement (Eq. \ref{eq:exp}) currently assumes that all species share the same rate parameter with no variation. To allow for variation in \(\lambda\) associated with relevant covariate information like species body size, \(\lambda\) is reparameterized as \(\lambda_{i} = \exp(\sum \beta^{T}\mathbf{X}_{i})\) with \(i\) indexing a given observation and its covariates, \(\beta\) is a vector of regression coefficients, and \(\mathbf{X}\) is a matrix of covariates. This is a standard regression formulation, where one column of \(\mathbf{X}\) is all 1-s and its corresponding \(\beta\) coefficient is the intercept. This approach is essentially a generalized linear model (GLM) approach where instead of normally distributed errors there are exponentially distributed errors \cite{Klein2003}.

To relax the assumption of age-independent extinction of the Law of Constant Extinction we substitute the Weibull distribution for the exponential \cite{Klein2003}. The Weibull distribution has a shape parameter \(\alpha\) and scale parameter \(\sigma\). Conceptually, \(\sigma\) is the inverse of \(\lambda\). \(\alpha\) modifies the impact of taxon age on extinction risk. When \(\alpha > 1\) then \(h(t)\) is a monotonically increasing function, but when \(\alpha < 1\) then \(h(t)\) is a monotonically decreasing function. When \(\alpha = 1\) then the Weibull distribution because equivalent to the exponential.

The Weibull distribution and sampling statement wre defined
\begin{align}
  p(y | \alpha, \sigma) &= \frac{\alpha}{\sigma} \left(\frac{y}{\sigma}\right)^{\alpha - 1} \exp\left(-\left(\frac{y}{\sigma}\right)^{\alpha}\right) \nonumber \\
  y &\sim \mathrm{Weibull}(\alpha, \sigma).
  \label{eq:weibull}
\end{align}
The corresponding \(S(t)\) and \(h(t)\) functions are defined
\begin{align}
  S(t) &= \exp\left(-\left(\frac{t}{\sigma}\right)^{\alpha}\right) \label{eq:wei_surv} \\
  h(t) &= \frac{\alpha}{\sigma}\left(\frac{t}{\sigma}\right)^{\alpha - 1} \label{eq:wei_haz}.
\end{align}

To allow for \(\sigma\) to vary with a given observation's covariate information it is reparameterized in a similar fashion to \(\lambda\) with a few key differences. Because \(\sigma = 1/\lambda\) in order to preserve the interpretation of \(\beta\), while taking \(\alpha\) into account, \(\sigma\) is reparameterized as
\begin{equation}
  \sigma_{i} = \exp\left(\frac{-(\sum \beta^{T}\mathbf{X}_{i})}{\alpha}\right).
  \label{eq:reg}
\end{equation}

The model described here was the final model at the end of a continuous model development framework where the sampling and prior distributions were iteratively modified to best reflect theory, knowledge of the data, the inclusion of important covariates, and the fit to the data. This follows the approach described in \cite{Gelman2007} and \cite{Gelman2013d}.
A survival model was fit in a Bayesian context where species duration were assumed to be drawn from a Wiebull distribution (Eq. \ref{eq:weibull}) with shape \(\alpha\) and scale \(\sigma\) parameters. \(\alpha\) was assumed constant, which is standard practice in survival analysis \cite{Klein2003}. \(\alpha\) was given a weakly informative half-Cauchy (C\(^{+}\)) prior. \(\sigma\) was reparameterized as an exponentiated regression model (Eq. \ref{eq:reg}). This was further expanded (Eq. \ref{eq:wei_reg_ext}) to allow for two hierarchical factors as discussed below. This is written
\begin{equation}
  \sigma_{i} = \exp\left(\frac{-(h_{i} + \eta_{j[i]} + \sum \beta^{T} \mathbf{X}_{i})}{\alpha}\right)
  \label{eq:wei_reg_ext}
\end{equation}
where equivalent statement for the exponential distribution is defined
\begin{equation}
  \lambda_{i} = \exp\left(h_{i} + \eta_{j[i]} + \sum \beta^{T} \mathbf{X}_{i})\right).
  \label{eq:exp_reg_ext}
\end{equation}

\(K\) species-level covariates were included as a \(n \times K\) matrix, \(\mathbf{X}\). Two of the covariates of interest are the logit of mean relative occupancy and the logarithm of body size (g). The discrete covariate index variables of dietary and locomotor category were transformed into \(n \times (k - 1)\) matrices where each column is an indicator variable (0/1) for that species's category, \(k\) being the number of categories of the index variable (3 and 4, respectively). Only \(k - 1\) indicator variables are necessary as the intercept takes on the remaining value. Finally, a vector of 1-s was included in the matrix \(\mathbf{X}\) whose corresponding \(\beta\) coefficient is the intercept, making \(K\) equal eight.

\(\beta\) is a vector of regression coefficients, where each element is given a unique, weakly informative Normally distributed prior. These priors were chosen because it is expected that the effect size of each variable on duration will be small, as is generally the case with binary covariates. %In all cases, posterior inference was not effected by changes to this choice of prior. Do I have proof?

Regression coefficients are not directly comparable without first standardizing the input variables to have equal standard deviations. This is accomplished by subtracting the mean of the covariate from all values and then dividing by the standard deviation, resulting in a variable with mean of zero and a standard deviation of one. This linear transform greatly improves the interpretability of the coefficients as expected change in mean duration given a difference of one standard deviation in the covariate \cite{Schielzeth2010}. Additionally, this makes the intercept directly interpretable as the estimate of mean (transformed) \(\sigma\) (Eq. \ref{eq:reg}). However, because the expected standard deviation for a binary variable is 0.5, in order to make comparisons between the binary and continuous variables, the continuous inputs must instead by divided by twice their standard deviation \cite{Gelman2008}. 

\subsubsection{Hierarchical effects}

The two hierarchical effects of interest in this study are origination cohort and shared evolutionary history, or phylogeny. Hierarchical modeling can be considered an intermediate between complete and no pooling of groups \cite{Gelman2007}, where complete pooling is when the differences between groups are ignored and no pooling is where different groups are analyzed separately. By allowing for partial pooling, we are modeling the appropriate compromise between these two extremes, allowing for better and potentially more informative overall inference. This is done by having all of the groups share the same Normal prior with mean 0 and a scale parameter estimated from the data, which then acts as an indicator of the amount of pooling. A scale of 0 and \(\infty\) indicate complete and no pooling, respectively. The choice of mean 0 allows for the individual groups estimates to be interpreted as deviations from the intercept. Hierarchical modeling is analogous to mixed-effects modeling \cite{Gelman2007}.

Origination cohort is defined as the group of species which all originated during the same 2 My temporal bin. Because the most recent temporal bin, 0-2 Mya was excluded, there are 32 different cohorts. The effect of origination cohort \(j\) was modeled with each group being a sample from a common cohort effect, \(\eta\), which was considered Normally distributed with mean 0, and standard deviation \(\sigma_{c}\). The value of \(\sigma_{c}\) was then estimated from the data itself, corresponding to the amount of pooling in the individual estimates of \(\eta_{j}\). This approach is a conceptual and statistical unification between dynamic and cohort survival analysis in paleontology \cite{Foote1988,Raup1978,Raup1975,VanValen1979,Baumiller1993}, with \(\sigma_{c}\) acting as a measure of compromise between these two end members.

\begin{align*}
  \eta_{j} &\sim \mathcal{N}(0, \sigma_{c}) \\
  \sigma_{c} &\sim \mathrm{C}^{+}(0, 2.5)
\end{align*}

The choice of the half-Cauchy prior on \(\sigma_{c}\) follows \cite{Gelman2006a}

The impact of shared evolutionary history, or phylogeny, was modeling as an individual effect where each observation, \(i\), is distributed as a multivariate normal, \(h\), where the covariance matrix \(\Sigma\) is known up to a constant, \(\sigma_{p}^{2}\) \cite{Lynch1991,Housworth2004}. This is written

\begin{align*}
  h &\sim \mathrm{Multivariate\ }\mathcal{N}(0, \mathbf{\Sigma}) \\
  \mathbf{\Sigma} &= \sigma_{p}^{2} \mathbf{V}_{phy} \\
  \sigma_{p} &\sim \mathrm{C}^{+}(0, 2.5).
\end{align*}

\(\mathbf{V}_{phy}\) is the phylogenetic covariance matrix defined as an \(n \times n\) matrix where the diagonal elements are the distance from root to tip, in branch length, for each observation and the off-diagonal elements are the amount of shared history, measured in branch length, between observations \(i\) and \(j\). \(\sigma_{p}\) was given a weakly informative half-Cauchy hyperprior. 


\subsubsection{Censored observations} \label{sec:censor}

An important part of survival analysis is the inclusion of censored observations where the failure time has not been observed \cite{Ibrahim2001,Klein2003}. The most common censored observation is right censored, where the point of extinction had not yet been observed in the period of study, such as taxa that are still extant. Left censored observations, on the other hand, correspond to observations that went extinct any time between 0 and some known point. In order to account for the minimum resolution of the fossil record encountered here, taxa that occurred in only a single time bin were left censored.

Censored data is modeled using the survival function of the distribution, \(S(t)\), defined earlier for the Weibull distribution (Eq. \ref{eq:wei_surv}) with \(\sigma\) is defined as above (Eq. \ref{eq:wei_reg_ext}). \(S(t)\) is the probability that an observation will survive longer than a given time \(t\). The likelihood of uncensored observations is evaluated as normal using Equation \ref{eq:weibull} while right censored observations are evaluated at \(S(t)\) and left censored observations are evaluated at \(1 - S(t)\). Note, \(1 - S(t)\) is equivalent to the cumulative density function and \(S(t)\) is equivalent to the complementary cumulative density function \cite{Gelman2013d}.

The full likelihood for both uncensored and both right and left censored observations is written
\begin{equation*}
  L \propto \prod_{i \in C} \mathrm{Weibull}(y_{i} | \alpha, \sigma) \prod_{j \in R} S(y_j | \alpha, \sigma) \prod_{k \in L} \left(1 - S(y_{k} | \alpha, \sigma)\right),
\end{equation*}
where \(C\) is the set of uncensored observations, \(R\) is the set of right censored observations, and \(L\) is the set of left censored observations.



\subsubsection{Estimation}
Parameter posteriors were approximated using a Markov-chain Monte Carlo (MCMC) routine implemented in the Stan programming language \cite{2014stan}. Stan implements a Hamiltonian Monte Carlo using a No-U-Turn sampler \cite{Hoffman-Gelman:2011}. Posterior approximation was done using four parallel MCMC chains where convergence was evaluated using the scale reduction factor, \(\hat{R}\). Values of \(\hat{R}\) close to 1, or less than or equal to 1.1, indicate approximate convergence. Convergence means that the chains are approximately stationary and the samples are well mixed \cite{Gelman2013d}.

In order to speed up the poisterior approximation, a custom multivariate normal sampler was used to estimate the unknown constant term in the covariance matrix. This is necessary because inverting and solving the complete covariance matrix on every iteration is a memory intense procedure. The custom sampler limits the necessary number of operations and matrix inversions per iteration.

Four different MCMC chains were run for 20000 steps and were thinned to every twentieth sample split evenly between warm-up and sampling. 



\subsection{Posterior predictive checks}

The most basic assessment of model fit is that simulated data generated using the fitted model should be similar to the observed. This is the idea behind posterior predictive checks. Using the covariates from each of the observed durations, and randomly drawn parameter estimates from their marginal posteriors, a simulated data set \(y^{rep}\) was generated. This process was repeated 1000 times and the distribution of \(y^{rep}\) was compared with the observed \cite{Gelman2013d}.

An example posterior predictive check used in this study is a graphical comparison between the Kaplan-Meier (K-M) survival curve estimated from the observed data and the K-M survival curves estimated from 1000 simulation sets. K-M survival curves are non-parametric estimates of \(S(t)\) \cite{Klein2003}. Other posterior predictive checks used here include comparison of the mean and quantiles of the observed durations to the distributions of the same quantities from the simulations, and inspection of the deviance residuals, defined below.

In standard linear regression, residuals are defined as \(r_{i} = y_{i} - y_{i}^{est}\). For the model used here, this definition is inadequate. The equivalent values for survival analysis are deviance residuals. To define how deviance residuals are calculated, we first define the cummulative hazard function \cite{Klein2003}. Given \(S(t)\) (Eq. \ref{eq:wei_surv}), we define the cumulative hazard function as 
\begin{equation*}
  \Lambda(t) = -log\left(S\left(t\right)\right).
\end{equation*}

Next, we define martingale residuals \(m\) as
\begin{equation*}
  m_{i} = I_{i} - \Lambda(t_i).
\end{equation*}
\(I\) is the inclusion vector of length \(n\), where \(I_{i} = 1\) means the observation is completely observed and \(I_{i} = 0\) means the observation is censored. Martingale residuals have a mean of 0, range between 1 and \(-\infty\), and can be viewed as the difference between the observed number of deaths between 0 and \(t_{i}\) and the expected number of deaths based on the model. However, martingale residuals are asymmetrically distributed, and can not be interpreted in the same manner as standard residuals. 

The solution to this is to use the deviance residuals, \(D\). This is defined as a function of martingale residuals and takes the form
\begin{equation*}
  D_{i} = \text{sign}(m_{i}) \sqrt{-2[m_{i} + I_{i}log(I_{i} - m_{i})]}.
\end{equation*}
Deviance residuals have a mean of 0 and a standard deviation of 1 by definition.


\subsection{Variance partitioning}
There are three different variance components in this model: sample \(\sigma_{y}^{2}\), cohort \(\sigma_{c}^{2}\), and phylogenetic \(\sigma_{p}^{2}\). The sample variance, \(\sigma_{y}^{2}\), is similar to the residual variance from a normal linear regression. Partitioning the variance between these sources allows the relative amount of unexplained variance of the sample to be compared. However, the Weibull based model used here (Eq. \ref{eq:weibull}) does not include an estimate of the sample variance, \(\sigma_{y}^{2}\). Partitioning the variance between these three components was approximated via a simulation approach modified from \cite{Goldstein2002}.

The procedure is as follows:
\begin{enumerate}
  \item Simulate \(w\) (50,000) values of \(\eta\); \(\eta \sim \mathcal{N}(0, \sigma_{c})\).
  \item For a given value of \(\beta^{T} \mathbf{X}\), calculate \(\sigma^{c*}\) (Eq. \ref{eq:reg}) for all \(w\) simulations, holding \(h\) constant at 0.
  \item Calculate \(\upsilon_{c}\), the Weibull variance (Eq. \ref{eq:wei_var}) of each element of \(\sigma^{c*}\) with \(\alpha\) drawn from the posterior estimate.
  \item Simulate \(w\) values of \(h\); \(h \sim \mathcal{N}(0, \sigma_{p})\). 
  \item For a given value of \(\beta^{T} \mathbf{X}\), calculate \(\sigma^{p*}\) (Eq. \ref{eq:reg}) for all \(w\) simulations, holding \(\eta\) constant at 0.
  \item Calculate \(\upsilon_{p}\), the Weibull variance (Eq. \ref{eq:wei_var}) of each element of \(\sigma^{p*}\) with \(\alpha\) drawn from the posterior estimate.
  \item \(\sigma_{y*}^{2} = \frac{1}{2} \left(\left(\frac{1}{w} \sum_{i}^{w} \upsilon_{pi}\right) + \left(\frac{1}{w} \sum_{j}^{w} \upsilon_{cj}\right)\right)\).
  \item \(\sigma_{c*}^{2} = var(\upsilon_{c})\) and \(\sigma_{p*}^{2} = var(\upsilon_{p})\).
\end{enumerate}

The simulated values of \(h\) were drawn from a univariate normal distribution because each simulated value is in isolation, so there are is no concern of phylogenetic autocorrelation. The chosen value for \(\beta^{T} \mathbf{X}\) was a draw from the posterior estimate of the intercept. Because input variables were standardized prior to model fitting, the intercept corresponds to the estimated effect on survival of the sample mean.

Weibull variance is calculated as
\begin{equation}
  var(x) = \sigma^{2}\left(\Gamma\left(1 + \frac{2}{\alpha}\right) - \left(\Gamma\left(1 + \frac{1}{\alpha}\right)\right)^{2}\right),
  \label{eq:wei_var}
\end{equation}
where \(\Gamma\) is the gamma function. 

The variance partitioning coefficients are then calculated, for example, as \(VPC_{phylo} = \frac{\sigma_{p*}^{2}}{\sigma_{y*}^{2} + \sigma_{c*}^{2} + \sigma_{p*}^{2}}\) and similarly for the other components.

I used variance partitioning coefficients (VPC) to estimate of the relative size importance of the different variance components \cite{Gelman2007}. Phylogenetic heritability, \(h_{p}^{2}\) \cite{Lynch1991,Housworth2004}, is identical to the VPC of the phylogenetic effect. Additionally, because phylogenetic effect was estimated using a principally taxonomy based tree the estimates derived here can be considered minimum estimates of the phylogenetic effect.


%\subsection{Model Adequacy}
%Deviance residuals.
%
%Point estimates.
%
%K-M estimates of S(t).

%\section{Supplementary tables}
%% latex table generated in R 3.1.3 by xtable 1.7-4 package
% Mon Apr  6 13:32:21 2015
\begin{table}[ht]
\centering
\begin{tabular}{lrl}
  \hline
Species & Mass (g) & Source \\ 
  \hline
Aaptoryctes ivyi & 9.40 & PBDB + regression \\ 
  Absarokius abbotti & 6.30 & PBDB + regression \\ 
  Absarokius metoecus & 6.10 & PBDB + regression \\ 
  Acarictis ryani & 8.30 & PBDB + regression \\ 
  Achaenodon robustus & 484.90 & PBDB + regression \\ 
  Achlyoscapter longirostris & 11.94 & cite{Tomiya2013} \\ 
  Acmeodon secans & 12.10 & PBDB + regression \\ 
  Acritohippus isonesus & 135944.23 & cite{Tomiya2013} \\ 
  Acritohippus quinni & 178082.11 & cite{Tomiya2013} \\ 
  Acritoparamys atwateri & 5.70 & PBDB + regression \\ 
  Acritoparamys francesi & 4.50 & PBDB + regression \\ 
  Acritoparamys pattersoni & 9.07 & PBDB \\ 
  Acritoparamys wyomingensis & 7.02 & PBDB \\ 
  Adeloblarina berklandi & 12.68 & cite{Tomiya2013} \\ 
  Adelphailurus kansensis & 33189.87 & cite{Tomiya2013} \\ 
  Adilophontes brachykolos & 119372.01 & cite{Tomiya2013} \\ 
  Adjidaumo burkei & 0.72 & PBDB \\ 
  Adjidaumo craigi & 0.90 & PBDB + regression \\ 
  Adjidaumo intermedius & 1.10 & PBDB + regression \\ 
  Adjidaumo maximus & 1.60 & PBDB + regression \\ 
  Adjidaumo minimus & 0.90 & PBDB + regression \\ 
  Adjidaumo minutus & 1.40 & PBDB + regression \\ 
  Adunator ladae & 2.00 & PBDB + regression \\ 
  Aelurodon asthenostylus & 22026.47 & cite{Tomiya2013} \\ 
  Aelurodon ferox & 26370.47 & cite{Tomiya2013} \\ 
  Aelurodon mcgrewi & 22247.84 & cite{Tomiya2013} \\ 
  Aelurodon montanensis & 27722.51 & cite{Tomiya2013} \\ 
  Aelurodon stirtoni & 20537.34 & cite{Tomiya2013} \\ 
  Aelurodon taxoides & 29436.77 & cite{Tomiya2013} \\ 
  Aepinacodon americanus & 809.20 & cite{Baskin2011} \\ 
  Aepycamelus bradyi & 922.04 & cite{Dawson2012} \\ 
  Aepycamelus giraffinus & 901.44 & PBDB \\ 
  Aepycamelus robustus & 420836.64 & cite{Tomiya2013} \\ 
  Aepycamelus stocki & 348014.70 & cite{Tomiya2013} \\ 
  Aethomylos simplicidens & 2.80 & PBDB + regression \\ 
  Ageitodendron matthewi & 8.90 & PBDB + regression \\ 
  Agriochoerus antiquus & 56387.34 & cite{Tomiya2013} \\ 
  Agriochoerus guyotianus & 59000.00 & cite{McKenna2011} \\ 
  Agriotherium schneideri & 355045.06 & cite{Tomiya2013} \\ 
  Alagomys russelli & 0.70 & PBDB + regression \\ 
  Aletodon conardae & 17.20 & PBDB + regression \\ 
  Aletodon gunnelli & 17.90 & PBDB + regression \\ 
  Aletodon quadravus & 13.00 & PBDB + regression \\ 
  Aletomeryx gracilis & 27173.57 & cite{Tomiya2013} \\ 
  Alforjas taylori & 408399.03 & cite{Tomiya2013} \\ 
  Alilepus vagus & 7.20 & PBDB + regression \\ 
  Alilepus wilsoni & 4.90 & PBDB + regression \\ 
  Allomys cristabrevis & 6.50 & PBDB + regression \\ 
  Allomys simplicidens & 122.00 & cite{McKenna2011} \\ 
  Allomys storeri & 3.80 & PBDB + regression \\ 
  Alluvisorex arcadentes & 5.70 & cite{Tomiya2013} \\ 
  Alphagaulus pristinus & 521.00 & cite{McKenna2011} \\ 
  Alphagaulus vetus & 523.22 & cite{Tomiya2013} \\ 
  Alticonus gazini & 17.50 & PBDB + regression \\ 
  Alveojunctus minutus & 2.20 & PBDB + regression \\ 
  Alveugena carbonensis & 21.27 & PBDB \\ 
  Alwoodia harkseni & 7.30 & PBDB + regression \\ 
  Alwoodia magna & 226.00 & cite{McKenna2011} \\ 
  Amebelodon floridanus & 8243.20 & PBDB + regression \\ 
  Amelotabes simpsoni & 3.40 & PBDB + regression \\ 
  Ammospermophilus hanfordi & 58.00 & cite{McKenna2011} \\ 
  Ammospermophilus junturensis & 53.52 & cite{Tomiya2013} \\ 
  Amphechinus horncloudi & 175.91 & cite{Tomiya2013} \\ 
  Amphicaenopus platycephalus & 2397650.84 & cite{Tomiya2013} \\ 
  Amphicyon frendens & 245241.81 & cite{Tomiya2013} \\ 
  Amphicyon galushai & 138690.48 & cite{Tomiya2013} \\ 
  Amphicyon ingens & 600000.00 & cite{Sorkin2008} \\ 
  Amphicyon longiramus & 113550.16 & cite{Tomiya2013} \\ 
  Amphicyon riggsi & 443.30 & PBDB \\ 
  Amphimachairodus coloradensis & 31.90 & PBDB + regression \\ 
  Ampliconus antoni & 29.20 & PBDB + regression \\ 
  Amynodon advenus & 1197.30 & PBDB + regression \\ 
  Amynodon reedi & 994.15 & PBDB \\ 
  Amynodontopsis bodei & 548.30 & PBDB + regression \\ 
  Anaptomorphus aemulus & 4.50 & PBDB + regression \\ 
  Anaptomorphus westi & 6.50 & PBDB + regression \\ 
  Anasazia williamsoni & 6.49 & cite{Williamson2012} \\ 
  Anchitherium clarencei & 230960.04 & cite{Tomiya2013} \\ 
  Anconodon cochranensis & 57.00 & cite{Wilson2012} \\ 
  Anemorhysis natronensis & 2.50 & PBDB + regression \\ 
  Anemorhysis pattersoni & 3.40 & PBDB + regression \\ 
  Anemorhysis pearcei & 2.48 & cite{Albright2000} \\ 
  Anemorhysis wortmani & 2.90 & PBDB + regression \\ 
  Angustidens vireti & 18.17 & cite{Tomiya2013} \\ 
  Anisonchus oligistus & 9.70 & PBDB + regression \\ 
  Anisonchus onostus & 12.50 & PBDB + regression \\ 
  Anisonchus sectorius & 21.20 & PBDB + regression \\ 
  Ankalagon saurognathus & 17.90 & PBDB + regression \\ 
  Ankylodon annectens & 30.27 & cite{Tomiya2013} \\ 
  Anomoemys lewisi & 5.99 & cite{Simons1960} \\ 
  Ansomys hepburnensis & 44.70 & cite{Tomiya2013} \\ 
  Ansomys nevadensis & 51.94 & cite{Tomiya2013} \\ 
  Ansomys nexodens & 3.80 & PBDB + regression \\ 
  Antecalomys phthanus & 22.87 & cite{Tomiya2013} \\ 
  Antecalomys valensis & 13.60 & cite{Tomiya2013} \\ 
  Antecalomys vasquezi & 10.70 & PBDB \\ 
  Antiacodon pygmaeus & 10.70 & PBDB + regression \\ 
  Antiacodon venustus & 24.80 & PBDB + regression \\ 
  Apataelurus kayi & 60.00 & PBDB + regression \\ 
  Apatemys bellulus & 2.00 & PBDB + regression \\ 
  Apatemys bellus & 2.30 & PBDB + regression \\ 
  Apatemys chardini & 1.80 & PBDB + regression \\ 
  Apatemys downsi & 4.50 & PBDB + regression \\ 
  Apatemys hendryi & 1.60 & PBDB + regression \\ 
  Apatemys uintensis & 2.70 & PBDB + regression \\ 
  Apatosciuravus bifax & 2.20 & PBDB + regression \\ 
  Apatosciuravus jacobsi & 2.00 & PBDB + regression \\ 
  Apheliscus chydaeus & 3.70 & PBDB + regression \\ 
  Apheliscus insidiosus & 4.30 & PBDB + regression \\ 
  Apheliscus nitidus & 5.00 & PBDB + regression \\ 
  Apheliscus wapitiensis & 2.60 & PBDB + regression \\ 
  Aphelops malacorhinus & 3541284.24 & cite{Tomiya2013} \\ 
  Aphelops megalodus & 1689595.99 & cite{Tomiya2013} \\ 
  Aphelops mutilus & 4325334.34 & cite{Tomiya2013} \\ 
  Aphronorus fraudator & 6.30 & PBDB + regression \\ 
  Aphronorus orieli & 14.30 & PBDB + regression \\ 
  Aphronorus ratatoski & 8.30 & PBDB + regression \\ 
  Apletotomeus crassus & 1.00 & PBDB + regression \\ 
  Apternodus gregoryi & 5.50 & PBDB + regression \\ 
  Apternodus iliffensis & 5.40 & PBDB + regression \\ 
  Arapahovius advena & 2.90 & PBDB + regression \\ 
  Archaeocyon falkenbachi & 88.00 & cite{Stirton1932} \\ 
  Archaeocyon leptodus & 3533.34 & cite{Tomiya2013} \\ 
  Archaeocyon pavidus & 2275.60 & cite{Tomiya2013} \\ 
  Archaeohippus blackbergi & 33189.87 & cite{Tomiya2013} \\ 
  Archaeohippus mannulus & 104.00 & PBDB + regression \\ 
  Archaeohippus mourningi & 54176.36 & cite{Tomiya2013} \\ 
  Archaeohippus penultimus & 71682.36 & cite{Tomiya2013} \\ 
  Archaeohippus stenolophus & 123.76 & PBDB \\ 
  Archaeolagus acaricolus & 578.25 & cite{Tomiya2013} \\ 
  Archaeolagus emeraldensis & 2344.90 & cite{Tomiya2013} \\ 
  Archaeolagus ennisianus & 1064.22 & cite{Tomiya2013} \\ 
  Archaeolagus macrocephalus & 1826.21 & cite{Tomiya2013} \\ 
  Archaeolagus primigenius & 1540.71 & cite{Tomiya2013} \\ 
  Archaeotherium lemleyi & 932.90 & PBDB + regression \\ 
  Archaeotherium mortoni & 297.60 & PBDB + regression \\ 
  Archaeotherium trippensis & 1691.00 & PBDB + regression \\ 
  Arctocyon montanensis & 71.10 & PBDB + regression \\ 
  Arctocyon mumak & 149.00 & PBDB \\ 
  Arctodontomys nuptus & 8.70 & PBDB + regression \\ 
  Arctodontomys simplicidens & 7.10 & PBDB + regression \\ 
  Arctodontomys wilsoni & 7.40 & PBDB + regression \\ 
  Arctodus pristinus & 239.13 & cite{Smith2004} \\ 
  Arctonasua eurybates & 15994.50 & cite{Tomiya2013} \\ 
  Arctonasua gracilis & 8866.19 & cite{Tomiya2013} \\ 
  Arctonasua minima & 7044.48 & cite{Tomiya2013} \\ 
  Arctostylops steini & 11.30 & PBDB + regression \\ 
  Ardynomys occidentalis & 5.90 & PBDB + regression \\ 
  Arfia junnei & 17.90 & PBDB + regression \\ 
  Arfia opisthotoma & 61.70 & PBDB + regression \\ 
  Arfia shoshoniensis & 56.00 & PBDB + regression \\ 
  Arfia zele & 25.30 & PBDB + regression \\ 
  Arretotherium acridens & 179871.86 & cite{Tomiya2013} \\ 
  Arretotherium fricki & 138690.48 & cite{Tomiya2013} \\ 
  Arretotherium leptodus & 575.99 & PBDB \\ 
  Artimonius australis & 5.40 & PBDB + regression \\ 
  Artimonius nocerai & 3.70 & PBDB + regression \\ 
  Artimonius witteri & 4.40 & PBDB + regression \\ 
  Astrohippus stockii & 203.04 & PBDB \\ 
  Aulolithomys bounites & 3.20 & PBDB + regression \\ 
  Aulolithomys vexilliames & 1.50 & PBDB + regression \\ 
  Australocamelus orarius & 100709.96 & cite{Tomiya2013} \\ 
  Auxontodon pattersoni & 26.80 & PBDB + regression \\ 
  Avunculus didelphodonti & 4.30 & PBDB + regression \\ 
  Aycrossia lovei & 5.20 & PBDB + regression \\ 
  Aztlanolagus agilis & 27.14 & cite{Smith2004} \\ 
  Azygonyx ancylion & 74.20 & PBDB + regression \\ 
  Azygonyx grangeri & 94.00 & PBDB + regression \\ 
  Azygonyx xenicus & 48.40 & PBDB + regression \\ 
  Baioconodon denverensis & 35.70 & PBDB + regression \\ 
  Baioconodon nordicus & 21.60 & PBDB + regression \\ 
  Baiomys rexroadi & 0.90 & PBDB + regression \\ 
  Baiotomeus douglassi & 6.60 & PBDB + regression \\ 
  Baiotomeus rhothonion & 1.50 & PBDB + regression \\ 
  Barbourofelis fricki & 255250.32 & cite{Tomiya2013} \\ 
  Barbourofelis loveorum & 32.10 & PBDB + regression \\ 
  Barbourofelis morrisi & 90219.42 & cite{Tomiya2013} \\ 
  Barbourofelis osborni & 264.00 & cite{Martin2002a} \\ 
  Barbourofelis whitfordi & 77652.58 & cite{Tomiya2013} \\ 
  Barbouromeryx trigonocorneus & 33860.35 & cite{Tomiya2013} \\ 
  Barylambda faberi & 726.70 & PBDB + regression \\ 
  Barylambda jackwilsoni & 246.20 & PBDB + regression \\ 
  Basirepomys pliocenicus & 3.30 & PBDB + regression \\ 
  Basirepomys robertsi & 2.80 & PBDB + regression \\ 
  Bassariscus antiquus & 1881.83 & cite{Tomiya2013} \\ 
  Bassariscus casei & 1652.43 & cite{Tomiya2013} \\ 
  Bassariscus minimus & 6.15 & cite{Robinson1966} \\ 
  Bassariscus ogallalae & 21.94 & cite{Gidley1920} \\ 
  Bassariscus parvus & 1685.81 & cite{Tomiya2013} \\ 
  Bathygenys alpha & 26.50 & PBDB + regression \\ 
  Bathygenys reevesi & 24.00 & PBDB + regression \\ 
  Batodonoides powayensis & 0.70 & PBDB + regression \\ 
  Batodonoides vanhouteni & 0.40 & PBDB + regression \\ 
  Beckiasorex hibbardi & 1.10 & PBDB + regression \\ 
  Bensonomys arizonae & 1.73 & PBDB \\ 
  Bensonomys baskini & 1.89 & PBDB \\ 
  Bensonomys elachys & 1.00 & cite{Kirk2011} \\ 
  Bensonomys gidleyi & 1.30 & PBDB \\ 
  Bensonomys lindsayi & 1.00 & PBDB + regression \\ 
  Bensonomys meadensis & 1.55 & PBDB \\ 
  Bensonomys winklerorum & 1.02 & PBDB \\ 
  Bensonomys yazhi & 1.08 & cite{Rose2013a} \\ 
  Betonnia tsosia & 3.84 & cite{Clemens2005} \\ 
  Bisonalveus browni & 5.50 & PBDB + regression \\ 
  Bisonalveus holtzmani & 7.40 & PBDB + regression \\ 
  Blacktops latidens & 18.02 & PBDB \\ 
  Blacktops longinares & 15.14 & PBDB \\ 
  Blarina brevicauda & 16.40 & PBDB \\ 
  Blarina carolinensis & 3.10 & cite{Smith2004} \\ 
  Blastomeryx gemmifer & 10938.02 & cite{Tomiya2013} \\ 
  Blastomeryx pristinus & 168.00 & PBDB \\ 
  Blickomylus galushai & 16983.54 & cite{Tomiya2013} \\ 
  Boreameryx braskerudi & 56.30 & PBDB + regression \\ 
  Borophagus diversidens & 34891.55 & cite{Tomiya2013} \\ 
  Borophagus dudleyi & 247.00 & cite{Dalquest1978} \\ 
  Borophagus hilli & 29143.87 & cite{Tomiya2013} \\ 
  Borophagus littoralis & 23388.51 & cite{Tomiya2013} \\ 
  Borophagus orc & 16814.55 & cite{Tomiya2013} \\ 
  Borophagus parvus & 19341.34 & cite{Tomiya2013} \\ 
  Borophagus pugnator & 24100.79 & cite{Tomiya2013} \\ 
  Borophagus secundus & 24100.79 & cite{Tomiya2013} \\ 
  Bothriodon rostratus & 451.00 & cite{Cassiliano2008} \\ 
  Bouromeryx americanus & 68186.37 & cite{Tomiya2013} \\ 
  Bouromeryx submilleri & 50011.09 & cite{Tomiya2013} \\ 
  Brachycrus buwaldi & 250196.03 & cite{Tomiya2013} \\ 
  Brachycrus laticeps & 387.96 & PBDB \\ 
  Brachycrus rusticus & 185.22 & PBDB \\ 
  Brachycrus siouense & 145801.30 & cite{Tomiya2013} \\ 
  Brachyerix hibbardi & 5.04 & cite{Clemens2011} \\ 
  Brachyerix incertis & 79.84 & cite{Tomiya2013} \\ 
  Brachyerix macrotis & 131.63 & cite{Tomiya2013} \\ 
  Brachyerix richi & 340.36 & cite{Tomiya2013} \\ 
  Brachyhyops viensis & 298.00 & PBDB + regression \\ 
  Brachyhyops wyomingensis & 199.30 & PBDB + regression \\ 
  Brachypsalis modicus & 512.86 & cite{Tomiya2013} \\ 
  Brachypsalis obliquidens & 157.50 & PBDB \\ 
  Brachypsalis pachycephalus & 487.85 & cite{Tomiya2013} \\ 
  Brachyrhynchocyon dodgei & 14.10 & PBDB + regression \\ 
  Brachyrhynchocyon montanus & 11.40 & PBDB + regression \\ 
  Brontops tyleri & 571500.00 & PBDB \\ 
  Buisnictis breviramus & 5.50 & PBDB + regression \\ 
  Buisnictis burrowsi & 14.58 & PBDB \\ 
  Buisnictis schoffi & 22.42 & cite{Tomiya2013} \\ 
  Bunomeryx montanus & 17.20 & PBDB + regression \\ 
  Bunophorus etsagicus & 44.10 & PBDB + regression \\ 
  Bunophorus grangeri & 45.50 & PBDB + regression \\ 
  Bunophorus macropternus & 37.40 & PBDB + regression \\ 
  Bunophorus pattersoni & 27.00 & cite{Macdonald1956} \\ 
  Bunophorus robustus & 23.60 & PBDB + regression \\ 
  Bunophorus sinclairi & 50.40 & PBDB + regression \\ 
  Caenolambda jepseni & 235.70 & PBDB + regression \\ 
  Calippus cerasinus & 81633.91 & cite{Tomiya2013} \\ 
  Calippus elachistus & 43044.94 & cite{Tomiya2013} \\ 
  Calippus hondurensis & 71682.36 & cite{Tomiya2013} \\ 
  Calippus martini & 119372.01 & cite{Tomiya2013} \\ 
  Calippus placidus & 79221.26 & cite{Tomiya2013} \\ 
  Calippus proplacidus & 64860.88 & cite{Tomiya2013} \\ 
  Calippus regulus & 45251.90 & cite{Tomiya2013} \\ 
  Camelops hesternus & 420.31 & cite{Smith2004} \\ 
  Camelops traviswhitei & 774.70 & PBDB + regression \\ 
  Campestrallomys annectens & 3.50 & PBDB + regression \\ 
  Campestrallomys dawsonae & 298.87 & cite{Tomiya2013} \\ 
  Campestrallomys siouxensis & 159.17 & cite{Tomiya2013} \\ 
  Canis armbrusteri & 30333.26 & cite{Tomiya2013} \\ 
  Canis edwardii & 25.40 & PBDB + regression \\ 
  Canis latrans & 11765.00 & PBDB \\ 
  Canis lepophagus & 14617.87 & cite{Tomiya2013} \\ 
  Canis rufus & 15566.00 & PBDB \\ 
  Cantius abditus & 28.60 & PBDB + regression \\ 
  Cantius angulatus & 10.10 & PBDB + regression \\ 
  Cantius frugivorus & 15.50 & PBDB + regression \\ 
  Cantius mckennai & 15.40 & PBDB + regression \\ 
  Cantius nunienus & 18.10 & PBDB + regression \\ 
  Cantius ralstoni & 13.80 & PBDB + regression \\ 
  Cantius simonsi & 34.10 & PBDB + regression \\ 
  Cantius torresi & 10.00 & PBDB + regression \\ 
  Cantius trigonodus & 2000.00 & cite{Soligo2006} \\ 
  Capricamelus gettyi & 389.10 & PBDB + regression \\ 
  Capromeryx tauntonensis & 15835.35 & cite{Tomiya2013} \\ 
  Cardiolophus radinskyi & 79.10 & PBDB + regression \\ 
  Cardiolophus semihians & 80.80 & PBDB + regression \\ 
  Carpocristes cygneus & 2.20 & PBDB + regression \\ 
  Carpocristes hobackensis & 33.00 & cite{Soligo2006} \\ 
  Carpocyon compressus & 15214.44 & cite{Tomiya2013} \\ 
  Carpocyon robustus & 19341.34 & cite{Tomiya2013} \\ 
  Carpocyon webbi & 20537.34 & cite{Tomiya2013} \\ 
  Carpodaptes hazelae & 3.20 & PBDB + regression \\ 
  Carpodaptes stonleyi & 2.50 & PBDB + regression \\ 
  Carpolestes nigridens & 87.00 & cite{Scott2003a} \\ 
  Carpolestes simpsoni & 2.30 & PBDB + regression \\ 
  Carpomegodon jepseni & 6.10 & PBDB + regression \\ 
  Catopsalis alexanderi & 31.90 & PBDB + regression \\ 
  Catopsalis calgariensis & 90.20 & PBDB + regression \\ 
  Catopsalis foliatus & 48.60 & PBDB + regression \\ 
  Catopsalis joyneri & 2435.00 & cite{Wilson2012} \\ 
  Cedromus wardi & 6.60 & PBDB + regression \\ 
  Centetodon aztecus & 2.00 & PBDB + regression \\ 
  Centetodon bembicophagus & 1.50 & PBDB + regression \\ 
  Centetodon chadronensis & 2.40 & PBDB + regression \\ 
  Centetodon divaricatus & 30.57 & cite{Tomiya2013} \\ 
  Centetodon hendryi & 2.00 & PBDB + regression \\ 
  Centetodon kuenzii & 2.60 & PBDB + regression \\ 
  Centetodon magnus & 33.45 & cite{Tomiya2013} \\ 
  Centetodon neashami & 4.00 & PBDB + regression \\ 
  Centetodon pulcher & 2.80 & PBDB + regression \\ 
  Centimanomys major & 6.50 & PBDB + regression \\ 
  Ceratogaulus hatcheri & 80.00 & cite{Cassiliano2008} \\ 
  Cernictis hesperus & 177.68 & cite{Tomiya2013} \\ 
  Cernictis repenningi & 53.55 & cite{Hall1930} \\ 
  Chacomylus sladei & 5.80 & PBDB + regression \\ 
  Chadrolagus emryi & 3.60 & PBDB + regression \\ 
  Chalicomomys willwoodensis & 0.80 & PBDB + regression \\ 
  Chasmaporthetes ossifragus & 28.10 & PBDB + regression \\ 
  Chipetaia lamporea & 10.10 & PBDB + regression \\ 
  Chiromyoides caesor & 7.10 & PBDB + regression \\ 
  Chiromyoides minor & 3.80 & PBDB + regression \\ 
  Chiromyoides potior & 6.80 & PBDB + regression \\ 
  Chriacus badgleyi & 18.00 & PBDB + regression \\ 
  Chriacus baldwini & 20.70 & PBDB + regression \\ 
  Chriacus gallinae & 20.10 & PBDB + regression \\ 
  Chriacus pelvidens & 39.50 & PBDB + regression \\ 
  Chumashius balchi & 4.90 & PBDB + regression \\ 
  Churcheria baroni & 2.70 & PBDB + regression \\ 
  Cimexomys minor & 2.30 & PBDB + regression \\ 
  Cimolestes incisus & 9.10 & PBDB + regression \\ 
  Colodon cingulatus & 282.74 & PBDB \\ 
  Colodon kayi & 104.20 & PBDB + regression \\ 
  Colodon occidentalis & 178.10 & PBDB + regression \\ 
  Colodon stovalli & 115.70 & PBDB + regression \\ 
  Colodon woodi & 150.60 & PBDB + regression \\ 
  Conacodon cophater & 7.40 & PBDB + regression \\ 
  Conacodon delphae & 31.40 & PBDB + regression \\ 
  Conacodon entoconus & 25.50 & PBDB + regression \\ 
  Conacodon kohlbergeri & 12.20 & PBDB + regression \\ 
  Conoryctes comma & 76.10 & PBDB + regression \\ 
  Copecion brachypternus & 37.10 & PBDB + regression \\ 
  Copecion davisi & 28.60 & PBDB + regression \\ 
  Copedelphys innominata & 2.10 & PBDB + regression \\ 
  Copedelphys titanelix & 1.00 & PBDB + regression \\ 
  Copelemur australotutus & 21.30 & PBDB + regression \\ 
  Copelemur tutus & 22.90 & PBDB + regression \\ 
  Copemys barstowensis & 32.14 & cite{Tomiya2013} \\ 
  Copemys esmeraldensis & 27.94 & cite{Tomiya2013} \\ 
  Copemys lindsayi & 14.88 & cite{Tomiya2013} \\ 
  Copemys longidens & 26.84 & cite{Tomiya2013} \\ 
  Copemys loxodon & 28.79 & cite{Tomiya2013} \\ 
  Copemys mariae & 31.50 & cite{Tomiya2013} \\ 
  Copemys pagei & 15.18 & cite{Tomiya2013} \\ 
  Copemys russelli & 24.29 & cite{Tomiya2013} \\ 
  Copemys shotwelli & 1.10 & PBDB + regression \\ 
  Copemys tenuis & 23.34 & cite{Tomiya2013} \\ 
  Coriphagus encinensis & 10.40 & PBDB + regression \\ 
  Coriphagus montanus & 6.90 & PBDB + regression \\ 
  Cormocyon copei & 4817.45 & cite{Tomiya2013} \\ 
  Cormocyon haydeni & 4359.01 & cite{Tomiya2013} \\ 
  Coryphodon armatus & 882.70 & PBDB + regression \\ 
  Coryphodon eocaenus & 659.70 & PBDB + regression \\ 
  Coryphodon lobatus & 1189.40 & PBDB + regression \\ 
  Coryphodon proterus & 1068.80 & PBDB + regression \\ 
  Coryphodon radians & 864.30 & PBDB + regression \\ 
  Cosoryx cerroensis & 16814.55 & cite{Tomiya2013} \\ 
  Cosoryx furcatus & 44.87 & PBDB \\ 
  Cranioceras clarendonensis & 18.90 & PBDB + regression \\ 
  Cranioceras teres & 96761.07 & cite{Tomiya2013} \\ 
  Cranioceras unicornis & 134591.56 & cite{Tomiya2013} \\ 
  Cratogeomys sansimonensis & 4.80 & PBDB + regression \\ 
  Crucimys milleri & 36.60 & cite{Tomiya2013} \\ 
  Crypholestes vaughni & 2.09 & PBDB \\ 
  Cryptotis adamsi & 13.33 & cite{Tomiya2013} \\ 
  Cryptotis kansasensis & 3.00 & PBDB + regression \\ 
  Cryptotis parva & 4.10 & PBDB \\ 
  Cupidinimus avawatzensis & 20.09 & cite{Tomiya2013} \\ 
  Cupidinimus bidahochiensis & 27.39 & cite{Tomiya2013} \\ 
  Cupidinimus boronensis & 14.59 & cite{Tomiya2013} \\ 
  Cupidinimus eurekensis & 0.54 & PBDB \\ 
  Cupidinimus halli & 1.10 & PBDB + regression \\ 
  Cupidinimus lindsayi & 7.61 & cite{Tomiya2013} \\ 
  Cupidinimus madisonensis & 1.11 & PBDB \\ 
  Cupidinimus magnus & 1.70 & PBDB + regression \\ 
  Cupidinimus nebraskensis & 9.30 & cite{Tomiya2013} \\ 
  Cupidinimus prattensis & 21.33 & cite{Tomiya2013} \\ 
  Cupidinimus tertius & 16.44 & cite{Tomiya2013} \\ 
  Cupidinimus whitlocki & 16.78 & cite{Tomiya2013} \\ 
  Cuvieronius tropicus & 10934.90 & PBDB + regression \\ 
  Cuyamalagus dawsoni & 639.06 & cite{Tomiya2013} \\ 
  Cylindrodon fontis & 3.70 & PBDB + regression \\ 
  Cylindrodon nebraskensis & 5.30 & PBDB + regression \\ 
  Cynarctoides acridens & 2921.93 & cite{Tomiya2013} \\ 
  Cynarctoides gawnae & 2643.87 & cite{Tomiya2013} \\ 
  Cynarctoides harlowi & 1863.11 & cite{Tomiya2013} \\ 
  Cynarctoides lemur & 2321.57 & cite{Tomiya2013} \\ 
  Cynarctoides luskensis & 2392.27 & cite{Tomiya2013} \\ 
  Cynarctoides roii & 1826.21 & cite{Tomiya2013} \\ 
  Cynarctus crucidens & 4964.16 & cite{Tomiya2013} \\ 
  Cynarctus galushai & 9228.02 & cite{Tomiya2013} \\ 
  Cynarctus saxatilis & 10097.06 & cite{Tomiya2013} \\ 
  Cynelos caroniavorus & 16647.24 & cite{Tomiya2013} \\ 
  Cynelos idoneus & 105873.47 & cite{Tomiya2013} \\ 
  Cynelos sinapius & 213202.99 & cite{Tomiya2013} \\ 
  Cynodesmus martini & 14185.85 & cite{Tomiya2013} \\ 
  Cynodesmus thooides & 9228.02 & cite{Tomiya2013} \\ 
  Cynorca occidentale & 20537.34 & cite{Tomiya2013} \\ 
  Cynorca sociale & 82.80 & PBDB + regression \\ 
  Cyriacotherium psamminum & 70.30 & PBDB \\ 
  Dakotallomys lillegraveni & 5.81 & PBDB \\ 
  Dakotallomys pelycomyoides & 6.57 & cite{Mihlbachler2006} \\ 
  Daphoenodon falkenbachi & 137310.49 & cite{Tomiya2013} \\ 
  Daphoenodon notionastes & 43477.55 & cite{Tomiya2013} \\ 
  Daphoenodon superbus & 77652.58 & cite{Tomiya2013} \\ 
  Daphoenus hartshornianus & 13.90 & PBDB + regression \\ 
  Daphoenus lambei & 11.40 & PBDB + regression \\ 
  Daphoenus ruber & 13.70 & PBDB + regression \\ 
  Daphoenus socialis & 13000.00 & cite{McKenna2011} \\ 
  Daphoenus vetus & 19535.72 & cite{Tomiya2013} \\ 
  Dartonius jepseni & 1.47 & cite{Hay1969} \\ 
  Dasypus bellus & 104.90 & cite{Smith2004} \\ 
  Delahomeryx browni & 38948.67 & cite{Tomiya2013} \\ 
  Desmatippus avus & 179871.86 & cite{Tomiya2013} \\ 
  Desmatippus texanus & 169.36 & PBDB \\ 
  Desmatochoerus hesperus & 296.24 & cite{McGrew1939} \\ 
  Desmatochoerus megalodon & 335100.00 & cite{McKenna2011} \\ 
  Desmatoclaenus hermaeus & 61.70 & PBDB + regression \\ 
  Desmatolagus schizopetrus & 6.72 & PBDB \\ 
  Desmocyon matthewi & 8103.08 & cite{Tomiya2013} \\ 
  Desmocyon thomsoni & 6974.39 & cite{Tomiya2013} \\ 
  Diacocherus meizon & 2.10 & PBDB + regression \\ 
  Diacocherus minutus & 3.00 & PBDB + regression \\ 
  Diacodexis gracilis & 14.30 & PBDB + regression \\ 
  Diacodexis kelleyi & 20.70 & PBDB + regression \\ 
  Diacodexis metsiacus & 17.80 & PBDB + regression \\ 
  Diacodexis minutus & 11.80 & PBDB + regression \\ 
  Diacodexis primus & 19.50 & PBDB + regression \\ 
  Diacodexis secans & 24.60 & PBDB + regression \\ 
  Diacodon alticuspis & 8.80 & PBDB + regression \\ 
  Diceratherium annectens & 864580.76 & cite{Tomiya2013} \\ 
  Diceratherium armatum & 3541284.24 & cite{Tomiya2013} \\ 
  Diceratherium gregorii & 734.10 & PBDB + regression \\ 
  Diceratherium niobrarense & 2105366.25 & cite{Tomiya2013} \\ 
  Diceratherium tridactylum & 965112.54 & cite{Tomiya2013} \\ 
  Didelphodus absarokae & 12.90 & PBDB + regression \\ 
  Didelphodus altidens & 10.20 & PBDB + regression \\ 
  Didelphodus rheos & 7.20 & PBDB + regression \\ 
  Didelphodus serus & 7.00 & PBDB + regression \\ 
  Didymictis altidens & 105.86 & PBDB \\ 
  Didymictis leptomylus & 8.61 & PBDB + regression \\ 
  Didymictis protenus & 11.30 & PBDB + regression \\ 
  Didymictis proteus & 8.95 & PBDB + regression \\ 
  Didymictis vancleveae & 160.55 & cite{Scott2004} \\ 
  Dikkomys matthewi & 47.47 & cite{Tomiya2013} \\ 
  Dillerlemur pagei & 5.40 & PBDB + regression \\ 
  Dilophodon destitutus & 29.93 & cite{Sinclair1915} \\ 
  Dilophodon minusculus & 70.40 & PBDB + regression \\ 
  Dinaelurus crassus & 37320.00 & cite{McKenna2011} \\ 
  Dinofelis palaeoonca & 24.30 & PBDB + regression \\ 
  Dinohippus interpolatus & 257815.63 & cite{Tomiya2013} \\ 
  Dinohippus leardi & 392385.48 & cite{Tomiya2013} \\ 
  Dinohippus leidyanus & 229900.00 & cite{MacFadden1986} \\ 
  Dinohippus mexicanus & 609259.77 & cite{Tomiya2013} \\ 
  Dinohippus spectans & 536500.00 & cite{McKenna2011} \\ 
  Dinohyus hollandi & 1215.50 & PBDB + regression \\ 
  Diplobunops matthewi & 167.50 & PBDB + regression \\ 
  Dipodomys compactus & 5.43 & cite{Smith2004} \\ 
  Dipodomys gidleyi & 1.50 & PBDB + regression \\ 
  Dipodomys hibbardi & 17.81 & cite{Tomiya2013} \\ 
  Diprionomys agrarius & 2.00 & PBDB + regression \\ 
  Diprionomys parvus & 11.00 & cite{McKenna2011} \\ 
  Dipsalidictis aequidens & 126.40 & PBDB + regression \\ 
  Dipsalidictis platypus & 54.00 & PBDB + regression \\ 
  Dipsalidictis transiens & 89.10 & PBDB + regression \\ 
  Dipsalodon churchillorum & 98.20 & PBDB + regression \\ 
  Dipsalodon matthewi & 162.00 & PBDB + regression \\ 
  Dissacus navajovius & 10.80 & PBDB + regression \\ 
  Dissacus praenuntius & 14.40 & PBDB + regression \\ 
  Domnina dakotensis & 28.50 & cite{Tomiya2013} \\ 
  Domnina gradata & 36.60 & cite{Tomiya2013} \\ 
  Domnina greeni & 33.78 & cite{Tomiya2013} \\ 
  Domnina thompsoni & 2.20 & PBDB + regression \\ 
  Domninoides hessei & 149.90 & cite{Tomiya2013} \\ 
  Domninoides knoxjonesi & 2.88 & PBDB \\ 
  Domninoides mimicus & 135.64 & cite{Tomiya2013} \\ 
  Domninoides riparensis & 56.83 & cite{Tomiya2013} \\ 
  Dorraletes diminutivus & 3.20 & PBDB + regression \\ 
  Douglassciurus jeffersoni & 8.19 & cite{Mihlbachler2006} \\ 
  Downsimus chadwicki & 56.26 & cite{Tomiya2013} \\ 
  Drepanomeryx falciformis & 90219.42 & cite{Tomiya2013} \\ 
  Dromomeryx borealis & 144350.55 & cite{Tomiya2013} \\ 
  Dryomomys dulcifer & 0.85 & cite{Novacek1977} \\ 
  Dryomomys szalayi & 1.00 & cite{Novacek1977} \\ 
  Duchesneodus uintensis & 1358.50 & PBDB + regression \\ 
  Dyseohyus fricki & 21807.30 & cite{Tomiya2013} \\ 
  Dyseolemur pacificus & 3.80 & PBDB + regression \\ 
  Ectocion cedrus & 38.60 & PBDB + regression \\ 
  Ectocion collinus & 42.90 & PBDB + regression \\ 
  Ectocion major & 64.70 & PBDB + regression \\ 
  Ectocion mediotuber & 41.20 & PBDB + regression \\ 
  Ectocion osbornianus & 46.50 & PBDB + regression \\ 
  Ectocion parvus & 30.30 & PBDB + regression \\ 
  Ectocion superstes & 68.50 & PBDB + regression \\ 
  Ectoconus ditrigonus & 89.20 & PBDB + regression \\ 
  Ectoganus gliriformes & 140.40 & cite{Novacek1977} \\ 
  Ectoganus gliriformis & 140.40 & PBDB + regression \\ 
  Ectopocynus antiquus & 8266.78 & cite{Tomiya2013} \\ 
  Ectopocynus intermedius & 12456.53 & cite{Tomiya2013} \\ 
  Ectypodus aphronorus & 18.00 & cite{Wilson2012} \\ 
  Ectypodus childei & 1.60 & PBDB + regression \\ 
  Ectypodus lovei & 1.60 & PBDB + regression \\ 
  Ectypodus musculus & 2.40 & PBDB + regression \\ 
  Ectypodus powelli & 1.90 & PBDB + regression \\ 
  Ectypodus szalayi & 1.80 & PBDB + regression \\ 
  Ectypodus tardus & 1.70 & PBDB + regression \\ 
  Edworthia lerbekmoi & 2.10 & PBDB + regression \\ 
  Ekgmowechashala philotau & 2079.74 & cite{Tomiya2013} \\ 
  Elomeryx armatus & 157944.66 & cite{Tomiya2013} \\ 
  Elphidotarsius florencae & 1.80 & PBDB + regression \\ 
  Elphidotarsius russelli & 2.10 & PBDB + regression \\ 
  Elphidotarsius wightoni & 2.30 & PBDB + regression \\ 
  Elpidophorus elegans & 14.70 & PBDB + regression \\ 
  Elpidophorus minor & 5.80 & PBDB + regression \\ 
  Elwynella oreas & 4.60 & PBDB + regression \\ 
  Elymys complexus & 0.80 & PBDB + regression \\ 
  Enhydritherium terraenovae & 16.50 & PBDB + regression \\ 
  Enhydrocyon basilatus & 20332.99 & cite{Tomiya2013} \\ 
  Enhydrocyon crassidens & 18582.95 & cite{Tomiya2013} \\ 
  Enhydrocyon pahinsintewakpa & 14764.78 & cite{Tomiya2013} \\ 
  Enhydrocyon stenocephalus & 14044.69 & cite{Tomiya2013} \\ 
  Entoptychus grandiplanus & 59.74 & cite{Tomiya2013} \\ 
  Entoptychus individens & 84.00 & cite{McKenna2011} \\ 
  Entoptychus planifrons & 134.29 & cite{Tomiya2013} \\ 
  Entoptychus sheppardi & 94.63 & cite{Tomiya2013} \\ 
  Entoptychus wheelerensis & 84.00 & cite{McKenna2011} \\ 
  Eoconodon gaudrianus & 8.50 & cite{Zonneveld2003} \\ 
  Eoconodon hutchisoni & 55.68 & PBDB \\ 
  Eoconodon nidhoggi & 6.90 & PBDB + regression \\ 
  Eohaplomys matutinus & 18.60 & PBDB + regression \\ 
  Eohaplomys serus & 14.20 & PBDB + regression \\ 
  Eohaplomys tradux & 12.60 & PBDB + regression \\ 
  Eomoropus amarorum & 128.80 & PBDB + regression \\ 
  Eoryctes melanus & 4.00 & PBDB + regression \\ 
  Eotitanops borealis & 297.00 & PBDB + regression \\ 
  Eotitanops minimus & 14.90 & PBDB + regression \\ 
  Eotitanotherium osborni & 988.00 & PBDB \\ 
  Eotylopus reedi & 88.90 & PBDB + regression \\ 
  Epeiromys spanius & 4.00 & PBDB + regression \\ 
  Epicyon haydeni & 41772.77 & cite{Tomiya2013} \\ 
  Epicyon saevus & 27722.51 & cite{Tomiya2013} \\ 
  Epihippus gracilis & 54.50 & PBDB + regression \\ 
  Epihippus intermedius & 61.09 & PBDB \\ 
  Epitriplopus uintensis & 32600.00 & cite{MacFadden1986} \\ 
  Eporeodon occidentalis & 118300.00 & cite{McKenna2011} \\ 
  Equus complicatus & 270.97 & cite{Smith2004} \\ 
  Equus conversidens & 241.29 & cite{Smith2004} \\ 
  Equus cumminsii & 354.80 & PBDB + regression \\ 
  Equus francisci & 278.00 & PBDB + regression \\ 
  Equus fromanius & 172311.00 & cite{McKenna2011} \\ 
  Equus giganteus & 270.97 & cite{Smith2004} \\ 
  Equus idahoensis & 560.70 & PBDB + regression \\ 
  Equus leidyi & 337.30 & PBDB + regression \\ 
  Equus occidentalis & 317.03 & cite{Smith2004} \\ 
  Equus scotti & 312.31 & cite{Smith2004} \\ 
  Equus simplicidens & 296558.57 & cite{Tomiya2013} \\ 
  Eremotherium eomigrans & 2584400.00 & cite{McDonald1995} \\ 
  Eremotherium laurillardi & 366.13 & cite{Smith2004} \\ 
  Erethizon bathygnathum & 51.10 & PBDB + regression \\ 
  Erethizon kleini & 31.30 & PBDB + regression \\ 
  Escavadodon zygus & 5.50 & PBDB + regression \\ 
  Esthonyx acutidens & 80.80 & PBDB + regression \\ 
  Esthonyx bisulcatus & 48.00 & PBDB + regression \\ 
  Esthonyx spatularius & 60.80 & PBDB + regression \\ 
  Eucyon davisi & 10509.13 & cite{Tomiya2013} \\ 
  Eudaemonema cuspidata & 11.50 & PBDB + regression \\ 
  Eumys brachyodus & 126.47 & cite{Tomiya2013} \\ 
  Eumys elegans & 4.10 & PBDB + regression \\ 
  Euoplocyon brachygnathus & 11271.13 & cite{Tomiya2013} \\ 
  Euoplocyon spissidens & 9798.65 & cite{Tomiya2013} \\ 
  Eusmilus cerebralis & 804.32 & cite{Tomiya2013} \\ 
  Eusmilus sicarius & 19.20 & PBDB + regression \\ 
  Eutypomys acares & 3.60 & PBDB + regression \\ 
  Eutypomys hibernodus & 12.20 & PBDB + regression \\ 
  Eutypomys inexpectatus & 12.00 & PBDB + regression \\ 
  Eutypomys montanensis & 943.88 & cite{Tomiya2013} \\ 
  Eutypomys obliquidens & 5.90 & PBDB + regression \\ 
  Eutypomys parvus & 7.00 & PBDB + regression \\ 
  Eutypomys thomsoni & 10.70 & PBDB + regression \\ 
  Fanimus clasoni & 3.67 & PBDB \\ 
  Fanimus ultimus & 164.02 & cite{Tomiya2013} \\ 
  Felis rexroadensis & 30333.26 & cite{Tomiya2013} \\ 
  Ferinestrix vorax & 103.50 & PBDB \\ 
  Florentiamys agnewi & 84.77 & cite{Tomiya2013} \\ 
  Florentiamys kinseyi & 157.59 & cite{Tomiya2013} \\ 
  Florentiamys loomisi & 151.41 & cite{Tomiya2013} \\ 
  Florentiamys tiptoni & 117.92 & cite{Tomiya2013} \\ 
  Floridachoerus olseni & 35596.41 & cite{Tomiya2013} \\ 
  Floridameryx floridanus & 8184.52 & cite{Tomiya2013} \\ 
  Floridatragulus dolichanthereus & 43477.55 & cite{Tomiya2013} \\ 
  Floridatragulus texanus & 61083.68 & cite{Tomiya2013} \\ 
  Fouchia elyensis & 36.80 & PBDB + regression \\ 
  Galbreathia bettae & 165.67 & cite{Tomiya2013} \\ 
  Galbreathia novellus & 6.97 & PBDB \\ 
  Galecyon mordax & 23.70 & PBDB + regression \\ 
  Gazinius amplus & 875.00 & cite{Soligo2006} \\ 
  Gazinocyon vulpeculus & 39.60 & PBDB + regression \\ 
  Gelastops joni & 8.50 & PBDB + regression \\ 
  Gelastops parcus & 7.50 & PBDB + regression \\ 
  Geomys carranzai & 2.00 & PBDB + regression \\ 
  Geringia gloveri & 26.84 & cite{Tomiya2013} \\ 
  Geringia mcgregori & 41.26 & cite{Tomiya2013} \\ 
  Gigantocamelus spatulus & 1435.10 & PBDB + regression \\ 
  Glossotherium chapadmalense & 310540.00 & cite{McDonald1995} \\ 
  Glyptotherium arizonae & 789680.00 & cite{McDonald1995} \\ 
  Gomphotherium obscurum & 10056.96 & PBDB \\ 
  Gomphotherium osborni & 117.00 & cite{Wang2014} \\ 
  Goniodontomys disjunctus & 44.26 & cite{Tomiya2013} \\ 
  Gracilocyon winkleri & 13.30 & PBDB + regression \\ 
  Grangeria anarsius & 126.00 & PBDB + regression \\ 
  Gregorymys curtus & 106.70 & cite{Tomiya2013} \\ 
  Gregorymys formosus & 77.48 & cite{Tomiya2013} \\ 
  Gregorymys riograndensis & 43.82 & cite{Tomiya2013} \\ 
  Gripholagomys lavocati & 507.76 & cite{Tomiya2013} \\ 
  Griphomys alecer & 1.40 & PBDB + regression \\ 
  Griphomys toltecus & 1.83 & cite{Mihlbachler2006} \\ 
  Guanajuatomys hibbardi & 5.30 & PBDB + regression \\ 
  Guildayomys hibbardi & 82.27 & cite{Tomiya2013} \\ 
  Hapalodectes anthracinus & 4.09 & PBDB + regression \\ 
  Hapalodectes leptognathus & 4.94 & PBDB + regression \\ 
  Haplaletes disceptatrix & 5.30 & PBDB + regression \\ 
  Haplaletes pelicatus & 6.90 & PBDB + regression \\ 
  Haploconus angustus & 18.00 & PBDB + regression \\ 
  Haplohippus texanus & 63.70 & PBDB + regression \\ 
  Haplolambda quinni & 276.64 & PBDB \\ 
  Haplolambda simpsoni & 595.34 & cite{Tedford1994} \\ 
  Haplomylus bozemanensis & 5.40 & PBDB + regression \\ 
  Haplomylus palustris & 4.80 & PBDB + regression \\ 
  Haplomylus scottianus & 6.40 & PBDB + regression \\ 
  Haplomylus simpsoni & 7.90 & PBDB + regression \\ 
  Haplomylus speirianus & 5.80 & PBDB + regression \\ 
  Haplomys galbreathi & 2.30 & PBDB + regression \\ 
  Haplomys liolophus & 39.00 & cite{McKenna2011} \\ 
  Harpagolestes leotensis & 30.60 & PBDB + regression \\ 
  Harpagolestes uintensis & 28.00 & PBDB + regression \\ 
  Harrymys irvini & 83.93 & cite{Tomiya2013} \\ 
  Harrymys magnus & 50.40 & cite{Tomiya2013} \\ 
  Harrymys woodi & 2.48 & cite{Kirk2011} \\ 
  Heliscomys hatcheri & 0.89 & cite{Wilson2012} \\ 
  Heliscomys ostranderi & 0.60 & PBDB + regression \\ 
  Heliscomys senex & 0.70 & PBDB + regression \\ 
  Heliscomys vetus & 0.70 & PBDB + regression \\ 
  Heliscomys woodi & 1.10 & PBDB + regression \\ 
  Hemiacodon engardae & 11.60 & PBDB + regression \\ 
  Hemiacodon gracilis & 11.90 & PBDB + regression \\ 
  Hemiauchenia gracilis & 225.00 & PBDB + regression \\ 
  Hemiauchenia macrocephala & 154.62 & cite{Smith2004} \\ 
  Hemiauchenia minima & 85819.37 & cite{Tomiya2013} \\ 
  Hemiauchenia vera & 299.50 & PBDB + regression \\ 
  Hemipsalodon grandis & 453.84 & cite{Scott1937} \\ 
  Hemithlaeus harbourae & 18.67 & cite{Loomis1932} \\ 
  Hendryomeryx defordi & 25.31 & PBDB \\ 
  Hendryomeryx esulcatus & 27.80 & PBDB + regression \\ 
  Hendryomeryx wilsoni & 16.50 & PBDB + regression \\ 
  Heptacodon pellionis & 153.70 & PBDB + regression \\ 
  Heptodon calciculus & 69.50 & PBDB + regression \\ 
  Herpetotherium fugax & 1.90 & PBDB + regression \\ 
  Herpetotherium knighti & 3.20 & PBDB + regression \\ 
  Herpetotherium merriami & 2.20 & PBDB + regression \\ 
  Herpetotherium valens & 3.40 & PBDB + regression \\ 
  Herpetotherium youngi & 2.80 & PBDB + regression \\ 
  Hesperhys pinensis & 68186.37 & cite{Tomiya2013} \\ 
  Hesperhys vagrans & 73865.41 & cite{Tomiya2013} \\ 
  Hesperocyon gregarius & 3533.34 & cite{Tomiya2013} \\ 
  Hesperolagomys fluviatilis & 169.02 & cite{Tomiya2013} \\ 
  Hesperolagomys galbreathi & 149.90 & cite{Tomiya2013} \\ 
  Hesperoscalops mcgrewi & 8.60 & PBDB + regression \\ 
  Heteraletes leotanus & 45.60 & PBDB + regression \\ 
  Heteromeryx dispar & 107.00 & PBDB + regression \\ 
  Heteropliohippus hulberti & 302549.45 & cite{Tomiya2013} \\ 
  Hexacodus pelodes & 12.60 & PBDB + regression \\ 
  Hexameryx simpsoni & 30638.11 & cite{Tomiya2013} \\ 
  Hexobelomeryx fricki & 118.50 & PBDB + regression \\ 
  Hibbarderix obfuscatus & 3.00 & PBDB + regression \\ 
  Hibbardomys fayae & 3.00 & PBDB + regression \\ 
  Hibbardomys marthae & 3.60 & PBDB + regression \\ 
  Hibbardomys skinneri & 3.00 & PBDB + regression \\ 
  Hibbardomys voorhiesi & 3.00 & PBDB + regression \\ 
  Hipparion forcei & 194852.86 & cite{Tomiya2013} \\ 
  Hipparion tehonense & 100709.96 & cite{Tomiya2013} \\ 
  Hippotherium emsliei & 264.70 & PBDB + regression \\ 
  Hippotherium ingenuum & 176.70 & PBDB + regression \\ 
  Hippotherium plicatile & 232.00 & PBDB + regression \\ 
  Hippotherium quinni & 431.00 & PBDB + regression \\ 
  Hitonkala macdonaldtau & 24.53 & cite{Tomiya2013} \\ 
  Holmesina floridanus & 164.90 & PBDB + regression \\ 
  Homacodon vagans & 28.00 & PBDB + regression \\ 
  Homogalax protapirinus & 57.30 & PBDB + regression \\ 
  Homotherium crusafonti & 322.50 & PBDB \\ 
  Homotherium idahoensis & 257900.00 & cite{McKenna2011} \\ 
  Homotherium johnstoni & 29.50 & PBDB + regression \\ 
  Hoplophoneus mentalis & 206.00 & PBDB + regression \\ 
  Hoplophoneus primaevus & 16.00 & PBDB + regression \\ 
  Huerfanodon polecatensis & 104.30 & PBDB + regression \\ 
  Huerfanodon torrejonius & 85.80 & PBDB + regression \\ 
  Hyaenodon brevirostrus & 118.10 & PBDB + regression \\ 
  Hyaenodon crucians & 71.10 & PBDB + regression \\ 
  Hyaenodon horridus & 193.20 & PBDB + regression \\ 
  Hyaenodon montanus & 101.60 & PBDB + regression \\ 
  Hyaenodon mustelinus & 51.10 & PBDB + regression \\ 
  Hyaenodon raineyi & 17.10 & PBDB + regression \\ 
  Hyaenodon venturae & 27.20 & PBDB + regression \\ 
  Hyaenodon vetus & 111.00 & PBDB + regression \\ 
  Hylomeryx quadricuspis & 23.00 & PBDB + regression \\ 
  Hyopsodus fastigatus & 23.10 & PBDB + regression \\ 
  Hyopsodus lepidus & 12.60 & PBDB + regression \\ 
  Hyopsodus loomisi & 11.00 & PBDB + regression \\ 
  Hyopsodus lysitensis & 15.60 & PBDB + regression \\ 
  Hyopsodus mentalis & 20.20 & PBDB + regression \\ 
  Hyopsodus minor & 10.40 & PBDB + regression \\ 
  Hyopsodus minusculus & 10.40 & PBDB + regression \\ 
  Hyopsodus paulus & 14.10 & PBDB + regression \\ 
  Hyopsodus pauxillus & 6.40 & PBDB + regression \\ 
  Hyopsodus powellianus & 19.70 & PBDB + regression \\ 
  Hyopsodus simplex & 13.70 & PBDB + regression \\ 
  Hyopsodus tonksi & 10.70 & PBDB + regression \\ 
  Hyopsodus uintensis & 20.30 & PBDB + regression \\ 
  Hyopsodus walcottianus & 44.10 & PBDB + regression \\ 
  Hyopsodus wortmani & 14.40 & PBDB + regression \\ 
  Hypertragulus calcaratus & 8880.00 & cite{McKenna2011} \\ 
  Hypertragulus hesperius & 4572.00 & cite{McKenna2011} \\ 
  Hypohippus affinis & 442413.39 & cite{Tomiya2013} \\ 
  Hypohippus equinus & 271034.12 & cite{Tomiya2013} \\ 
  Hypohippus osborni & 299539.03 & cite{Tomiya2013} \\ 
  Hypolagus edensis & 665.14 & cite{Tomiya2013} \\ 
  Hypolagus fontinalis & 1211.97 & cite{Tomiya2013} \\ 
  Hypolagus furlongi & 678.58 & cite{Tomiya2013} \\ 
  Hypolagus gidleyi & 1998.20 & cite{Tomiya2013} \\ 
  Hypolagus oregonensis & 2344.90 & cite{Tomiya2013} \\ 
  Hypolagus parviplicatus & 1702.75 & cite{Tomiya2013} \\ 
  Hypolagus ringoldensis & 319.00 & cite{McKenna2011} \\ 
  Hypolagus vetus & 2892.86 & cite{Tomiya2013} \\ 
  Hypolagus voorhiesi & 319.00 & cite{McKenna2011} \\ 
  Hypsiops bannackensis & 214.80 & cite{Wang1999} \\ 
  Hypsiops breviceps & 156373.08 & cite{Tomiya2013} \\ 
  Hyrachyus affinis & 178.70 & PBDB + regression \\ 
  Hyrachyus eximius & 262.00 & PBDB + regression \\ 
  Hyrachyus modestus & 278.50 & PBDB + regression \\ 
  Hyracodon leidyanus & 455.80 & PBDB + regression \\ 
  Hyracodon medius & 169.40 & PBDB + regression \\ 
  Hyracodon nebraskensis & 211081.59 & cite{Tomiya2013} \\ 
  Hyracodon petersoni & 354.50 & PBDB + regression \\ 
  Hyracodon priscidens & 381.90 & PBDB + regression \\ 
  Hyracotherium cristatum & 64.84 & PBDB \\ 
  Hyracotherium vasacciense & 24900.00 & cite{MacFadden1986} \\ 
  Ictidopappus mustelinus & 4.10 & PBDB + regression \\ 
  Ignacius fremontensis & 1.90 & PBDB + regression \\ 
  Ignacius frugivorus & 3.90 & PBDB + regression \\ 
  Ignacius graybullianus & 5.30 & PBDB + regression \\ 
  Indarctos nevadensis & 42.80 & PBDB + regression \\ 
  Indarctos oregonensis & 302549.45 & cite{Tomiya2013} \\ 
  Ischyrocyon gidleyi & 282095.23 & cite{Tomiya2013} \\ 
  Ischyromys typus & 12.50 & PBDB + regression \\ 
  Ischyromys veterior & 9.60 & PBDB + regression \\ 
  Jacobsomys verdensis & 50.91 & cite{Tomiya2013} \\ 
  Janimus dawsonae & 1.50 & PBDB + regression \\ 
  Jaywilsonomys ojinagaensis & 13.70 & PBDB + regression \\ 
  Jemezius szalayi & 4.50 & PBDB + regression \\ 
  Jepsenella praepropera & 2.10 & PBDB + regression \\ 
  Jimomys labaughi & 69.41 & cite{Tomiya2013} \\ 
  Jimomys lulli & 28.00 & cite{McKenna2011} \\ 
  Kansasimys dubius & 135.64 & cite{Tomiya2013} \\ 
  Kansasimys wilsoni & 6.33 & cite{Matthew1901} \\ 
  Kimbetohia mziae & 72.00 & cite{Wilson2012} \\ 
  Knightomys cremneus & 1.90 & PBDB + regression \\ 
  Knightomys cuspidatus & 2.62 & PBDB \\ 
  Knightomys depressus & 3.10 & PBDB + regression \\ 
  Knightomys huerfanensis & 4.70 & PBDB + regression \\ 
  Knightomys minor & 1.90 & PBDB + regression \\ 
  Knightomys reginensis & 1.10 & PBDB + regression \\ 
  Knightomys senior & 2.50 & PBDB + regression \\ 
  Kolponomos clallamensis & 253.10 & cite{Tseng2009} \\ 
  Kolponomos newportensis & 664.45 & cite{Scott2013} \\ 
  Kyptoceras amatorum & 369534.73 & cite{Tomiya2013} \\ 
  Labidolemur kayi & 3.00 & PBDB + regression \\ 
  Labidolemur serus & 4.60 & PBDB + regression \\ 
  Labidolemur soricoides & 1.80 & PBDB + regression \\ 
  Lambdotherium popoagicum & 74.70 & PBDB + regression \\ 
  Lambertocyon eximius & 43.60 & PBDB + regression \\ 
  Lambertocyon ischyrus & 23.90 & PBDB + regression \\ 
  Lanthanotherium sawini & 5.50 & PBDB + regression \\ 
  Laredochoerus edwardsi & 41.50 & PBDB + regression \\ 
  Laredomys riograndensis & 0.80 & PBDB + regression \\ 
  Leidymys cerasus & 41.68 & cite{Tomiya2013} \\ 
  Leidymys korthi & 2.20 & PBDB + regression \\ 
  Leipsanolestes siegfriedti & 2.60 & PBDB + regression \\ 
  Lemoynea biradicularis & 45.60 & cite{Tomiya2013} \\ 
  Lepoides lepoides & 3640.95 & cite{Tomiya2013} \\ 
  Leptacodon munusculum & 1.10 & PBDB + regression \\ 
  Leptacodon packi & 1.80 & PBDB + regression \\ 
  Leptacodon tener & 1.30 & PBDB + regression \\ 
  Leptarctus mummorum & 129.10 & PBDB + regression \\ 
  Leptarctus oregonensis & 2580.00 & cite{McKenna2011} \\ 
  Leptarctus primus & 149.90 & cite{Tomiya2013} \\ 
  Leptarctus supremus & 125.10 & PBDB \\ 
  Leptauchenia decora & 20130.67 & cite{Tomiya2013} \\ 
  Leptauchenia major & 30946.03 & cite{Tomiya2013} \\ 
  Leptictis dakotensis & 63.70 & PBDB + regression \\ 
  Leptochoerus elegans & 33.10 & PBDB + regression \\ 
  Leptochoerus spectabilis & 34.10 & PBDB + regression \\ 
  Leptochoerus supremus & 56.50 & PBDB + regression \\ 
  Leptocyon mollis & 3300.00 & cite{McKenna2011} \\ 
  Leptocyon vafer & 5377.61 & cite{Tomiya2013} \\ 
  Leptodontomys douglassi & 0.70 & PBDB + regression \\ 
  Leptodontomys stirtoni & 0.90 & cite{Stock1937} \\ 
  Leptolambda schmidti & 383.51 & cite{Zack2005} \\ 
  Leptomeryx blacki & 21.80 & PBDB + regression \\ 
  Leptomeryx elissae & 27.50 & PBDB + regression \\ 
  Leptomeryx evansi & 35.10 & PBDB + regression \\ 
  Leptomeryx mammifer & 55.40 & PBDB + regression \\ 
  Leptomeryx speciosus & 34.60 & PBDB + regression \\ 
  Leptomeryx yoderi & 38.20 & PBDB + regression \\ 
  Leptoreodon edwardsi & 44.00 & PBDB + regression \\ 
  Leptoreodon leptolophus & 34.60 & PBDB + regression \\ 
  Leptoreodon major & 79.60 & PBDB + regression \\ 
  Leptoreodon marshi & 58.90 & PBDB + regression \\ 
  Leptoreodon pusillus & 19.80 & PBDB + regression \\ 
  Leptoreodon stocki & 59.50 & PBDB + regression \\ 
  Leptoromys wilsoni & 4.20 & PBDB + regression \\ 
  Leptotomus caryophilus & 13.50 & PBDB + regression \\ 
  Leptotomus leptodus & 26.40 & PBDB + regression \\ 
  Leptotomus parvus & 14.10 & PBDB + regression \\ 
  Leptotragulus clarki & 24.20 & PBDB + regression \\ 
  Leptotragulus medius & 44.10 & PBDB + regression \\ 
  Leptotragulus proavus & 40.70 & PBDB + regression \\ 
  Lepus californicus & 2288.00 & PBDB \\ 
  Lignimus austridakotensis & 14.88 & cite{Tomiya2013} \\ 
  Lignimus montis & 2.30 & PBDB + regression \\ 
  Limaconyssus habrus & 2.10 & PBDB + regression \\ 
  Limnoecus niobrarensis & 7.17 & cite{Tomiya2013} \\ 
  Limnoecus tricuspis & 5.10 & cite{Tomiya2013} \\ 
  Liodontia alexandrae & 198.34 & cite{Tomiya2013} \\ 
  Litaletes disjunctus & 13.20 & PBDB + regression \\ 
  Litocherus lacunatus & 5.30 & PBDB + regression \\ 
  Litocherus notissimus & 3.20 & PBDB + regression \\ 
  Litocherus zygeus & 4.10 & PBDB + regression \\ 
  Litolagus molidens & 4.70 & PBDB + regression \\ 
  Litolestes ignotus & 2.10 & PBDB + regression \\ 
  Litomylus dissentaneus & 6.70 & PBDB + regression \\ 
  Litomylus orthronepius & 6.40 & PBDB + regression \\ 
  Litoyoderimys auogoleus & 2.60 & PBDB + regression \\ 
  Longirostromeryx clarendonensis & 13226.80 & cite{Tomiya2013} \\ 
  Longirostromeryx wellsi & 17500.77 & cite{Tomiya2013} \\ 
  Lophiparamys debequensis & 2.00 & PBDB + regression \\ 
  Lophiparamys murinus & 3.70 & PBDB + regression \\ 
  Loveina minuta & 3.00 & PBDB + regression \\ 
  Loveina zephyri & 3.70 & PBDB + regression \\ 
  Loxolophus criswelli & 24.70 & PBDB + regression \\ 
  Loxolophus hyattianus & 23.70 & PBDB + regression \\ 
  Loxolophus pentacus & 71.30 & PBDB + regression \\ 
  Loxolophus priscus & 34.30 & PBDB + regression \\ 
  Loxolophus schizophrenus & 15.20 & PBDB + regression \\ 
  Loxolophus spiekeri & 80.00 & PBDB + regression \\ 
  Lutravus halli & 454.86 & cite{Tomiya2013} \\ 
  Lycophocyon hutchisoni & 9.90 & PBDB + regression \\ 
  Lynx proterolyncis & 15677.78 & cite{Tomiya2013} \\ 
  Lynx rufus & 10482.00 & PBDB \\ 
  Machaeromeryx gilchristensis & 4536.90 & cite{Tomiya2013} \\ 
  Machaeromeryx tragulus & 4536.90 & cite{Tomiya2013} \\ 
  Macrocranion junnei & 1.90 & PBDB + regression \\ 
  Macrocranion nitens & 3.00 & PBDB + regression \\ 
  Macrogenis crassigenis & 174.00 & PBDB \\ 
  Macrognathomys gemmacolis & 0.80 & PBDB + regression \\ 
  Macrognathomys nanus & 5.16 & cite{Tomiya2013} \\ 
  Macrotarsius montanus & 17.60 & PBDB + regression \\ 
  Macrotarsius roederi & 1640.00 & cite{Soligo2006} \\ 
  Macrotarsius siegerti & 17.90 & PBDB + regression \\ 
  Mahgarita stevensi & 9.29 & cite{Wortman1893} \\ 
  Malaquiferus tourteloti & 47.30 & PBDB + regression \\ 
  Mammacyon obtusidens & 70262.96 & cite{Tomiya2013} \\ 
  Mammut americanum & 777.43 & cite{Smith2004} \\ 
  Mammut cosoensis & 7029.00 & cite{Secord2008a} \\ 
  Mammut furlongi & 7102.00 & PBDB + regression \\ 
  Mammuthus columbi & 995.26 & cite{Smith2004} \\ 
  Mammut raki & 8929.90 & PBDB + regression \\ 
  Marmota korthi & 2038.56 & cite{Tomiya2013} \\ 
  Marmota vetus & 658.52 & cite{Tomiya2013} \\ 
  Martes gazini & 1767.00 & cite{McKenna2011} \\ 
  Martes parviloba & 42.75 & cite{Macdonald1951} \\ 
  Martinogale alveodens & 17.99 & cite{Tomiya2013} \\ 
  Mattimys kalicola & 2.10 & PBDB + regression \\ 
  Megacamelus merriami & 1905014.16 & cite{Tomiya2013} \\ 
  Megadelphus lundeliusi & 32.90 & PBDB + regression \\ 
  Megahippus matthewi & 882046.45 & cite{Tomiya2013} \\ 
  Megalagus abaconis & 1118.79 & cite{Tomiya2013} \\ 
  Megalagus brachyodon & 9.50 & PBDB + regression \\ 
  Megalagus primitivus & 3165.29 & cite{Tomiya2013} \\ 
  Megalagus turgidus & 3071.74 & cite{Tomiya2013} \\ 
  Megalesthonyx hopsoni & 132.29 & cite{Hay1916} \\ 
  Megalictis ferox & 1587.63 & cite{Tomiya2013} \\ 
  Megalictis frazieri & 111.47 & PBDB \\ 
  Megalonyx curvidens & 185050.00 & cite{McDonald1995} \\ 
  Megalonyx leptostomus & 355.40 & PBDB + regression \\ 
  Megalonyx wheatleyi & 336.20 & PBDB + regression \\ 
  Megantereon hesperus & 67507.91 & cite{Tomiya2013} \\ 
  Megapeomys bobwilsoni & 411.58 & cite{Tomiya2013} \\ 
  Megasminthus tiheni & 55.70 & cite{Tomiya2013} \\ 
  Megatylopus cochrani & 702.00 & PBDB + regression \\ 
  Megatylopus matthewi & 1383324.20 & cite{Tomiya2013} \\ 
  Megatylopus primaevus & 701.90 & PBDB + regression \\ 
  Meniscomys hippodus & 69.41 & cite{Tomiya2013} \\ 
  Meniscomys uhtoffi & 70.00 & cite{McKenna2011} \\ 
  Meniscotherium chamense & 77.80 & PBDB + regression \\ 
  Meniscotherium tapiacitum & 22.90 & PBDB + regression \\ 
  Menoceras arikarense & 597195.61 & cite{Tomiya2013} \\ 
  Menoceras barbouri & 1251683.50 & cite{Tomiya2013} \\ 
  Menops bakeri & 1897.40 & PBDB \\ 
  Mentoclaenodon acrogenius & 118.30 & PBDB + regression \\ 
  Mephitis mephitis & 2055.00 & PBDB \\ 
  Merriamoceros coronatus & 15214.44 & cite{Tomiya2013} \\ 
  Merychippus brevidontus & 124243.67 & cite{Tomiya2013} \\ 
  Merychippus calamarius & 196811.17 & cite{Tomiya2013} \\ 
  Merychippus californicus & 86681.87 & cite{Tomiya2013} \\ 
  Merychippus goorisi & 282.80 & PBDB + regression \\ 
  Merychippus gunteri & 92041.97 & cite{Tomiya2013} \\ 
  Merychippus insignis & 125492.34 & cite{Tomiya2013} \\ 
  Merychippus primus & 95798.28 & cite{Tomiya2013} \\ 
  Merychippus relictus & 60471.00 & cite{McKenna2011} \\ 
  Merychippus sejunctus & 75357.60 & cite{Tomiya2013} \\ 
  Merychyus arenarum & 109.50 & PBDB + regression \\ 
  Merychyus crabilli & 117.00 & cite{Johanson1996} \\ 
  Merychyus elegans & 45706.69 & cite{Tomiya2013} \\ 
  Merychyus medius & 108012.26 & cite{Tomiya2013} \\ 
  Merychyus minimus & 74.70 & PBDB + regression \\ 
  Merychyus relictus & 164.00 & PBDB \\ 
  Merychyus smithi & 203.50 & PBDB + regression \\ 
  Merycochoerus carrikeri & 354.04 & cite{Rose2011a} \\ 
  Merycochoerus chelydra & 336.00 & PBDB \\ 
  Merycochoerus magnus & 331041.82 & cite{Tomiya2013} \\ 
  Merycochoerus matthewi & 275.00 & PBDB + regression \\ 
  Merycochoerus proprius & 265667.29 & cite{Tomiya2013} \\ 
  Merycochoerus superbus & 438400.00 & cite{McKenna2011} \\ 
  Merycodus prodromus & 9.00 & cite{Kelley1954} \\ 
  Merycodus sabulonis & 10509.13 & cite{Tomiya2013} \\ 
  Merycodus warreni & 42.50 & cite{Baskin2004} \\ 
  Merycoides harrisonensis & 90219.42 & cite{Tomiya2013} \\ 
  Merycoides longiceps & 215.00 & cite{Stock1948} \\ 
  Merycoides pariogonus & 77000.00 & cite{McKenna2011} \\ 
  Merycoidodon bullatus & 109500.00 & cite{McKenna2011} \\ 
  Merycoidodon culbertsoni & 184.60 & PBDB + regression \\ 
  Merycoidodon major & 62200.00 & cite{McKenna2011} \\ 
  Mescalerolemur horneri & 6.62 & cite{MacIntyre1966} \\ 
  Mesocyon brachyops & 7942.63 & cite{Tomiya2013} \\ 
  Mesocyon coryphaeus & 10198.54 & cite{Tomiya2013} \\ 
  Mesocyon temnodon & 7186.79 & cite{Tomiya2013} \\ 
  Mesodma ambigua & 5.00 & PBDB + regression \\ 
  Mesodma formosa & 2.50 & PBDB + regression \\ 
  Mesodma garfieldensis & 2.70 & PBDB + regression \\ 
  Mesodma pygmaea & 1.20 & PBDB + regression \\ 
  Mesodma thompsoni & 3.30 & PBDB + regression \\ 
  Mesogaulus paniensis & 13.97 & PBDB \\ 
  Mesohippus bairdi & 89.90 & PBDB + regression \\ 
  Mesohippus exoletus & 102.10 & PBDB + regression \\ 
  Mesohippus texanus & 84.40 & PBDB + regression \\ 
  Mesohippus westoni & 90.10 & PBDB + regression \\ 
  Mesonyx obtusidens & 279.00 & PBDB \\ 
  Mesoreodon chelonyx & 201.25 & cite{Bever2003} \\ 
  Mesoreodon floridensis & 104820.01 & cite{Tomiya2013} \\ 
  Mesoreodon minor & 159700.00 & cite{McKenna2011} \\ 
  Mesoscalops montanensis & 85.63 & cite{Tomiya2013} \\ 
  Mesoscalops scopelotemos & 95.58 & cite{Tomiya2013} \\ 
  Metadjidaumo hendryi & 10.38 & cite{Tomiya2013} \\ 
  Metaliomys sevierensis & 1.40 & PBDB + regression \\ 
  Metanoiamys agorus & 1.20 & PBDB + regression \\ 
  Metanoiamys fantasma & 1.60 & PBDB + regression \\ 
  Metanoiamys korthi & 1.30 & PBDB + regression \\ 
  Metanoiamys lacus & 1.10 & PBDB + regression \\ 
  Metanoiamys marinus & 1.00 & PBDB + regression \\ 
  Metarhinus pater & 810.00 & PBDB \\ 
  Metatomarctus canavus & 10938.02 & cite{Tomiya2013} \\ 
  Metechinus amplior & 24.12 & cite{Tedford1994} \\ 
  Miacis deutschi & 16.00 & PBDB + regression \\ 
  Miacis exiguus & 22.30 & PBDB + regression \\ 
  Miacis hookwayi & 26.00 & cite{Patton1973} \\ 
  Miacis latidens & 25.30 & PBDB + regression \\ 
  Miacis parvivorus & 26.20 & PBDB + regression \\ 
  Miacis petilus & 21.00 & PBDB + regression \\ 
  Michenia agatensis & 99707.88 & cite{Tomiya2013} \\ 
  Michenia exilis & 65512.75 & cite{Tomiya2013} \\ 
  Microcosmodon conus & 1.70 & PBDB + regression \\ 
  Microcosmodon rosei & 2.30 & PBDB + regression \\ 
  Microeutypomys karenae & 1.20 & PBDB + regression \\ 
  Microeutypomys tilliei & 1.70 & PBDB + regression \\ 
  Micromomys antelucanus & 0.90 & PBDB \\ 
  Microparamys cheradius & 2.90 & PBDB + regression \\ 
  Microparamys dubius & 1.60 & PBDB + regression \\ 
  Microparamys hunterae & 1.32 & PBDB \\ 
  Microparamys minutus & 1.80 & PBDB + regression \\ 
  Microparamys nimius & 2.00 & PBDB + regression \\ 
  Microparamys sambucus & 2.40 & PBDB + regression \\ 
  Microparamys tricus & 2.90 & PBDB + regression \\ 
  Microparamys woodi & 2.10 & PBDB + regression \\ 
  Micropternodus borealis & 3.40 & PBDB + regression \\ 
  Micropternodus montrosensis & 6.00 & PBDB + regression \\ 
  Micropternodus morgani & 72.24 & cite{Tomiya2013} \\ 
  Microsus cuspidatus & 15.00 & PBDB + regression \\ 
  Microsyops angustidens & 9.20 & PBDB + regression \\ 
  Microsyops annectens & 20.50 & PBDB + regression \\ 
  Microsyops cardiorestes & 6.50 & PBDB + regression \\ 
  Microsyops elegans & 13.80 & PBDB + regression \\ 
  Microsyops knightensis & 11.40 & PBDB + regression \\ 
  Microsyops kratos & 20.00 & PBDB + regression \\ 
  Microsyops latidens & 10.80 & PBDB + regression \\ 
  Microsyops scottianus & 12.70 & PBDB + regression \\ 
  Microtomarctus conferta & 8866.19 & cite{Tomiya2013} \\ 
  Mictomys vetus & 3.40 & PBDB + regression \\ 
  Mimatuta morgoth & 11.30 & PBDB + regression \\ 
  Mimetodon silberlingi & 2.00 & PBDB + regression \\ 
  Mimomys mcknighti & 110.70 & PBDB \\ 
  Mimomys panacaensis & 3.60 & PBDB + regression \\ 
  Mimomys parvus & 99.10 & PBDB \\ 
  Mimomys primus & 163.60 & PBDB \\ 
  Mimomys taylori & 80.90 & PBDB \\ 
  Mimoperadectes labrus & 8.20 & PBDB + regression \\ 
  Mimotricentes fremontensis & 27.90 & PBDB + regression \\ 
  Mimotricentes subtrigonus & 23.70 & PBDB + regression \\ 
  Miniochoerus affinis & 84.70 & PBDB \\ 
  Miniochoerus gracilis & 48.60 & PBDB \\ 
  Minippus index & 22.55 & PBDB \\ 
  Mioclaenus turgidus & 36.70 & PBDB + regression \\ 
  Miocyon scotti & 125.60 & PBDB + regression \\ 
  Miocyon vallisrubrae & 69.70 & PBDB + regression \\ 
  Mioheteromys amplissimus & 41.26 & cite{Tomiya2013} \\ 
  Miohippus anceps & 41650.00 & cite{McKenna2011} \\ 
  Miohippus assiniboiensis & 142.10 & PBDB + regression \\ 
  Miohippus equiceps & 41650.00 & cite{McKenna2011} \\ 
  Miohippus equinanus & 35954.16 & cite{Tomiya2013} \\ 
  Miohippus gidleyi & 197.80 & PBDB + regression \\ 
  Miohippus grandis & 139.38 & PBDB \\ 
  Miohippus intermedius & 82454.34 & cite{Tomiya2013} \\ 
  Miohippus obliquidens & 52575.21 & cite{Tomiya2013} \\ 
  Miolabis fricki & 240385.70 & cite{Tomiya2013} \\ 
  Miomustela madisonae & 13.46 & cite{Tomiya2013} \\ 
  Mionictis elegans & 11.00 & PBDB \\ 
  Mionictis incertus & 12.30 & PBDB \\ 
  Mionictis letifer & 134.29 & cite{Tomiya2013} \\ 
  Mionictis pristinus & 18.70 & PBDB + regression \\ 
  Miosciurus ballovianus & 36.60 & cite{Tomiya2013} \\ 
  Miosicista angulus & 0.96 & PBDB \\ 
  Miospermophilus lavertyi & 106.70 & cite{Tomiya2013} \\ 
  Miospermophilus wyomingensis & 83.10 & cite{Tomiya2013} \\ 
  Miotapirus harrisonensis & 325.20 & PBDB + regression \\ 
  Miotylopus gibbi & 48050.12 & cite{Tomiya2013} \\ 
  Miotylopus leonardi & 43044.94 & cite{Tomiya2013} \\ 
  Miotylopus taylori & 73865.41 & cite{Tomiya2013} \\ 
  Miracinonyx inexpectatus & 20.40 & PBDB + regression \\ 
  Miracinonyx studeri & 17.30 & PBDB + regression \\ 
  Mithrandir gillianus & 12.80 & PBDB + regression \\ 
  Mixodectes malaris & 11.60 & PBDB + regression \\ 
  Mixodectes pungens & 20.80 & PBDB + regression \\ 
  Mojavemys galushai & 28.22 & cite{Tomiya2013} \\ 
  Mojavemys lophatus & 1.36 & PBDB \\ 
  Mojavemys magnumarcus & 56.26 & cite{Tomiya2013} \\ 
  Montanatylopus matthewi & 242.70 & PBDB + regression \\ 
  Mookomys altifluminis & 13.07 & cite{Tomiya2013} \\ 
  Mookomys formicarum & 1.19 & PBDB \\ 
  Mookomys thrinax & 11.94 & cite{Tomiya2013} \\ 
  Moropus elatus & 707858.86 & cite{Tomiya2013} \\ 
  Moropus merriami & 173767.00 & cite{McKenna2011} \\ 
  Moropus oregonensis & 189094.09 & cite{Tomiya2013} \\ 
  Mustela frenata & 8.73 & cite{Smith2004} \\ 
  Mustela meltoni & 5.60 & PBDB + regression \\ 
  Mustela rexroadensis & 9.58 & cite{Tomiya2013} \\ 
  Mylanodon rosei & 2.30 & PBDB + regression \\ 
  Mylohyus elmorei & 235.60 & PBDB + regression \\ 
  Myrmecoboides montanensis & 3.60 & PBDB + regression \\ 
  Myrmecophaga tridactyla & 32544.00 & PBDB \\ 
  Mysops parvus & 2.80 & PBDB + regression \\ 
  Mystipterus martini & 29.37 & cite{Tomiya2013} \\ 
  Mystipterus pacificus & 14.88 & cite{Tomiya2013} \\ 
  Mytonolagus petersoni & 7.00 & PBDB + regression \\ 
  Mytonolagus wyomingensis & 4.20 & PBDB + regression \\ 
  Mytonomeryx scotti & 24.50 & PBDB + regression \\ 
  Mytonomys burkei & 25.90 & PBDB + regression \\ 
  Mytonomys mytonensis & 16.00 & PBDB + regression \\ 
  Mytonomys robustus & 26.10 & PBDB + regression \\ 
  Namatomys lloydi & 1.80 & PBDB + regression \\ 
  Nannippus aztecus & 173.50 & PBDB + regression \\ 
  Nannippus beckensis & 205.70 & PBDB + regression \\ 
  Nannippus peninsulatus & 68871.66 & cite{Tomiya2013} \\ 
  Nannippus westoni & 62317.65 & cite{Tomiya2013} \\ 
  Nannodectes gidleyi & 11.40 & PBDB + regression \\ 
  Nannodectes intermedius & 5.40 & PBDB + regression \\ 
  Nannodectes simpsoni & 9.90 & PBDB + regression \\ 
  Nanodelphys hunti & 1.10 & PBDB + regression \\ 
  Nanotragulus loomisi & 2724.39 & cite{Tomiya2013} \\ 
  Nanotragulus ordinatus & 5541.39 & cite{Tomiya2013} \\ 
  Nanotragulus planiceps & 3678.00 & cite{McKenna2011} \\ 
  Navajovius kohlhaasae & 1.50 & PBDB + regression \\ 
  Nebraskomys mcgrewi & 46.06 & cite{Tomiya2013} \\ 
  Nebraskomys rexroadensis & 2.60 & PBDB + regression \\ 
  Nekrolagus progressus & 1510.20 & cite{Tomiya2013} \\ 
  Neohipparion affine & 167711.41 & cite{Tomiya2013} \\ 
  Neohipparion eurystyle & 133252.35 & cite{Tomiya2013} \\ 
  Neohipparion leptode & 233281.23 & cite{Tomiya2013} \\ 
  Neohipparion trampasense & 120571.71 & cite{Tomiya2013} \\ 
  Neoliotomus conventus & 23.10 & PBDB + regression \\ 
  Neoliotomus ultimus & 25.40 & PBDB + regression \\ 
  Neoplagiaulax donaldorum & 52.00 & cite{Wilson2012} \\ 
  Neoplagiaulax grangeri & 93.00 & cite{Wilson2012} \\ 
  Neoplagiaulax hazeni & 4.50 & PBDB + regression \\ 
  Neoplagiaulax hunteri & 2.90 & PBDB + regression \\ 
  Neoplagiaulax macrotomeus & 1.60 & PBDB + regression \\ 
  Neoplagiaulax mckennai & 3.50 & PBDB + regression \\ 
  Neotoma cinerea & 11.89 & cite{Smith2004} \\ 
  Neotoma fossilis & 5.20 & PBDB + regression \\ 
  Neotoma leucopetrica & 8.56 & cite{Grohe2010} \\ 
  Neotoma quadriplicata & 9.60 & PBDB + regression \\ 
  Neotoma sawrockensis & 99.48 & cite{Tomiya2013} \\ 
  Neotoma taylori & 5.70 & PBDB + regression \\ 
  Neotoma vaughani & 5.70 & PBDB + regression \\ 
  Nerterogeomys anzensis & 36.20 & PBDB \\ 
  Nerterogeomys garbanii & 43.40 & PBDB \\ 
  Nerterogeomys minor & 2.30 & PBDB + regression \\ 
  Nerterogeomys persimilis & 2.67 & cite{Grohe2010} \\ 
  Nexuotapirus marslandensis & 144350.55 & cite{Tomiya2013} \\ 
  Nexuotapirus robustus & 302549.45 & cite{Tomiya2013} \\ 
  Niglarodon koerneri & 68.03 & cite{Tomiya2013} \\ 
  Nimravides galiani & 159532.03 & cite{Tomiya2013} \\ 
  Nimravides thinobates & 2460.50 & PBDB \\ 
  Nimravus brachyops & 3789.54 & cite{Tomiya2013} \\ 
  Nimravus sectator & 28.50 & PBDB + regression \\ 
  Niptomomys doreenae & 1.80 & PBDB + regression \\ 
  Niptomomys thelmae & 1.90 & PBDB + regression \\ 
  Nonomys gutzleri & 1.60 & PBDB + regression \\ 
  Nonomys simplicidens & 1.40 & PBDB + regression \\ 
  Notharctus pugnax & 29.90 & PBDB + regression \\ 
  Notharctus robinsoni & 31.40 & PBDB + regression \\ 
  Notharctus robustior & 6900.00 & cite{Soligo2006} \\ 
  Notharctus tenebrosus & 36.30 & PBDB + regression \\ 
  Notharctus venticolus & 28.00 & PBDB + regression \\ 
  Nothokemas floridanus & 100709.96 & cite{Tomiya2013} \\ 
  Nothokemas waldropi & 22925.38 & cite{Tomiya2013} \\ 
  Nothotylopus camptognathus & 334368.85 & cite{Tomiya2013} \\ 
  Nothrotheriops shastensis & 613762.01 & cite{Brook2004a} \\ 
  Nothrotheriops texanus & 228.60 & PBDB + regression \\ 
  Notiosorex jacksoni & 1.52 & cite{Madden1985} \\ 
  Notiosorex repenningi & 1.98 & cite{Lillegraven1977} \\ 
  Notiotitanops mississippiensis & 2039.60 & PBDB + regression \\ 
  Notolagus lepusculus & 372.41 & cite{Tomiya2013} \\ 
  Nototamias hulberti & 17.12 & cite{Tomiya2013} \\ 
  Nototamias quadratus & 35.87 & cite{Tomiya2013} \\ 
  Nyctitherium serotinum & 1.60 & PBDB + regression \\ 
  Nyctitherium velox & 2.50 & PBDB + regression \\ 
  Ochotona spanglei & 188.67 & cite{Tomiya2013} \\ 
  Odocoileus virginianus & 52607.00 & PBDB \\ 
  Ogmodontomys poaphagus & 131.63 & cite{Tomiya2013} \\ 
  Ogmodontomys sawrockensis & 140.40 & PBDB \\ 
  Oklahomalagus whisenhunti & 387.61 & cite{Tomiya2013} \\ 
  Oligobunis floridanus & 17.00 & PBDB + regression \\ 
  Oligoryctes cameronensis & 0.90 & PBDB + regression \\ 
  Oligoscalops galbreathi & 2.90 & PBDB + regression \\ 
  Oligospermophilus douglassi & 4.30 & PBDB + regression \\ 
  Omomys carteri & 6.20 & PBDB + regression \\ 
  Omomys lloydi & 3.70 & PBDB + regression \\ 
  Onychomys hollisteri & 1.60 & PBDB + regression \\ 
  Onychomys martini & 18.36 & cite{Tomiya2013} \\ 
  Onychomys pedroensis & 36.23 & cite{Tomiya2013} \\ 
  Oregonomys magnus & 1.60 & PBDB + regression \\ 
  Oregonomys pebblespringsensis & 26.58 & cite{Tomiya2013} \\ 
  Oregonomys sargenti & 26.05 & cite{Tomiya2013} \\ 
  Oreodontoides oregonensis & 25336.47 & cite{Tomiya2013} \\ 
  Oreolagus colteri & 3.10 & PBDB + regression \\ 
  Oreolagus nebrascensis & 336.97 & cite{Tomiya2013} \\ 
  Oreolagus nevadensis & 78.00 & cite{McKenna2011} \\ 
  Oreolagus wallacei & 354.25 & cite{Tomiya2013} \\ 
  Oreotalpa florissantensis & 1.50 & PBDB + regression \\ 
  Orohippus pumilus & 39.20 & PBDB + regression \\ 
  Orohippus sylvaticus & 37.00 & PBDB + regression \\ 
  Oromeryx plicatus & 32.90 & PBDB + regression \\ 
  Oropyctis pediasius & 5.20 & PBDB + regression \\ 
  Osbornoceros osborni & 17326.63 & cite{Tomiya2013} \\ 
  Osbornodon brachypus & 141.90 & cite{Chester2012} \\ 
  Osbornodon fricki & 25848.30 & cite{Tomiya2013} \\ 
  Osbornodon iamonensis & 13904.95 & cite{Tomiya2013} \\ 
  Osbornodon renjiei & 12.33 & PBDB + regression \\ 
  Osbornodon scitulus & 11849.01 & cite{Tomiya2013} \\ 
  Osbornodon sesnoni & 8349.86 & cite{Tomiya2013} \\ 
  Otarocyon cooki & 1826.21 & cite{Tomiya2013} \\ 
  Otarocyon macdonaldi & 6.30 & PBDB + regression \\ 
  Ottoceros peacevalleyensis & 13493.99 & cite{Tomiya2013} \\ 
  Ourayia hopsoni & 12.20 & PBDB + regression \\ 
  Ourayia uintensis & 14.00 & PBDB + regression \\ 
  Oxetocyon cuspidatus & 2440.60 & cite{Tomiya2013} \\ 
  Oxyacodon agapetillus & 9.20 & PBDB + regression \\ 
  Oxyacodon apiculatus & 17.30 & PBDB + regression \\ 
  Oxyacodon ferronensis & 12.50 & PBDB + regression \\ 
  Oxyacodon priscilla & 14.50 & PBDB + regression \\ 
  Oxyaena forcipata & 137.50 & PBDB + regression \\ 
  Oxyaena gulo & 126.30 & PBDB + regression \\ 
  Oxyaena intermedia & 130.30 & PBDB + regression \\ 
  Oxyclaenus cuspidatus & 5.30 & PBDB + regression \\ 
  Oxyclaenus pugnax & 7.10 & PBDB + regression \\ 
  Oxyclaenus simplex & 4.80 & PBDB + regression \\ 
  Oxydactylus longipes & 112420.32 & cite{Tomiya2013} \\ 
  Oxydactylus lulli & 275.00 & PBDB \\ 
  Oxyprimus erikseni & 5.40 & PBDB + regression \\ 
  Pachyaena gigantea & 24.50 & PBDB + regression \\ 
  Pachyaena gracilis & 19.80 & PBDB + regression \\ 
  Pachyaena ossifraga & 380.00 & PBDB \\ 
  Pachyarmatherium leiseyi & 15420.00 & cite{McDonald1995} \\ 
  Paciculus montanus & 3.20 & PBDB + regression \\ 
  Paciculus nebraskensis & 80.64 & cite{Tomiya2013} \\ 
  Paciculus woodi & 2.10 & cite{Korth1993} \\ 
  Paenemarmota barbouri & 10301.04 & cite{Tomiya2013} \\ 
  Paenemarmota mexicana & 56.60 & PBDB + regression \\ 
  Paenemarmota nevadensis & 7644.00 & cite{McKenna2011} \\ 
  Paenemarmota sawrockensis & 5943.18 & cite{Tomiya2013} \\ 
  Palaechthon alticuspis & 3.20 & PBDB + regression \\ 
  Palaechthon woodi & 2.30 & PBDB + regression \\ 
  Palaeictops bicuspis & 7.31 & cite{Simons1960} \\ 
  Palaeictops bridgeri & 7.20 & PBDB + regression \\ 
  Palaeictops multicuspis & 7.50 & PBDB + regression \\ 
  Palaeogale dorothiae & 607.89 & cite{Tomiya2013} \\ 
  Palaeogale minuta & 400.00 & NOW \\ 
  Palaeogale sectoria & 13.57 & PBDB \\ 
  Palaeolagus burkei & 307.97 & cite{Tomiya2013} \\ 
  Palaeolagus hemirhizis & 5.30 & PBDB + regression \\ 
  Palaeolagus hypsodus & 678.58 & cite{Tomiya2013} \\ 
  Palaeolagus philoi & 1211.97 & cite{Tomiya2013} \\ 
  Palaeolagus primus & 5.10 & PBDB + regression \\ 
  Palaeolagus temnodon & 6.20 & PBDB + regression \\ 
  Palaeonictis occidentalis & 117.30 & PBDB + regression \\ 
  Palaeonictis peloria & 154.40 & PBDB + regression \\ 
  Palaeoryctes cruoris & 4.40 & PBDB + regression \\ 
  Palaeoryctes puercensis & 2.50 & PBDB + regression \\ 
  Palaeosyops laevidens & 787.30 & PBDB + regression \\ 
  Palaeosyops paludosus & 945.00 & PBDB + regression \\ 
  Palaeosyops robustus & 1046.40 & PBDB + regression \\ 
  Palenochtha minor & 1.30 & PBDB + regression \\ 
  Palenochtha weissae & 1.30 & PBDB + regression \\ 
  Paleotomus junior & 6.20 & PBDB + regression \\ 
  Paleotomus radagasti & 19.20 & PBDB \\ 
  Panthera onca & 100000.00 & PBDB \\ 
  Pantolambda bathmodon & 308.05 & PBDB \\ 
  Pantolambda cavirictus & 214.50 & PBDB + regression \\ 
  Pantolambda intermedius & 147.80 & PBDB + regression \\ 
  Parablastomeryx galushi & 56.40 & PBDB + regression \\ 
  Paracosoryx furlongi & 13493.99 & cite{Tomiya2013} \\ 
  Paracosoryx wilsoni & 10.67 & PBDB \\ 
  Paracryptotis gidleyi & 28.22 & cite{Tomiya2013} \\ 
  Paracryptotis rex & 42.10 & cite{Tomiya2013} \\ 
  Paracynarctus kelloggi & 8349.86 & cite{Tomiya2013} \\ 
  Paracynarctus sinclairi & 8022.46 & cite{Tomiya2013} \\ 
  Paradaphoenus cuspigerus & 4023.87 & cite{Tomiya2013} \\ 
  Paradaphoenus minimus & 8.70 & PBDB + regression \\ 
  Paradaphoenus tooheyi & 3498.19 & cite{Tomiya2013} \\ 
  Paradjidaumo alberti & 1.40 & PBDB + regression \\ 
  Paradjidaumo hypsodus & 2.00 & PBDB + regression \\ 
  Paradjidaumo reynoldsi & 1.60 & PBDB + regression \\ 
  Paradjidaumo spokanensis & 2.80 & PBDB + regression \\ 
  Paradjidaumo trilophus & 2.10 & PBDB + regression \\ 
  Paradjidaumo validus & 3.00 & PBDB + regression \\ 
  Paradomnina relictus & 23.81 & cite{Tomiya2013} \\ 
  Paraenhydrocyon josephi & 7942.63 & cite{Tomiya2013} \\ 
  Paraenhydrocyon wallovianus & 14185.85 & cite{Tomiya2013} \\ 
  Parahippus leonensis & 94845.07 & cite{Tomiya2013} \\ 
  Parahippus pawniensis & 99707.88 & cite{Tomiya2013} \\ 
  Parahippus tyleri & 173.20 & PBDB + regression \\ 
  Parahippus wyomingensis & 98400.00 & cite{MacFadden1986} \\ 
  Parahyus vagus & 250.90 & PBDB + regression \\ 
  Paralabis cedrensis & 138.80 & PBDB + regression \\ 
  Parallomys americanus & 162.39 & cite{Tomiya2013} \\ 
  Paramiolabis taylori & 131926.47 & cite{Tomiya2013} \\ 
  Paramylodon harlani & 491.20 & PBDB + regression \\ 
  Paramys adamus & 2.70 & PBDB + regression \\ 
  Paramys atavus & 2.60 & PBDB + regression \\ 
  Paramys compressidens & 20.00 & PBDB + regression \\ 
  Paramys copei & 10.30 & PBDB + regression \\ 
  Paramys delicatior & 12.00 & PBDB + regression \\ 
  Paramys delicatus & 20.30 & PBDB + regression \\ 
  Paramys excavatus & 6.80 & PBDB + regression \\ 
  Paramys taurus & 7.60 & PBDB + regression \\ 
  Paranamatomys storeri & 0.92 & PBDB \\ 
  Parapliohippus carrizoensis & 80821.64 & cite{Tomiya2013} \\ 
  Parapliosaccomys oregonensis & 24.29 & cite{Tomiya2013} \\ 
  Parapliosaccomys transversus & 2.54 & PBDB \\ 
  Parapotos tedfordi & 3533.34 & cite{Tomiya2013} \\ 
  Pararyctes pattersoni & 2.00 & PBDB + regression \\ 
  Paratomarctus euthos & 14472.42 & cite{Tomiya2013} \\ 
  Paratomarctus temerarius & 11498.82 & cite{Tomiya2013} \\ 
  Paratylopus labiatus & 101.80 & PBDB + regression \\ 
  Paratylopus primaevus & 13.20 & PBDB + regression \\ 
  Parectypodus clemensi & 2.10 & PBDB + regression \\ 
  Parectypodus corystes & 2.80 & PBDB + regression \\ 
  Parectypodus laytoni & 1.40 & PBDB + regression \\ 
  Parectypodus lunatus & 2.00 & PBDB + regression \\ 
  Parectypodus simpsoni & 2.70 & PBDB + regression \\ 
  Parectypodus sinclairi & 19.00 & cite{Wilson2012} \\ 
  Parectypodus sylviae & 1.50 & PBDB + regression \\ 
  Parectypodus trovessartianus & 4.30 & PBDB + regression \\ 
  Pareumys boskeyi & 3.70 & PBDB + regression \\ 
  Pareumys grangeri & 2.90 & PBDB + regression \\ 
  Pareumys guensburgi & 4.80 & PBDB + regression \\ 
  Pareumys milleri & 3.40 & PBDB + regression \\ 
  Parictis parvus & 6.80 & PBDB + regression \\ 
  Parictis personi & 7.60 & PBDB + regression \\ 
  Paromomys depressidens & 3.40 & PBDB + regression \\ 
  Paromomys maturus & 6.70 & PBDB + regression \\ 
  Paronychomys alticuspis & 17.12 & cite{Tomiya2013} \\ 
  Paronychomys lemredfieldi & 20.09 & cite{Tomiya2013} \\ 
  Paronychomys tuttlei & 38.86 & cite{Tomiya2013} \\ 
  Paroreodon parvus & 51150.00 & cite{McKenna2011} \\ 
  Parvericius montanus & 41.26 & cite{Tomiya2013} \\ 
  Parvericius voorhiesi & 28.79 & cite{Tomiya2013} \\ 
  Patriofelis ferox & 22418.00 & cite{McKenna2011} \\ 
  Patriofelis ulta & 136.00 & PBDB + regression \\ 
  Patriolestes novaceki & 10.20 & PBDB + regression \\ 
  Pauromys exallos & 1.50 & PBDB + regression \\ 
  Pauromys lillegraveni & 1.20 & PBDB + regression \\ 
  Pauromys simplex & 1.30 & PBDB + regression \\ 
  Pauromys texensis & 1.50 & PBDB + regression \\ 
  Pediomeryx hemphillensis & 167711.41 & cite{Tomiya2013} \\ 
  Pelycodus jarrovii & 30.90 & PBDB + regression \\ 
  Pelycomys brulanus & 6.10 & PBDB + regression \\ 
  Pelycomys rugosus & 7.80 & PBDB + regression \\ 
  Penetrigonias hudsoni & 544.50 & PBDB \\ 
  Pentacemylus leotensis & 24.90 & PBDB + regression \\ 
  Pentacemylus progressus & 29.20 & PBDB + regression \\ 
  Pentacodon inversus & 17.70 & PBDB + regression \\ 
  Pentacodon occultus & 31.70 & PBDB + regression \\ 
  Peraceras hessei & 936589.16 & cite{Tomiya2013} \\ 
  Peraceras profectum & 2326789.55 & cite{Tomiya2013} \\ 
  Peraceras superciliosum & 1639660.88 & cite{Tomiya2013} \\ 
  Peradectes californicus & 1.00 & PBDB + regression \\ 
  Peradectes chesteri & 0.90 & PBDB + regression \\ 
  Peradectes elegans & 1.40 & PBDB + regression \\ 
  Peradectes minor & 1.40 & cite{Taylor1976} \\ 
  Peradectes protinnominatus & 1.50 & PBDB + regression \\ 
  Peratherium comstocki & 4.30 & PBDB + regression \\ 
  Peratherium marsupium & 4.40 & PBDB + regression \\ 
  Perchoerus probus & 94.00 & PBDB + regression \\ 
  Peridiomys halis & 63.43 & cite{Tomiya2013} \\ 
  Peridiomys oregonensis & 2.20 & PBDB + regression \\ 
  Peridiomys rusticus & 69.41 & cite{Tomiya2013} \\ 
  Periptychus carinidens & 106.00 & PBDB + regression \\ 
  Periptychus coarctatus & 90.00 & PBDB + regression \\ 
  Perognathus ancenensis & 11.47 & cite{Tomiya2013} \\ 
  Perognathus coquorum & 2.20 & PBDB + regression \\ 
  Perognathus dunklei & 7.92 & cite{Tomiya2013} \\ 
  Perognathus furlongi & 9.68 & cite{Tomiya2013} \\ 
  Perognathus gidleyi & 11.82 & cite{Tomiya2013} \\ 
  Perognathus maldei & 1.10 & PBDB + regression \\ 
  Perognathus mclaughlini & 8.33 & cite{Tomiya2013} \\ 
  Perognathus minutus & 5.93 & cite{Tomiya2013} \\ 
  Perognathus pearlettensis & 6.89 & cite{Tomiya2013} \\ 
  Perognathus rexroadensis & 1.50 & PBDB + regression \\ 
  Perognathus trojectioansrum & 3.63 & cite{Tomiya2013} \\ 
  Peromyscus antiquus & 26.84 & cite{Tomiya2013} \\ 
  Peromyscus brachygnathus & 1.10 & PBDB + regression \\ 
  Peromyscus complexus & 1.83 & PBDB \\ 
  Peromyscus cragini & 15.49 & cite{Tomiya2013} \\ 
  Peromyscus dentalis & 1.20 & PBDB + regression \\ 
  Peromyscus hagermanensis & 19.89 & cite{Tomiya2013} \\ 
  Peromyscus minimus & 0.70 & PBDB + regression \\ 
  Peromyscus nosher & 1.40 & PBDB + regression \\ 
  Peromyscus polionotus & 14.30 & PBDB \\ 
  Peromyscus sarmocophinus & 1.50 & PBDB + regression \\ 
  Petauristodon jamesi & 307.97 & cite{Tomiya2013} \\ 
  Petauristodon mathewsi & 214.86 & cite{Tomiya2013} \\ 
  Petauristodon pattersoni & 336.97 & cite{Tomiya2013} \\ 
  Pewelagus dawsonae & 57.70 & cite{Jepsen1932} \\ 
  Pewelagus mexicanus & 6.50 & PBDB + regression \\ 
  Phelosaccomys annae & 26.31 & cite{Tomiya2013} \\ 
  Phelosaccomys hibbardi & 46.53 & cite{Tomiya2013} \\ 
  Phelosaccomys neomexicanus & 19.89 & cite{Tomiya2013} \\ 
  Phelosaccomys shotwelli & 20.49 & cite{Tomiya2013} \\ 
  Phenacocoelus typus & 52052.08 & cite{Tomiya2013} \\ 
  Phenacodaptes sabulosus & 6.20 & PBDB + regression \\ 
  Phenacodus bisonensis & 76.90 & PBDB + regression \\ 
  Phenacodus grangeri & 101.90 & PBDB + regression \\ 
  Phenacodus intermedius & 131.50 & PBDB + regression \\ 
  Phenacodus magnus & 169.80 & PBDB + regression \\ 
  Phenacodus matthewi & 41.18 & cite{Cope1871} \\ 
  Phenacodus trilobatus & 159.30 & PBDB + regression \\ 
  Phenacodus vortmani & 59.40 & PBDB + regression \\ 
  Phenacolemur fortior & 6.50 & cite{Wood1962} \\ 
  Phenacolemur mcgrewi & 3.60 & PBDB + regression \\ 
  Phenacolemur praecox & 5.80 & PBDB + regression \\ 
  Phenacolemur simonsi & 2.80 & PBDB + regression \\ 
  Phenacomys gryci & 228.00 & cite{McKenna2011} \\ 
  Philotrox condoni & 11968.10 & cite{Tomiya2013} \\ 
  Phlaocyon achoros & 2951.30 & cite{Tomiya2013} \\ 
  Phlaocyon annectens & 3498.19 & cite{Tomiya2013} \\ 
  Phlaocyon latidens & 2779.43 & cite{Tomiya2013} \\ 
  Phlaocyon leucosteus & 3827.63 & cite{Tomiya2013} \\ 
  Phlaocyon minor & 3498.19 & cite{Tomiya2013} \\ 
  Phlaocyon taylori & 1939.14 & cite{Tomiya2013} \\ 
  Phlaocyon yatkolai & 9604.62 & cite{Tomiya2013} \\ 
  Phoberocyon johnhenryi & 179871.86 & cite{Tomiya2013} \\ 
  Picrodus calgariensis & 1.50 & PBDB + regression \\ 
  Picrodus canpacius & 3.10 & PBDB + regression \\ 
  Picrodus silberlingi & 3.10 & PBDB + regression \\ 
  Pipestoneomys bisulcatus & 2.60 & PBDB + regression \\ 
  Plagioctenodon krausae & 0.90 & PBDB + regression \\ 
  Plagioctenodon rosei & 1.50 & PBDB + regression \\ 
  Plagiomene accola & 9.30 & PBDB + regression \\ 
  Plagiomene multicuspis & 14.90 & PBDB + regression \\ 
  Planisorex dixonensis & 1.60 & PBDB + regression \\ 
  Platygonus bicalcaratus & 182.50 & PBDB + regression \\ 
  Platygonus oregonensis & 40134.84 & cite{Tomiya2013} \\ 
  Platygonus pearcei & 55826.28 & cite{Tomiya2013} \\ 
  Platygonus vetus & 65463.62 & cite{Brook2004a} \\ 
  Plesiadapis anceps & 8.30 & PBDB + regression \\ 
  Plesiadapis churchilli & 15.10 & PBDB + regression \\ 
  Plesiadapis cookei & 36.80 & PBDB + regression \\ 
  Plesiadapis dubius & 10.10 & PBDB + regression \\ 
  Plesiadapis fodinatus & 13.50 & PBDB + regression \\ 
  Plesiadapis gingerichi & 28.20 & PBDB + regression \\ 
  Plesiadapis praecursor & 6.60 & PBDB + regression \\ 
  Plesiadapis rex & 13.30 & PBDB + regression \\ 
  Plesiocolopirus hancocki & 94.70 & PBDB + regression \\ 
  Plesiogulo lindsayi & 4628.55 & cite{Tomiya2013} \\ 
  Plesiogulo marshalli & 3133.79 & cite{Tomiya2013} \\ 
  Plesiolestes nacimienti & 233.00 & cite{Soligo2006} \\ 
  Plesiolestes problematicus & 5.00 & PBDB + regression \\ 
  Plesiolestes wilsoni & 9.40 & PBDB + regression \\ 
  Plesiosminthus clivosus & 9.30 & cite{Tomiya2013} \\ 
  Plesiosorex coloradensis & 192.48 & cite{Tomiya2013} \\ 
  Plesiosorex donroosai & 685.40 & cite{Tomiya2013} \\ 
  Pleurolicus dakotensis & 60.95 & cite{Tomiya2013} \\ 
  Pleurolicus exiguus & 1.80 & PBDB + regression \\ 
  Pleurolicus sellardsi & 2.77 & cite{Zakrzewski1991a} \\ 
  Pleurolicus sulcifrons & 83.93 & cite{Tomiya2013} \\ 
  Pliocyon medius & 172818.99 & cite{Tomiya2013} \\ 
  Pliocyon robustus & 176310.16 & cite{Tomiya2013} \\ 
  Pliogale furlongi & 10.60 & cite{Wood1962} \\ 
  Pliogale manka & 8.40 & PBDB + regression \\ 
  Pliogeomys parvus & 10.28 & cite{Tomiya2013} \\ 
  Pliogeomys russelli & 15.18 & cite{Tomiya2013} \\ 
  Pliohippus fossulatus & 257815.63 & cite{Tomiya2013} \\ 
  Pliohippus pernix & 198789.15 & cite{Tomiya2013} \\ 
  Pliohippus tehonensis & 231.80 & PBDB + regression \\ 
  Pliometanastes galushai & 53.00 & PBDB \\ 
  Pliometanastes protistus & 189.00 & PBDB + regression \\ 
  Plionarctos edensis & 56954.05 & cite{Tomiya2013} \\ 
  Plionarctos harroldorum & 21.30 & PBDB + regression \\ 
  Plionictis ogygia & 36.97 & cite{Tomiya2013} \\ 
  Pliophenacomys dixonensis & 2.70 & PBDB + regression \\ 
  Pliophenacomys finneyi & 3.10 & PBDB + regression \\ 
  Pliophenacomys meadensis & 2.80 & PBDB + regression \\ 
  Pliophenacomys osborni & 86.49 & cite{Tomiya2013} \\ 
  Pliophenacomys primaevus & 61.56 & cite{Tomiya2013} \\ 
  Pliosaccomys dubius & 27.66 & cite{Tomiya2013} \\ 
  Pliosaccomys higginsensis & 18.17 & cite{Tomiya2013} \\ 
  Pliotaxidea garberi & 9.60 & PBDB + regression \\ 
  Pliotaxidea nevadensis & 130.32 & cite{Tomiya2013} \\ 
  Pliotomodon primitivus & 107.77 & cite{Tomiya2013} \\ 
  Pliozapus solus & 18.73 & cite{Tomiya2013} \\ 
  Plithocyon ursinus & 189094.09 & cite{Tomiya2013} \\ 
  Poabromylus golzi & 40.90 & PBDB + regression \\ 
  Poabromylus kayi & 64.10 & PBDB + regression \\ 
  Poebrotherium eximium & 106.10 & PBDB + regression \\ 
  Poebrotherium wilsoni & 119.80 & PBDB + regression \\ 
  Pogonodon eileenae & 242.00 & cite{Fox2011b} \\ 
  Pratifelis martini & 215345.72 & cite{Tomiya2013} \\ 
  Pratilepus kansasensis & 972.63 & cite{Tomiya2013} \\ 
  Premnoides douglassi & 2.80 & PBDB + regression \\ 
  Presbymys lophatus & 3.30 & PBDB + regression \\ 
  Presbytherium rhodorugatus & 46.40 & PBDB + regression \\ 
  Princetonia yalensis & 21.80 & PBDB + regression \\ 
  Probassariscus matthewi & 7.80 & PBDB + regression \\ 
  Probathyopsis harrisorum & 162.70 & PBDB + regression \\ 
  Probathyopsis praecursor & 187.80 & PBDB + regression \\ 
  Problastomeryx primus & 14913.17 & cite{Tomiya2013} \\ 
  Procamelus grandis & 400312.19 & cite{Tomiya2013} \\ 
  Procamelus occidentalis & 189094.09 & cite{Tomiya2013} \\ 
  Procerberus formicarum & 4.60 & PBDB + regression \\ 
  Prochetodon cavus & 7.00 & PBDB + regression \\ 
  Prochetodon foxi & 6.80 & PBDB + regression \\ 
  Prochetodon speirsae & 10.10 & PBDB \\ 
  Prochetodon taxus & 10.90 & PBDB + regression \\ 
  Procranioceras skinneri & 169396.94 & cite{Tomiya2013} \\ 
  Procynodictis progressus & 9.00 & PBDB + regression \\ 
  Procyon lotor & 5814.00 & PBDB \\ 
  Procyon rexroadensis & 12.46 & PBDB + regression \\ 
  Prodiacodon concordiarcensis & 3.50 & PBDB + regression \\ 
  Prodiacodon crustulum & 10.01 & cite{Lim2001} \\ 
  Prodiacodon furor & 4.58 & cite{Scott1937} \\ 
  Prodiacodon puercensis & 9.66 & cite{Mellett1969} \\ 
  Prodiacodon tauricinerei & 5.20 & PBDB + regression \\ 
  Prodipodomys centralis & 1.65 & cite{Becker1981} \\ 
  Prodipodomys idahoensis & 22.42 & cite{Tomiya2013} \\ 
  Prodipodomys kansensis & 12.81 & cite{Tomiya2013} \\ 
  Prodipodomys timoteoensis & 1.60 & PBDB + regression \\ 
  Prohesperocyon wilsoni & 9.10 & PBDB + regression \\ 
  Proheteromys fedti & 1.50 & PBDB + regression \\ 
  Proheteromys floridanus & 5.37 & cite{Tomiya2013} \\ 
  Proheteromys gremmelsi & 1.93 & cite{Becker1981} \\ 
  Proheteromys ironcloudi & 10.59 & cite{Tomiya2013} \\ 
  Proheteromys maximus & 87.36 & cite{Tomiya2013} \\ 
  Proheteromys nebraskensis & 1.66 & PBDB \\ 
  Proheteromys sulculus & 1.40 & PBDB + regression \\ 
  Proheteromys toledoensis & 59.15 & cite{Tomiya2013} \\ 
  Prolapsus junctionis & 3.30 & PBDB + regression \\ 
  Prolapsus sibilatoris & 5.50 & PBDB + regression \\ 
  Prolimnocyon antiquus & 19.90 & PBDB + regression \\ 
  Prolimnocyon atavus & 26.70 & PBDB + regression \\ 
  Prolimnocyon haematus & 16.80 & PBDB + regression \\ 
  Promartes darbyi & 97.00 & cite{Dawson2007} \\ 
  Promartes gemmarosae & 86.00 & cite{Dawson2007} \\ 
  Promartes lepidus & 46.41 & cite{Mora2005} \\ 
  Promioclaenus acolytus & 10.00 & PBDB + regression \\ 
  Promioclaenus pipiringosi & 13.90 & PBDB + regression \\ 
  Promioclaenus thnetus & 7.80 & cite{Gazin1930} \\ 
  Promylagaulus riggsi & 85.63 & cite{Tomiya2013} \\ 
  Pronodens silberlingi & 22247.84 & cite{Tomiya2013} \\ 
  Pronothodectes gaoi & 7.70 & PBDB + regression \\ 
  Pronothodectes jepi & 7.50 & PBDB + regression \\ 
  Pronothodectes matthewi & 5.60 & PBDB + regression \\ 
  Pronotolagus apachensis & 445.86 & cite{Tomiya2013} \\ 
  Pronotolagus nevadensis & 60.34 & cite{Tomiya2013} \\ 
  Pronotolagus whitei & 1436.55 & cite{Tomiya2013} \\ 
  Proscalops miocaenus & 2.70 & PBDB + regression \\ 
  Proscalops secundus & 72.97 & cite{Tomiya2013} \\ 
  Proscalops tertius & 96.54 & cite{Tomiya2013} \\ 
  Prosciurus magnus & 4.10 & PBDB + regression \\ 
  Prosciurus parvus & 3.20 & PBDB + regression \\ 
  Prosciurus relictus & 3.10 & PBDB + regression \\ 
  Prosigmodon chihuahuensis & 3.30 & PBDB + regression \\ 
  Prosigmodon ferrusquiai & 2.63 & PBDB \\ 
  Prosigmodon holocuspis & 113.30 & cite{Tomiya2013} \\ 
  Prosigmodon oroscoi & 1.90 & PBDB + regression \\ 
  Prosomys mimus & 2.10 & PBDB + regression \\ 
  Prosthennops niobrarensis & 43044.94 & cite{Tomiya2013} \\ 
  Prosthennops serus & 53637.30 & cite{Tomiya2013} \\ 
  Prosthennops xiphodonticus & 23860.99 & cite{Tomiya2013} \\ 
  Prosynthetoceras francisi & 134591.56 & cite{Tomiya2013} \\ 
  Prosynthetoceras orthrionanus & 40134.84 & cite{Tomiya2013} \\ 
  Protadjidaumo pauli & 1.50 & PBDB + regression \\ 
  Protadjidaumo typus & 1.20 & PBDB + regression \\ 
  Protapirus obliquidens & 440.00 & cite{Rose1982a} \\ 
  Protapirus simplex & 253.90 & PBDB + regression \\ 
  Protepicyon raki & 23623.56 & cite{Tomiya2013} \\ 
  Proterix bicuspis & 46.00 & cite{Coombs1979} \\ 
  Proterix loomisi & 16.20 & PBDB + regression \\ 
  Proterixoides davisi & 10.10 & PBDB + regression \\ 
  Prothryptacodon albertensis & 10.70 & PBDB + regression \\ 
  Prothryptacodon furens & 19.20 & PBDB + regression \\ 
  Prothryptacodon hilli & 23.50 & PBDB + regression \\ 
  Protictis haydenianus & 7.43 & PBDB + regression \\ 
  Protictis microlestes & 13.63 & PBDB \\ 
  Protictis minor & 6.30 & PBDB + regression \\ 
  Protictis paralus & 4.10 & PBDB + regression \\ 
  Protictis paulus & 8.40 & PBDB \\ 
  Protictis simpsoni & 8.60 & PBDB + regression \\ 
  Protitanops curryi & 142477.00 & cite{McKenna2011} \\ 
  Protitanotherium superbum & 1522.00 & PBDB + regression \\ 
  Protoceras celer & 172.20 & PBDB + regression \\ 
  Protoceras skinneri & 272.16 & PBDB \\ 
  Protohippus gidleyi & 164390.50 & cite{Tomiya2013} \\ 
  Protohippus perditus & 135944.23 & cite{Tomiya2013} \\ 
  Protohippus supremus & 167711.41 & cite{Tomiya2013} \\ 
  Protohippus vetus & 191.40 & PBDB \\ 
  Protolabis coartatus & 110194.25 & cite{Tomiya2013} \\ 
  Protolabis heterodontus & 219695.99 & cite{Tomiya2013} \\ 
  Protomarctus optatus & 11271.13 & cite{Tomiya2013} \\ 
  Protoreodon pacificus & 82.10 & PBDB + regression \\ 
  Protoreodon parvus & 80.30 & PBDB + regression \\ 
  Protoreodon pearcei & 88.30 & PBDB + regression \\ 
  Protoreodon petersoni & 55.40 & PBDB + regression \\ 
  Protoreodon pumilus & 108.10 & PBDB + regression \\ 
  Protoreodon walshi & 118.90 & PBDB + regression \\ 
  Protorohippus venticolus & 47.20 & PBDB + regression \\ 
  Protosciurus mengi & 8.70 & PBDB + regression \\ 
  Protosciurus tecuyensis & 478.19 & cite{Tomiya2013} \\ 
  Protoselene griphus & 21.20 & PBDB + regression \\ 
  Protoselene opisthacus & 21.80 & PBDB + regression \\ 
  Protospermophilus kelloggi & 202.35 & cite{Tomiya2013} \\ 
  Protospermophilus malheurensis & 103.54 & cite{Tomiya2013} \\ 
  Protospermophilus oregonensis & 450.34 & cite{Tomiya2013} \\ 
  Protospermophilus quatalensis & 273.14 & cite{Tomiya2013} \\ 
  Protospermophilus vortmani & 230.44 & cite{Tomiya2013} \\ 
  Prototomus deimos & 11.20 & PBDB + regression \\ 
  Prototomus martis & 32.70 & PBDB + regression \\ 
  Prototomus phobos & 28.00 & PBDB + regression \\ 
  Prototomus robustus & 26.50 & PBDB + regression \\ 
  Prototomus secundarius & 20.50 & PBDB + regression \\ 
  Protungulatum donnae & 11.70 & PBDB + regression \\ 
  Protylopus annectens & 47.30 & PBDB + regression \\ 
  Protylopus pearsonensis & 88.60 & PBDB + regression \\ 
  Protylopus petersoni & 54.70 & PBDB + regression \\ 
  Protylopus robustus & 53.10 & PBDB + regression \\ 
  Protylopus stocki & 48.60 & PBDB + regression \\ 
  Proviverroides piercei & 48.50 & PBDB + regression \\ 
  Psalidocyon marianae & 8777.97 & cite{Tomiya2013} \\ 
  Pseudaelurus aeluroides & 31571.18 & cite{Tomiya2013} \\ 
  Pseudaelurus intrepidus & 40945.61 & cite{Tomiya2013} \\ 
  Pseudaelurus marshi & 33189.87 & cite{Tomiya2013} \\ 
  Pseudaelurus stouti & 5767.53 & cite{Tomiya2013} \\ 
  Pseudhipparion curtivallum & 58104.59 & cite{Tomiya2013} \\ 
  Pseudhipparion gratum & 108012.26 & cite{Tomiya2013} \\ 
  Pseudhipparion hessei & 75357.60 & cite{Tomiya2013} \\ 
  Pseudhipparion retrusum & 86681.87 & cite{Tomiya2013} \\ 
  Pseudhipparion simpsoni & 48050.12 & cite{Tomiya2013} \\ 
  Pseudhipparion skinneri & 54176.36 & cite{Tomiya2013} \\ 
  Pseudoblastomeryx advena & 10097.06 & cite{Tomiya2013} \\ 
  Pseudoceras skinneri & 10938.02 & cite{Tomiya2013} \\ 
  Pseudocylindrodon lateriviae & 4.00 & PBDB + regression \\ 
  Pseudocylindrodon medius & 2.50 & PBDB + regression \\ 
  Pseudocylindrodon neglectus & 3.70 & PBDB + regression \\ 
  Pseudocylindrodon pintoensis & 5.78 & MIOMAP \\ 
  Pseudodiplacodon progressum & 888.00 & PBDB \\ 
  Pseudolabis dakotensis & 59874.14 & cite{Tomiya2013} \\ 
  Pseudoparablastomeryx francescita & 22.80 & PBDB \\ 
  Pseudoparablastomeryx scotti & 5884.05 & cite{Tomiya2013} \\ 
  Pseudoprotoceras longinaris & 83.40 & PBDB + regression \\ 
  Pseudoprotoceras minor & 57.60 & PBDB + regression \\ 
  Pseudotheridomys cuyamensis & 12.43 & cite{Tomiya2013} \\ 
  Pseudotheridomys hesperus & 14.15 & cite{Tomiya2013} \\ 
  Pseudotheridomys pagei & 8.17 & cite{Tomiya2013} \\ 
  Pseudotomus californicus & 30.40 & PBDB + regression \\ 
  Pseudotomus eugenei & 65.90 & PBDB + regression \\ 
  Pseudotomus hians & 90.00 & PBDB \\ 
  Pseudotomus horribilis & 27.02 & cite{Carraway2010} \\ 
  Pseudotomus johanniculi & 60.20 & PBDB + regression \\ 
  Pseudotomus littoralis & 24.50 & PBDB + regression \\ 
  Pseudotomus petersoni & 32.90 & cite{Carraway2010} \\ 
  Pseudotomus robustus & 32.20 & PBDB + regression \\ 
  Pseudotrimylus mawbyi & 7.30 & PBDB + regression \\ 
  Ptilodus fractus & 105.00 & cite{Wilson2012} \\ 
  Ptilodus gnomus & 4.30 & PBDB + regression \\ 
  Ptilodus kummae & 6.90 & PBDB + regression \\ 
  Ptilodus mediaevus & 8.60 & PBDB + regression \\ 
  Ptilodus montanus & 9.50 & PBDB + regression \\ 
  Ptilodus wyomingensis & 147.00 & cite{Wilson2012} \\ 
  Puercolestes simpsoni & 10.60 & PBDB + regression \\ 
  Puma concolor & 48009.00 & PBDB \\ 
  Puma lacustris & 15.30 & PBDB + regression \\ 
  Pyrocyon dioctetus & 22.70 & PBDB + regression \\ 
  Quadratomus grandis & 19.00 & PBDB + regression \\ 
  Quadratomus grossus & 34.70 & PBDB + regression \\ 
  Quadrodens wilsoni & 46.06 & cite{Tomiya2013} \\ 
  Rakomeryx sinclairi & 111301.72 & cite{Tomiya2013} \\ 
  Rapamys fricki & 15.00 & PBDB + regression \\ 
  Raphictis gausion & 3.70 & PBDB + regression \\ 
  Reithrodontomys galushai & 1.10 & PBDB + regression \\ 
  Reithrodontomys rexroadensis & 0.90 & PBDB + regression \\ 
  Reithrodontomys wetmorei & 9.03 & cite{Tomiya2013} \\ 
  Reithroparamys debequensis & 5.70 & PBDB + regression \\ 
  Reithroparamys delicatissimus & 9.40 & PBDB + regression \\ 
  Reithroparamys huerfanensis & 7.90 & PBDB + regression \\ 
  Reithroparamys sciuroides & 9.66 & cite{Wang1994a} \\ 
  Repomys arizonensis & 2.80 & PBDB + regression \\ 
  Repomys gustelyi & 81.45 & cite{Tomiya2013} \\ 
  Repomys maxumi & 122.73 & cite{Tomiya2013} \\ 
  Repomys panacaensis & 38.47 & cite{Tomiya2013} \\ 
  Rhizocyon oregonensis & 3361.02 & cite{Tomiya2013} \\ 
  Rhynchotherium falconeri & 10697.20 & PBDB + regression \\ 
  Russellagus vonhofi & 214.86 & cite{Tomiya2013} \\ 
  Sanctimus falkenbachi & 151.41 & cite{Tomiya2013} \\ 
  Sanctimus stouti & 120.30 & cite{Tomiya2013} \\ 
  Sanctimus stuartae & 100.48 & cite{Tomiya2013} \\ 
  Satherium piscinarium & 934.49 & cite{Tomiya2013} \\ 
  Saxonella naylori & 3.00 & PBDB + regression \\ 
  Scalopoides isodens & 32.14 & cite{Tomiya2013} \\ 
  Scalopoides ripafodiator & 27.39 & cite{Tomiya2013} \\ 
  Scalopus aquaticus & 39.60 & PBDB \\ 
  Scapanoscapter simplicidens & 58.56 & cite{Tomiya2013} \\ 
  Scapanus hagermanensis & 7.00 & cite{McKenna2011} \\ 
  Scapanus latimanus & 55.00 & PBDB \\ 
  Scapanus proceridens & 65.37 & cite{Tomiya2013} \\ 
  Scapanus shultzi & 97.00 & cite{McKenna2011} \\ 
  Scapanus townsendii & 8.59 & cite{Smith2004} \\ 
  Scaphohippus sumani & 171.00 & PBDB \\ 
  Scenopagus curtidens & 2.60 & PBDB + regression \\ 
  Scenopagus edenensis & 4.80 & PBDB + regression \\ 
  Scenopagus priscus & 2.00 & PBDB + regression \\ 
  Schaubeumys galbreathi & 1.57 & cite{Loomis1911} \\ 
  Schaubeumys grangeri & 1.60 & PBDB + regression \\ 
  Schaubeumys sabrae & 1.48 & cite{Matthew1901} \\ 
  Schizodontomys amnicolus & 111.05 & cite{Tomiya2013} \\ 
  Schizodontomys greeni & 94.00 & cite{McKenna2011} \\ 
  Schizodontomys harkseni & 105.64 & cite{Tomiya2013} \\ 
  Sciuravus bridgeri & 2.40 & PBDB + regression \\ 
  Sciuravus nitidus & 5.30 & PBDB + regression \\ 
  Sciuravus popi & 7.90 & PBDB + regression \\ 
  Sciuravus powayensis & 3.70 & PBDB + regression \\ 
  Sciuravus wilsoni & 3.60 & PBDB + regression \\ 
  Sciurion campestre & 1.90 & PBDB + regression \\ 
  Sciurus carolinensis & 518.00 & PBDB \\ 
  Sciurus olsoni & 1.71 & PBDB \\ 
  Scottimus exiguus & 3.21 & cite{Novacek1977} \\ 
  Scottimus longiquus & 111.05 & cite{Tomiya2013} \\ 
  Scottimus lophatus & 4.65 & PBDB \\ 
  Scottimus viduus & 2.50 & PBDB + regression \\ 
  Selenaletes scopaeus & 32.10 & PBDB + regression \\ 
  Serbelodon barbourensis & 910.00 & cite{Secord2008a} \\ 
  Sespedectes singularis & 2.70 & PBDB + regression \\ 
  Sespedectes stocki & 2.40 & PBDB + regression \\ 
  Sespemys thurstoni & 290.03 & cite{Tomiya2013} \\ 
  Sespia californica & 3604.72 & cite{Tomiya2013} \\ 
  Sespia nitida & 91.00 & cite{VanValkenburgh2007a} \\ 
  Shoshonius bowni & 5.60 & PBDB + regression \\ 
  Shoshonius cooperi & 3.40 & PBDB + regression \\ 
  Sifrhippus aemulor & 45.36 & cite{Osborn1933} \\ 
  Sifrhippus grangeri & 40.20 & PBDB + regression \\ 
  Sifrhippus sandrae & 38.00 & PBDB + regression \\ 
  Sigmodon curtisi & 3.80 & PBDB + regression \\ 
  Sigmodon hudspethensis & 3.20 & PBDB + regression \\ 
  Sigmodon minor & 52.98 & cite{Tomiya2013} \\ 
  Simidectes magnus & 9.10 & PBDB + regression \\ 
  Simidectes medius & 8.30 & PBDB + regression \\ 
  Simidectes merriami & 11.10 & PBDB + regression \\ 
  Similisciurus maxwelli & 259.82 & cite{Tomiya2013} \\ 
  Simimeryx hudsoni & 20.50 & PBDB + regression \\ 
  Simimeryx minutus & 11.50 & PBDB + regression \\ 
  Simimys landeri & 2.30 & PBDB + regression \\ 
  Simimys simplex & 1.30 & PBDB + regression \\ 
  Simocyon primigenius & 70000.00 & NOW \\ 
  Simojovelhyus pocitosense & 57.20 & cite{Wang1999} \\ 
  Simpsonictis pegus & 3.62 & PBDB + regression \\ 
  Simpsonictis tenuis & 2.90 & PBDB + regression \\ 
  Simpsonlemur citatus & 171.00 & cite{Soligo2006} \\ 
  Simpsonlemur jepseni & 121.00 & cite{Soligo2006} \\ 
  Simpsonodus chacensis & 18.50 & PBDB + regression \\ 
  Sinclairella dakotensis & 7.20 & PBDB + regression \\ 
  Sinopa major & 49.30 & PBDB + regression \\ 
  Sinopa rapax & 31.30 & PBDB + regression \\ 
  Smilodectes gracilis & 16.60 & PBDB + regression \\ 
  Smilodectes mcgrewi & 18.10 & PBDB + regression \\ 
  Smilodectes sororis & 15.70 & PBDB + regression \\ 
  Smilodon gracilis & 22.70 & PBDB + regression \\ 
  Sminthosinis bowleri & 100.48 & cite{Tomiya2013} \\ 
  Sorex cinereus & 3.80 & PBDB \\ 
  Sorex edwardsi & 1.13 & PBDB \\ 
  Sorex hagermanensis & 1.20 & PBDB + regression \\ 
  Sorex meltoni & 3.46 & cite{Tomiya2013} \\ 
  Sorex palustris & 3.09 & cite{Smith2004} \\ 
  Sorex powersi & 7.77 & cite{Tomiya2013} \\ 
  Sorex rexroadensis & 8.00 & cite{McKenna2011} \\ 
  Sorex yatkolai & 0.92 & PBDB \\ 
  Spermophilus argonautus & 131.63 & cite{Tomiya2013} \\ 
  Spermophilus bensoni & 184.93 & cite{Tomiya2013} \\ 
  Spermophilus boothi & 9.83 & cite{Secord2008a} \\ 
  Spermophilus cragini & 578.25 & cite{Tomiya2013} \\ 
  Spermophilus dotti & 270.43 & cite{Tomiya2013} \\ 
  Spermophilus fricki & 6.38 & cite{Simons1960} \\ 
  Spermophilus gidleyi & 700.00 & cite{McKenna2011} \\ 
  Spermophilus howelli & 139.77 & cite{Tomiya2013} \\ 
  Spermophilus jerae & 87.36 & cite{Tomiya2013} \\ 
  Spermophilus matachicensis & 6.20 & PBDB + regression \\ 
  Spermophilus matthewi & 8.68 & cite{Beatty2009} \\ 
  Spermophilus meadensis & 100.48 & cite{Tomiya2013} \\ 
  Spermophilus rexroadensis & 8.00 & PBDB + regression \\ 
  Spermophilus russelli & 166.00 & cite{McKenna2011} \\ 
  Spermophilus shotwelli & 204.00 & cite{McKenna2011} \\ 
  Spermophilus tephrus & 89.12 & cite{Tomiya2013} \\ 
  Spermophilus wellingtonensis & 295.89 & cite{Tomiya2013} \\ 
  Spermophilus wilsoni & 247.15 & cite{Tomiya2013} \\ 
  Sphacorhysis burntforkensis & 2.80 & PBDB + regression \\ 
  Sphenophalos nevadanus & 63.00 & PBDB \\ 
  Spilogale microdens & 6.30 & cite{Chester2012} \\ 
  Spilogale putorius & 12.59 & cite{Smith2004} \\ 
  Spilogale rexroadi & 5.60 & PBDB + regression \\ 
  Stegomastodon mirificus & 7985.90 & PBDB + regression \\ 
  Steinius annectens & 397.75 & cite{Strait2001} \\ 
  Steinius vespertinus & 5.10 & PBDB + regression \\ 
  Stelocyon arctylos & 7.50 & PBDB + regression \\ 
  Stenoechinus tantalus & 48.42 & cite{Tomiya2013} \\ 
  Stenomylus gracilis & 44801.64 & cite{Tomiya2013} \\ 
  Stenomylus hitchcocki & 38948.67 & cite{Tomiya2013} \\ 
  Stenomylus taylori & 192.72 & PBDB \\ 
  Sthenictis dolichops & 665.14 & cite{Tomiya2013} \\ 
  Sthenictis junturensis & 330.30 & cite{Tomiya2013} \\ 
  Stibarus montanus & 17.60 & PBDB + regression \\ 
  Stibarus obtusilobus & 19.70 & PBDB + regression \\ 
  Stibarus quadricuspis & 24.40 & PBDB + regression \\ 
  Stockia powayensis & 6.00 & PBDB + regression \\ 
  Stratimus strobeli & 20.09 & cite{Tomiya2013} \\ 
  Strigorhysis bridgerensis & 4.40 & PBDB + regression \\ 
  Strigorhysis huerfanensis & 6.30 & PBDB + regression \\ 
  Stygimys gratus & 5.30 & PBDB + regression \\ 
  Stygimys jepseni & 84.00 & cite{Wilson2012} \\ 
  Stygimys kuszmauli & 8.30 & PBDB + regression \\ 
  Stylinodon mirus & 135.50 & PBDB + regression \\ 
  Subdromomeryx antilopinus & 59874.14 & cite{Tomiya2013} \\ 
  Subhyracodon mitis & 455.00 & cite{Scott1940} \\ 
  Subhyracodon occidentalis & 580.10 & PBDB + regression \\ 
  Sunkahetanka geringensis & 11158.98 & cite{Tomiya2013} \\ 
  Swaindelphys cifellii & 2.14 & PBDB \\ 
  Symmetrodontomys simplicidens & 26.84 & cite{Tomiya2013} \\ 
  Syndyoceras cooki & 73865.41 & cite{Tomiya2013} \\ 
  Tachylagus gawneae & 5.70 & PBDB + regression \\ 
  Taeniolabis taoensis & 173.80 & PBDB + regression \\ 
  Talpavoides dartoni & 1.20 & PBDB + regression \\ 
  Talpavus conjunctus & 3.00 & PBDB + regression \\ 
  Talpavus duplus & 1.80 & PBDB + regression \\ 
  Talpavus nitidus & 1.40 & PBDB + regression \\ 
  Tamias ateles & 2.10 & PBDB + regression \\ 
  Tanymykter brachyodontus & 102744.44 & cite{Tomiya2013} \\ 
  Tapiravus validus & 64860.88 & cite{Tomiya2013} \\ 
  Tapirus simpsoni & 369534.73 & cite{Tomiya2013} \\ 
  Tapochoerus egressus & 35.70 & PBDB + regression \\ 
  Tapochoerus mcmillini & 21.10 & PBDB + regression \\ 
  Tapocyon dawsonae & 77.90 & PBDB + regression \\ 
  Tapocyon robustus & 87.70 & PBDB + regression \\ 
  Tardontia nevadans & 157.59 & cite{Tomiya2013} \\ 
  Tardontia occidentale & 5.60 & PBDB + regression \\ 
  Tarka stylifera & 34.10 & PBDB + regression \\ 
  Tatmanius szalayi & 3.04 & cite{Ferrusquia-Villafranca2006} \\ 
  Taxidea mexicana & 12.80 & cite{Bloch2007} \\ 
  Taxidea taxus & 47.09 & cite{Smith2004} \\ 
  Tayassu protervus & 98.60 & PBDB \\ 
  Teilhardina americana & 3.70 & PBDB + regression \\ 
  Teilhardina crassidens & 3.40 & PBDB + regression \\ 
  Teleoceras meridianum & 2022813.66 & cite{Tomiya2013} \\ 
  Teletaceras mortivallis & 155.00 & PBDB + regression \\ 
  Telmatherium altidens & 1236.90 & PBDB + regression \\ 
  Telmatherium cultridens & 560.00 & PBDB + regression \\ 
  Telmatherium manteoceras & 468.00 & PBDB + regression \\ 
  Temnocyon altigenis & 32532.67 & cite{Tomiya2013} \\ 
  Temnocyon percussor & 68871.66 & cite{Tomiya2013} \\ 
  Tenudomys bodei & 43.38 & cite{Tomiya2013} \\ 
  Tenudomys macdonaldi & 79.84 & cite{Tomiya2013} \\ 
  Tephrocyon rurestris & 13095.19 & cite{Tomiya2013} \\ 
  Tetonius ambiguus & 4.43 & cite{Brown1980} \\ 
  Tetonius matthewi & 4.80 & PBDB + regression \\ 
  Tetonius mckennai & 3.40 & PBDB + regression \\ 
  Tetraclaenodon puercensis & 55.60 & PBDB + regression \\ 
  Tetrapassalus mckennai & 0.90 & cite{Brown1980} \\ 
  Texomys ritchiei & 97.51 & cite{Tomiya2013} \\ 
  Thinobadistes segnis & 645890.00 & cite{McDonald1995} \\ 
  Thinocyon velox & 23.30 & PBDB + regression \\ 
  Thinohyus lentus & 101400.00 & cite{McKenna2011} \\ 
  Thisbemys corrugatus & 14.30 & PBDB + regression \\ 
  Thisbemys elachistos & 4.80 & PBDB + regression \\ 
  Thisbemys perditus & 10.80 & PBDB + regression \\ 
  Thisbemys uintensis & 20.20 & PBDB + regression \\ 
  Thomomys bottae & 7.85 & cite{Smith2004} \\ 
  Thomomys carsonensis & 2.00 & PBDB + regression \\ 
  Thomomys gidleyi & 37.34 & cite{Tomiya2013} \\ 
  Thryptacodon antiquus & 35.30 & PBDB + regression \\ 
  Thryptacodon australis & 28.30 & PBDB + regression \\ 
  Thryptacodon orthogonius & 20.60 & PBDB + regression \\ 
  Thryptacodon pseudarctos & 43.70 & PBDB + regression \\ 
  Thylacaelurus campester & 2.60 & PBDB + regression \\ 
  Thylacaelurus montanus & 3.60 & PBDB + regression \\ 
  Thylacodon pusillus & 3.10 & PBDB + regression \\ 
  Ticholeptus zygomaticus & 106937.52 & cite{Tomiya2013} \\ 
  Tillodon fodiens & 266.10 & PBDB + regression \\ 
  Tillomys senex & 3.60 & PBDB + regression \\ 
  Tinimomys graybulliensis & 1.40 & PBDB + regression \\ 
  Tinimomys tribos & 0.93 & cite{Skinner1972} \\ 
  Titanoides gidleyi & 260.10 & PBDB + regression \\ 
  Titanoides nanus & 223.10 & PBDB + regression \\ 
  Titanoides primaevus & 335.40 & PBDB + regression \\ 
  Tomarctus brevirostris & 17500.77 & cite{Tomiya2013} \\ 
  Tomarctus hippophaga & 13766.59 & cite{Tomiya2013} \\ 
  Torrejonia sirokyi & 10.40 & PBDB + regression \\ 
  Toxotherium hunteri & 157.70 & PBDB + regression \\ 
  Tregosorex holmani & 26.84 & cite{Tomiya2013} \\ 
  Tremarctos floridanus & 176.97 & cite{Smith2004} \\ 
  Trigenicus profectus & 73.20 & PBDB + regression \\ 
  Trigonias osborni & 542.90 & PBDB + regression \\ 
  Trigonias yoderensis & 316.90 & PBDB + regression \\ 
  Trigonictis cookii & 11.40 & PBDB + regression \\ 
  Trigonictis macrodon & 419.89 & cite{Tomiya2013} \\ 
  Triisodon quivirensis & 16.50 & PBDB + regression \\ 
  Trilaccogaulus ovatus & 103.54 & cite{Tomiya2013} \\ 
  Triplopus cubitalis & 82.00 & PBDB + regression \\ 
  Triplopus implicatus & 124.90 & PBDB + regression \\ 
  Triplopus obliquidens & 161.30 & PBDB + regression \\ 
  Triplopus rhinocerinus & 154.00 & PBDB \\ 
  Triplopus woodi & 131.13 & PBDB \\ 
  Tritemnodon agilis & 10780.00 & cite{Egi2001} \\ 
  Tritemnodon strenuus & 38.00 & PBDB + regression \\ 
  Trogolemur amplior & 3.50 & PBDB + regression \\ 
  Trogolemur myodes & 2.50 & PBDB + regression \\ 
  Trogomys rupinimenthae & 12.55 & cite{Tomiya2013} \\ 
  Trogosus castoridens & 178.20 & PBDB + regression \\ 
  Trogosus grangeri & 242.10 & PBDB + regression \\ 
  Trogosus latidens & 373.10 & PBDB + regression \\ 
  Tubulodon atopum & 6.50 & PBDB + regression \\ 
  Tubulodon taylori & 4.94 & PBDB \\ 
  Tuscahomys medius & 3.60 & PBDB \\ 
  Tuscahomys minor & 2.50 & PBDB \\ 
  Tylocephalonyx skinneri & 2018.54 & PBDB \\ 
  Uintaceras radinskyi & 492.30 & PBDB + regression \\ 
  Uintacyon asodes & 43.80 & PBDB + regression \\ 
  Uintacyon massetericus & 30.80 & PBDB + regression \\ 
  Uintacyon rudis & 22.90 & PBDB + regression \\ 
  Uintanius ameghini & 3.10 & PBDB + regression \\ 
  Uintanius rutherfurdi & 3.20 & PBDB + regression \\ 
  Uintasorex montezumicus & 0.60 & PBDB + regression \\ 
  Uintasorex parvulus & 1.00 & PBDB + regression \\ 
  Uintatherium anceps & 33.50 & PBDB + regression \\ 
  Untermannerix copiosus & 121.51 & cite{Tomiya2013} \\ 
  Unuchinia dysmathes & 6.37 & PBDB \\ 
  Uriscus californicus & 4.60 & PBDB + regression \\ 
  Urocyon cinereoargenteus & 3829.00 & PBDB \\ 
  Ursavus brevirhinus & 80000.00 & NOW \\ 
  Ursavus pawniensis & 61697.58 & cite{Tomiya2013} \\ 
  Ursavus primaevus & 90000.00 & NOW \\ 
  Ursus abstrusus & 20.70 & PBDB + regression \\ 
  Ursus americanus & 93431.00 & PBDB \\ 
  Utahia carina & 1.80 & PBDB + regression \\ 
  Utahia kayi & 2.70 & PBDB + regression \\ 
  Valenia wilsoni & 8.10 & PBDB + regression \\ 
  Vassacyon promicrodon & 41.10 & PBDB + regression \\ 
  Viverravus acutus & 4.55 & PBDB + regression \\ 
  Viverravus gracilis & 5.50 & PBDB + regression \\ 
  Viverravus laytoni & 3.60 & PBDB + regression \\ 
  Viverravus lutosus & 4.90 & PBDB + regression \\ 
  Viverravus minutus & 4.90 & PBDB + regression \\ 
  Viverravus politus & 5.83 & PBDB + regression \\ 
  Viverravus rosei & 3.06 & PBDB + regression \\ 
  Viverravus sicarius & 6.80 & PBDB + regression \\ 
  Vulpavus australis & 22.10 & PBDB + regression \\ 
  Vulpavus palustris & 28.80 & PBDB + regression \\ 
  Vulpavus profectus & 100.00 & cite{Williamson2013} \\ 
  Vulpes stenognathus & 7331.97 & cite{Tomiya2013} \\ 
  Vulpes velox & 28.28 & cite{Smith2004} \\ 
  Washakius insignis & 4.90 & PBDB + regression \\ 
  Washakius izetti & 3.90 & PBDB + regression \\ 
  Washakius woodringi & 3.10 & PBDB + regression \\ 
  Wilsoneumys planidens & 51.70 & PBDB \\ 
  Worlandia inusitata & 3.90 & PBDB + regression \\ 
  Wyolestes apheles & 6.80 & PBDB + regression \\ 
  Wyolestes iglesius & 6.33 & PBDB + regression \\ 
  Wyonycteris chalix & 1.30 & PBDB + regression \\ 
  Xenicohippus craspedotum & 10895.00 & cite{McKenna2011} \\ 
  Xenicohippus grangeri & 41.80 & PBDB + regression \\ 
  Ysengrinia americana & 110194.25 & cite{Tomiya2013} \\ 
  Yumaceras figginsi & 293607.76 & cite{Tomiya2013} \\ 
  Yumaceras hamiltoni & 247706.54 & cite{Tomiya2013} \\ 
  Yumaceras ruminalis & 314896.72 & cite{Tomiya2013} \\ 
  Zapus burti & 21.33 & cite{Tomiya2013} \\ 
  Zapus rinkeri & 1.90 & PBDB + regression \\ 
  Zapus sandersi & 18.92 & cite{Tomiya2013} \\ 
  Zemiodontomys burkei & 4.40 & PBDB + regression \\ 
  Zetamys nebraskensis & 60.34 & cite{Tomiya2013} \\ 
   \hline
\end{tabular}
\label{tab:mass_data}
\end{table}



% bibliography
\bibliographystyle{plain}
\bibliography{newbib,packages}

\end{document}
