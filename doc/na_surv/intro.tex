\documentclass[12pt,letterpaper]{article}

\usepackage{amsmath, amsthm}
\usepackage{graphicx,hyperref}
\usepackage{microtype, parskip}
\usepackage[numbers,sort&compress]{natbib}
\usepackage{lineno}
\usepackage{docmute}
\usepackage[font=small]{caption}
\usepackage{subcaption, multirow, morefloats}
\usepackage{wrapfig}
\usepackage{titlesec}
\usepackage{authblk, attrib, fullpage}
\usepackage{lineno}

\frenchspacing

\captionsetup[subfigure]{position = top, labelfont = bf, textfont = normalfont, singlelinecheck = off, justification = raggedright}

\begin{document}
\section{Introduction}

Extinction is one of most important components of diversification, second only to speciation. Why species go extinct at different rates remains one of the most fundamental questions in paleobiology \citep{Simpson1944,VanValen1973,Raup1991b,Raup1994,Quental2013,Wagner2014b,Jablonski2005,Payne2007,Kitchell1986}. Determining which and how biological traits influence extinction risk is vital for understanding the differential diversification of life during the Phanerozoic and making predictions about species' vulnerability to anthropogenic impacts. Yet investigations into the factors that affect extinction rate have proved inconclusive and often contradictory results. Additionally, \citet{VanValen1973} proposed the Law of Constant Extinction which states that the instantaneous rate of extinction for a taxon is independent of it's age, which is the basis for the Red Queen Hypothesis and is a cornerstone of quantitative paleobiology.

The fossil record of North American mammals is well resolved and represents an ideal data set for modeling long term evolutionary patterns \citep{Quental2013,Alroy2009,Alroy1996a,Alroy1998,Alroy2000g,Simpson1944,Blois2009,Tomiya2013,Marcot2014}. Here I take a Bayesian hierarchical survival model approach for understanding the effects of organismal and species traits, time of origination, and shared evolutionary history on species duration while also allowing for potentially time--dependent extinction. This is a direct extension of the more traditional dynamic and cohort survival analyses from paleontology \citep{Simpson1944,VanValen1973,Foote1988,Raup1978,Raup1975,VanValen1979,Baumiller1993,Sepkoski1975}.

Background extinction, or extinction occurring not at a mass extinction, is considered to involve many potential factors influencing the instantaneous extinction risk of any given species \citep{Jablonski1986,Wang2003,Harnik2013,Kitchell1986,Nurnberg2013a,Payne2007}. Factors such as geographic range have well known effects on survival \citep{Payne2007,Jablonski1987} because the effect size is large and thus easy to identify for a given sample size. The relationship between extinction risk and other traits, specifically organismal ones, is less well known because the effect size is most likely much smaller which makes inference difficult.

The approach taken here focuses on identifying time invariance effects of these traits on the expected duration of a species. Time invariant effect means that when comparing taxa over a long time period there is a consistent effect that is generalizable over the entire period of interest. While the strength of that effect may vary over time, overall there is a consistent expected effect of that trait of interest. In many ways, this was the point of \citet{Alroy2000g}: there was no consistent, time invariant response to climate change. While it is almost certain that selection pressures vary with time, consistent effects reveal fundamental differences in fitness or responses to selection. % A shift in time invariant factors may be an indicator of a change in ``macroevolutionary regime'' \citep{Jablonski1986}.

The organismal and species traits studied here are biogeographic occupancy, body mass, and both dietary and locomotor categories. This is a mixture of well established factors that potentially influencing extinction risk (i.e. occupancy and body mass) and less well understood ones (i.e. dietary and locomotor categories). Dietary category is of particular interest as it has been shown to influence diversification rate \citep{Price2012} though because of the limitations of many phylogenetic comparative methods it is unknown if this is due to differences in speciation, extinction, or both \citep{Rabosky2010a}.


\end{document}
