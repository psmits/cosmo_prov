\documentclass[12pt,letterpaper]{article}

\usepackage{amsmath, amsthm}
\usepackage{graphicx,hyperref}
\usepackage{microtype, parskip}
\usepackage[comma,sort&compress]{natbib}
\usepackage{lineno}
\usepackage{docmute}
\usepackage{subcaption, multirow, morefloats}
\usepackage{wrapfig}

\frenchspacing

\captionsetup[subfigure]{position = top, labelfont = bf, textfont = normalfont, singlelinecheck = off, justification = raggedright}

\begin{document}
\section{Introduction}

Extinction is one of most important components of diversification, second only to speciation. Why species go extinct at different rates remains one of the most fundamental questions in paleobiology \citep{Simpson1944,VanValen1973,Raup1991b,Raup1994,Quental2013,Wagner2014b,Jablonski2005,Payne2007,Kitchell1986}. Determining which and how biological traits influence extinction risk is vital for understanding the differential diversification of life during the Phanerozoic and making predictions about species' vulnerability to anthropogenic impacts. Yet investigations into the factors that affect extinction rate have proved inconclusive and often contradictory results. 

Here I take a Bayesian hierarchical survival model approach for understanding the effects of organismal and species traits, time of origination, and shared evolutionary history on species duration while also allowing for potentially time--dependent extinction. This is a direct extension of the more traditional dynamic and cohort survival analyses from paleontology \citep{Simpson1944,VanValen1973,Foote1988,Raup1978,Raup1975,VanValen1979,Baumiller1993,Sepkoski1975}. I am testing the following hypotheses: is extinction non-random with respect to ecology during times of background extinction, and is extinction taxon age independent among Cenozoic mammals? Cenozoic mammals represent an ideal group and time period because their fossil record is well sampled, well resolved both temporally and spatially, and the ecology and phylogeny of individual species are generally understood \citep{Alroy2009,Alroy2000g,Jernvall2002,Liow2008,Smith2004,Quental2013,Alroy1996a,Alroy1998,Simpson1944,Blois2009,Tomiya2013,Marcot2014}. 

This study focuses on identifying the time invariant effects of organismal and species traits on the expected duration of a species. A time invariant effect means that when comparing taxa over a long time period there is a consistent effect that is generalizable over the entire period of interest. While the strength of that effect may vary over time, the direction of that effect does not change. For example, geographic range size has been identified as a potentially time-invariant factor throughout the Phanerozoic, where large geographic range is associated with a decreased expected extinction risk \citep{Payne2007}.  In many ways, this was the point of \citet{Alroy2000g}: there was no consistent, time invariant response to climate change. While it is almost certain that selection pressures vary with time, consistent effects reveal fundamental differences in fitness or responses to selection. % A shift in time invariant factors may be an indicator of a change in ``macroevolutionary regime'' \citep{Jablonski1986}.

Background extinction, or extinction occurring not at a mass extinction, is considered to involve many potential factors influencing the instantaneous extinction risk of any given species \citep{Jablonski1986,Wang2003,Harnik2013,Kitchell1986,Nurnberg2013a,Payne2007}. Factors such as geographic range have well known effects on survival \citep{Payne2007,Jablonski1987} because the effect size is large and thus easy to identify for a given sample size. The relationship between extinction risk and other traits, specifically organismal ones, is less well known because the effect size is most likely much smaller which makes inference difficult.

Periods of background extinction provide a great opportunity to study how traits are related to survival because they represent the majority of geologic time, remain relatively predictable, and change slowly \citep{Jablonski1986,Raup1988}. The Law of Constant Extinction \citep{VanValen1973} states that a taxon's extinction risk within a given adaptive zone is age-independent (memoryless). This law is the foundation for the Red Queen hypothesis as well as most approaches for quantifying extinction. However there is some evidence contrary to this law \citep{Drake2014,Raup1975,Sepkoski1975,Finnegan2008}. By analyzing survival patterns within adaptive zones during periods of background extinction, it should be possible to both estimate the effects of various ecological strategies on survival and determine if extinction is age-independent or dependent. 

The organismal and species traits studied here are both dietary and locomotor categories, bioprovince occupancy, and body mass. Each of these traits describe different aspects of a taxon's adaptive zone such as energetic cost, population density, expected home range size, set of potential prey items, and dispersal ability \citep{Smith2004,Smith2008b,Damuth1981a,Damuth1979,Jernvall2004,Lyons2005,Lyons2010}. This is a mixture of well established factors that potentially influencing extinction risk (i.e. occupancy and body mass) and less well understood ones (i.e. dietary and locomotor categories). It is expected that species with larger geographic ranges have lower extinction rates than species with smaller geographic ranges \citep{Jablonski1986,Harnik2013,Nurnberg2013a,Jablonski2003,Roy2009c}, though this pattern may be random with respect to differences in organismal traits \citep{Raup1991b}. However, organismal traits directly related to species--environment interactions may play an important role in determining extinction risk.  The inclusion of organismal trait data are necessary to determine if and how both species and organismal traits contribute to differences in extinction risk or not. By modeling extinction via traits related to environmental preference, the relative importance of species and organismal level properties can be elucidated. 

Dietary category roughly describes the trophic relations of a taxon, a central component of its biotic environment. The categories used here are coarse groupings of similar ecologies: carnivore, herbivore, omnivore, and insectivore. The first three of these represent commonly used groupings of mammals in paleontological studies \citep{Jernvall2004,Price2012}, while the fourth is a biologically important grouping. \citet{Price2012} found that mammalian herbivores and carnivores have a greater diversification rates than omnivores which may indicate that these traits are better for survival. An increase in diversification can be due to either an increase in speciation relative to extinction or a decrease in extinction relative to speciation. Which scenario occurred, however, is impossible to determine from a phylogeny of only extant organisms \citep{Rabosky2010a}. By analyzing the fossil record of extinct organisms, the results of \citet{Price2012} can be better understood from a mechanistic perspective.

Locomotor categories describe the motility of a taxon, plausibility of occurrence in a particular habitat, and dispersal ability. Dispersal ability is important for determining both the extent of a taxon's geographic range and ability to track changing environments \citep{Birand2012,Jablonski2006a,Gaston2009} which then affects both extinction risk and community similarity. Here, mammals are categorized as either arboreal, ground dwelling, or scansorial. With the transition from primarily closed to open environments during the Cenozoic \citep{Blois2009,Janis1993a,Stromberg2005,Stromberg2013}, it is expected that arboreal taxa during the Paleogene will have a greater expected duration than Neogene taxa while the opposite will be true for ground dwelling taxa. In comparison, taxon duration of scansorial taxa is expected to remain relatively similar between the two time periods because it represents a mixed environmental preference that may be viable in either closed or open environments. 

Body size, here defined as mass, has an associated energetic cost in order to maintain homeostasis which in turn necessitates a supply of prey items. Many life history traits are associated with body size such as reproductive rate, metabolic rate, and home range size \cite{Peters1983a,Damuth1979,Brown1987,Smith2004}. Body size may affect extinction risk because as body size increases, home range size increases \citep{Damuth1979}. If individual home range size scales up to reflect a species geographic range, this would mean that extinction risk would decrease. Alternatively, it could be argued that as body size increases, reproductive rate decreases \citep{Johnson2002b}, populations get smaller \citep{White2007}, and generations get longer \citep{Martin1993a}, all of which increase extinction risk. A negative relationship between mammal body size and duration of genera has been observed \citep{Liow2008,Davidson2012} though this is inconsistent between continents \citep{Tomiya2013,Liow2008}. 

\end{document}
