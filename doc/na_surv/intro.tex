\documentclass[12pt,letterpaper]{article}

\usepackage{amsmath, amsthm}
\usepackage{graphicx,hyperref}
\usepackage{microtype, parskip}
\usepackage[comma,sort&compress]{natbib}
\usepackage{lineno}
\usepackage{docmute}
\usepackage{subcaption, multirow, morefloats}
\usepackage{wrapfig}

\frenchspacing

\captionsetup[subfigure]{position = top, labelfont = bf, textfont = normalfont, singlelinecheck = off, justification = raggedright}

\begin{document}
\section{Introduction}

Why species go extinct at different rates remains one of the most fundamental questions in paleobiology \citep{Simpson1944,VanValen1973,Raup1991b,Raup1994,Quental2013,Wagner2014b,Jablonski2005,Payne2007,Kitchell1986}. Determining which and how biological traits influence extinction risk is vital for understanding the differential diversification of life during the Phanerozoic and for making predictions about species' vulnerability to anthropogenic impacts. 

I am testing if extinction is non-random with respect to organismal- and species-level traits during times of background extinction, if and which traits have time-invariant effects on species duration, and if extinction is taxon-age independent among Cenozoic mammals? I use a Bayesian hierarchical survival modeling approach to estimate the effects of organismal and species traits, time of origination, and shared evolutionary history on species duration while allowing for potentially age--dependent extinction. This is a direct extension and statistical unification of the dynamic and cohort survival approaches from paleontology \citep{Simpson1944,VanValen1973,Foote1988,Raup1978,Raup1975,VanValen1979,Baumiller1993,Sepkoski1975}. Cenozoic mammals represent an ideal group and time period because their fossil record is well sampled, well resolved both temporally and spatially, and the ecology and taxonomy of individual species are generally understood \citep{Alroy2009,Alroy2000g,Jernvall2002,Liow2008,Smith2004,Quental2013,Alroy1996a,Alroy1998,Simpson1944,Blois2009,Tomiya2013,Marcot2014}. 

This study focuses on identifying the time-invariant effects of organismal and species traits on the expected duration of a species. A time-invariant effect is that when comparing taxa over a long time period there is a consistent effect that is generalizable over the entire period of interest. While the strength of that effect may vary over time, the direction does not change. For example, geographic range size has been identified as a generally time-invariant factor throughout the Phanerozoic, where large geographic range is associated with a decreased expected extinction risk \citep{Payne2007}. In many ways, this is a similar goal as \citet{Alroy2000g}: they found was no consistent, time-invariant evolutionary response to climate change. While it is almost certain that selection pressures vary with time, consistent effects reveal fundamental differences in fitness. Also, a shift in time-invariant factors may be an indicator of a change in ``macroevolutionary regime'' \citep{Jablonski1986} or a ``tipping point'' \citep{Barnosky2012a,Barnosky2011}, where the pattern of selection or ``rules'' are fundamentally changed or different between two time periods.

Background extinction, extinction not occurring at a mass extinction event, is considered to involve a mix of different factors which influence the instantaneous extinction risk of any given species \citep{Jablonski1986,Wang2003,Harnik2013,Kitchell1986,Nurnberg2013a,Payne2007}. Factors such as geographic range have well known effects on survival \citep{Payne2007,Jablonski1987} because the effect size is large and thus easy to identify. The relationship between extinction risk and other traits, specifically organismal ones, is less well known because the effect size is most likely much smaller, making inference difficult.

Periods of background extinction provide a great opportunity to study how traits are related to survival because they represent the majority of geologic time, remain relatively predictable, and change slowly \citep{Jablonski1986,Raup1988}. The Law of Constant Extinction \citep{VanValen1973} states that a taxon's extinction risk within a given adaptive zone is age-independent (memoryless). This law is the foundation for the Red Queen hypothesis as well as most approaches for quantifying extinction. However there is some evidence contrary to this law \citep{Drake2014,Raup1975,Sepkoski1975,Finnegan2008}. By analyzing survival patterns within adaptive zones during periods of background extinction, it should be possible to both estimate the effects of various ecological strategies on survival and determine if extinction is age-independent or dependent. 

The organismal and species traits studied here are both dietary and locomotor categories, bioprovince occupancy, and body mass. Each of these traits describe different aspects of a species' adaptive zone such as energetic cost, population density, expected home range size, set of potential prey items, and dispersal ability \citep{Smith2004,Smith2008b,Damuth1981a,Damuth1979,Jernvall2004,Lyons2005,Lyons2010}. This is a mixture of well established factors that potentially influence extinction risk (i.e. occupancy and body mass) and less well understood ones (i.e. dietary and locomotor categories). It is expected that species with larger geographic ranges have lower extinction rates than species with smaller geographic ranges \citep{Jablonski1986,Harnik2013,Nurnberg2013a,Jablonski2003,Roy2009c}, though this pattern may be random with respect to other traits \citep{Raup1991b}. However, organismal traits directly related to species--environment interactions may play an important role in determining extinction risk. By modeling extinction via traits related to environmental preference, the relative importance of species- and organismal-level properties can be elucidated. 

Dietary category roughly describes the trophic relations of a species, a central component of its biotic environment. The categories used here are coarse groupings of similar ecologies: carnivore, herbivore, omnivore, and insectivore. The first three of these represent commonly used groupings of mammals in paleobiological and macroevolutionary studies \citep{Jernvall2004,Price2012}, while the fourth is a biologically important grouping. \citet{Price2012} found that mammalian herbivores and carnivores have a greater diversification rates than omnivores which may indicate that these traits are better for survival. An increase in diversification can be due to either an increase in speciation relative to extinction or a decrease in extinction relative to speciation. Which scenario occurred, however, is currently impossible to determine from a phylogeny of only extant organisms \citep{Rabosky2010a}. By analyzing the fossil record of extinct organisms, the results of \citet{Price2012} can be decomposed into the relative contributions of speciation and extinction.

Locomotor categories describe the motility of a species, plausibility of occurrence in a particular habitat, and dispersal ability. Dispersal ability is important for determining both the extent of a species' geographic range and ability to track changing environments \citep{Birand2012,Jablonski2006a,Gaston2009} which then affects both extinction risk and community similarity. Here, mammals are categorized as either arboreal, ground dwelling, or scansorial. With the transition from primarily closed to open environments during the Cenozoic \citep{Blois2009,Janis1993a,Stromberg2005,Stromberg2013}, it is expected that arboreal species during the Paleogene will have a greater expected duration than Neogene species while the opposite will be true for ground dwelling species. In comparison, the durations of scansorial species are expected to remain relatively similar between the two time periods because it represents a mixed environmental preference that may be viable in either closed or open environments. 

Body size, here defined as mass, has an associated energetic cost in order to maintain homeostasis which in turn necessitates a supply of prey items. Many life history traits are associated with body size such as reproductive rate, metabolic rate, and home range size \citep{Peters1983a,Damuth1979,Brown1987,Smith2004}. Body size may affect extinction risk because as body size increases, home range size increases \citep{Damuth1979}. If individual home range size scales up to reflect a species geographic range, this would mean that extinction risk would decrease. Alternatively, it could be argued that as body size increases, reproductive rate decreases \citep{Johnson2002b}, populations get smaller \citep{White2007}, and generations get longer \citep{Martin1993a}, all of which increase extinction risk. A negative relationship between mammal body size and duration of genera has been observed \citep{Liow2008,Davidson2012} though this is inconsistent between continents \citep{Tomiya2013,Liow2008}. 

I analyze the effect of each of these species traits along with shared origination cohort and shared evolutionary history (i.e. phylogeny) in the context of a Bayesian survival model. Survival analysis is concerned with the time till an event \citep{Klein2003}, in this case the time from origination to extinction. The model used here is parameterized in a hierarchical, or mixed effects, context so that known structure in the data can be modeled along with the traits of interest. In that way, the importance and contribution to variance by species traits can be compared with these hierarchical effects.

The purpose of this study is the estimate the time-invariant effects of organismal- and speices-level traits upon expected duration while also estimating the effect of taxon age upon extinction risk. The Bayesian model used here allows for comparison of the marginal posterior estimates of each of the parameters acts as tests of the multiple trait effect hypotheses. All of these traits are analyzed with respect to the individual species origination cohort and phylogenetic position. From there the relative contribution species, cohort, and phylogeny to the unexplained variance can be estimated. Finally, the effect of time on extinction risk is explictly modeled allowing for inference about the applicability of the Law of Constant Extinction \citep{VanValen1973}.



\end{document}
