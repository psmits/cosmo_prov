\documentclass{letter}
\usepackage{microtype}
\frenchspacing

\signature{Peter D Smits}
\address{Committee on Evolutionary Biology \\ University of Chicago \\
         1025 E. 57th Street \\ Culver Hall 402 \\ Chicago, IL 60637}

\begin{document}
\begin{letter}{Editor \\ Science}
  \opening{Dear Editor,}
 
  Please fine enclosed my manuscript entitled: ``Death and taxa: time-invariant differences in mammal species duration'' which I am submitting for consideration as a Report in the journal \textit{Science}.

  In this study I tested for associations between extinction risk and multiple species-level traits during times of background extinction; which traits have time-invariant effects on species duration; and whether extinction is age-independent. The species-level traits analyzed here are geographic range size, body size, and both dietary and locomotor categories. Importantly, there are multiple competing hypotheses of how these traits effect species duration; either by increasing, decreasing, or having no effect on species duration.
 
  I was able to test all of these hypotheses simultaneously by using a statistical model of species duration as predicted by multiple species-level ecological traits while also taking into account with both temporal and shared evolutionary history (i.e. phylogeny). The results of my analysis are broadly consistent with the long standing ``survival of the unspecialized'' hypothesis as a time-invariant generalization of differences in species duration. I also find evidence of a decrease in extinction risk over the Cenozoic. Additionally, I find that both time of origination and shared evolutionary history both contribute to differences in species extinction risk.

  Finally, I compare the results of my analysis to known extinction risk factors for currently living mammal species from recent macroevolutionary and macroecological studies. I find that my results are partially incongruous with known risk factors, which has two major implications: that either the extinctions catalogued by fossil record are not a good predictor of further extinctions or that the current biodiversity crisis represents something completely different such as a mass extinction.

  This paper should have broad scientific interest for both methodological and empirical reasons. The statistical modeling approach used here is an example of the cutting edge of paleontological data analysis, including both temporal and evolutionary information along with the traits of interest. Normally, analysis is restricted to including either temporal or evolutionary information. This is an important step in dramatically improving our knowledge of macroevolutionary pattern and process. Empirically, the results of this paper are at the intersection of paleobiology/macroevolution along with both macroecology and conservation biology. Additionally, by including so many biological traits simultaneously in this model I was able to get better estimates of how these traits effect extinction risk and their relative importance, something which has not been possible in univariate or bivariate analyses common in the literature.

  Possible appropriate reviewers for this paper include Graham Slater (Smithsonian Institution, SlaterG@si.edu), Jonathan Marcot (University of Illinois at Urbana--Champagne, jmarcot@illinois.edu), Graeme Llyod (Maquarie University, graemetlloyd@gmail.com), and Samantha Price (University of California -- David, saprice@ucdavis.edu). All of these individuals have strong backgrounds in macroevolution, macroecology, and quantitative paleobiology. 

  Thank you for considering my work. Please send all correspondence regarding this manuscript to me via my email address (psmits@uchicago.edu).

  \closing{Sincerely,}

  \encl{Article; supplementary text, figures, tables.}
\end{letter}
\end{document}
