\documentclass{letter}
\usepackage{microtype}
\frenchspacing

\signature{Peter D Smits}
\address{Committee on Evolutionary Biology \\ University of Chicago \\
1025 E. 57th Street \\ Culver Hall 402 \\ Chicago, IL 60637}

\begin{document}
\begin{letter}{Editor \\ \textit{Proceedings of the National Academy of Sciences}}
  \opening{Dear Editor,}

  Please find enclosed my manuscript entitled: ``Death and taxa: time-invariant differences in mammal species duration'' which I am submitting for consideration as a Report in the journal \textit{Proceedings of the National Academy of Sciences}.

  In this study I tested for associations between extinction risk and multiple species-level traits using a fully Bayesian analysis of the fossil record of Cenozoic mammals from North America. The species-level traits analyzed here are geographic range size, body size, and both dietary and locomotor categories. Understanding which traits are important for long-term species survival is important for better understanding how anthropogenic effects are changing the nature of selection and possibly accelerating species extinction rates.
  
  The results of my analysis are broadly consistent with the long standing ``survival of the unspecialized'' hypothesis as a time-invariant description of the differences in species duration. I also find evidence of a decrease in extinction risk over the Cenozoic. Additionally, I find that both time of origination and shared evolutionary history both contribute to differences in species extinction risk. Normally, analyses are restricted to including either temporal or evolutionary information. By including both these factors, this is an important step forward in better understanding and quantifying the complexities underlying macroevolution pattern and process.
  
  Finally, I compare the results of my analysis to known extinction risk factors for currently living mammal species from recent macroevolutionary and macroecological studies. I find that my results are partially incongruous with known risk factors, which has two major implications: that either the extinctions catalogued by fossil record are not a good predictor of further extinctions or that the current biodiversity crisis represents something completely different such as a mass extinction.

  This study should have broad scientific interest for both empirical and methodological reasons. Empirically, this paper is at the intersection of paleobiology and macroevolution while having both macroecological and conservation biological implications. 

  Possible appropriate reviewers for this paper include Graham Slater (Smithsonian Institution, SlaterG@si.edu), Steve Wang (Swarthmore College, scwang@swarthmore.edu), Samantha Price (University of California -- Davis, saprice@ucdavis.edu), Graeme Llyod (Maquarie University, graemetlloyd@gmail.com), and Jonathan Marcot (University of Illinois at Urbana--Champagne, jmarcot@illinois.edu). All of these individuals have strong backgrounds in macroevolution, macroecology, and quantitative paleobiology. 

  Thank you for considering my work. Please send all correspondence regarding this manuscript to me via my email address (psmits@uchicago.edu).

  \closing{Sincerely,}

  \encl{Article; supplementary text, figures, tables.}
\end{letter}
\end{document}
