% latex table generated in R 3.1.2 by xtable 1.7-4 package
% Thu Jan  8 17:45:30 2015
\begin{table}[ht]
  \centering
  \caption{Summaries 1000 samples drawn from the marginal posteriors for the principle parameters of interest. Because of variable standardization, the intercept can be interpreted as the estimate for the mean observed species. As such, the other values are expected effects of trait values expressed as deviation from the mean. The categorical variables are binary index variables where an observation is of that category or not. \(\hat{R}\) values of less than 1.1 indicate chain convergence for the posterior samples.}
  \begin{tabular}{rrrrrrrrr}
    \hline
    & mean & sd & 2.5\% & 25\% & 50\% & 75\% & 97.5\% & \(\hat{R}\) \\ 
    \hline
    \hline
    alpha & 1.31 & 0.03 & 1.25 & 1.29 & 1.31 & 1.34 & 1.38 & 1.01 \\ 
    \hline
    intercept & -0.82 & 0.18 & -1.17 & -0.94 & -0.82 & -0.70 & -0.48 & 1.00 \\ 
    ground dwelling & -0.12 & 0.12 & -0.34 & -0.20 & -0.12 & -0.04 & 0.12 & 1.00 \\ 
    scansorial & -0.13 & 0.13 & -0.38 & -0.22 & -0.13 & -0.05 & 0.12 & 1.00 \\ 
    herbivore & 0.11 & 0.12 & -0.13 & 0.03 & 0.12 & 0.19 & 0.35 & 1.00 \\ 
    insectivore & 0.09 & 0.13 & -0.17 & 0.00 & 0.09 & 0.19 & 0.36 & 1.00 \\ 
    omnivore & -0.12 & 0.13 & -0.37 & -0.20 & -0.12 & -0.04 & 0.14 & 1.00 \\ 
    logit(occupancy) & -0.66 & 0.08 & -0.81 & -0.71 & -0.65 & -0.60 & -0.50 & 1.00 \\ 
    log(size) & -0.05 & 0.05 & -0.15 & -0.08 & -0.05 & -0.01 & 0.05 & 1.00 \\ 
    \hline
    sd cohort & 0.33 & 0.07 & 0.22 & 0.29 & 0.33 & 0.37 & 0.48 & 1.00 \\ 
    sd phylogeny & 0.21 & 0.10 & 0.07 & 0.14 & 0.19 & 0.26 & 0.46 & 1.05 \\ 
    \hline
  \end{tabular}
  \label{post_sum}
\end{table}
