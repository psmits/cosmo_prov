% latex table generated in R 3.1.2 by xtable 1.7-4 package
% Thu Jan  8 17:45:30 2015
\begin{table}[c]
  \centering
  \caption{Marginal posterior estimates for the praameters of interested based on 1000 posterior samples. The intercept can be interpreted as the estimate for the mean observed species. The other values are the effect of a trait on the expected species duration as expressed as deviation from the mean. The categorical variables are binary index variables where an observation is of that category or not. \(\hat{R}\) values of less than 1.1 indicate approximate chain convergence for the posterior samples.}
  \begin{tabular}{rrrrrrrrr}
    \hline
    & mean & sd & 2.5\% & 25\% & 50\% & 75\% & 97.5\% & \(\hat{R}\) \\ 
    \hline
    \hline
    alpha & 1.31 & 0.03 & 1.24 & 1.29 & 1.31 & 1.33 & 1.38 & 1.01 \\ 
    \hline
    intercept & -0.79 & 0.15 & -1.08 & -0.88 & -0.79 & -0.69 & -0.48 & 1.00 \\ 
    ground dwelling & -0.28 & 0.10 & -0.47 & -0.34 & -0.27 & -0.21 & -0.08 & 1.00 \\ 
    scansorial & -0.22 & 0.11 & -0.43 & -0.29 & -0.22 & -0.14 & 0.00 & 1.00 \\ 
    herbivore & 0.09 & 0.09 & -0.08 & 0.03 & 0.09 & 0.15 & 0.27 & 1.00 \\ 
    insectivore & 0.08 & 0.11 & -0.15 & 0.00 & 0.08 & 0.15 & 0.31 & 1.00 \\ 
    omnivore & -0.14 & 0.11 & -0.35 & -0.21 & -0.14 & -0.07 & 0.09 & 1.00 \\ 
    logit(occupancy) & -0.67 & 0.08 & -0.83 & -0.73 & -0.67 & -0.62 & -0.52 & 1.00 \\ 
    log(size) & -0.07 & 0.05 & -0.17 & -0.10 & -0.07 & -0.04 & 0.03 & 1.00 \\ 
    \hline
    sd cohort & 0.36 & 0.07 & 0.25 & 0.32 & 0.36 & 0.40 & 0.51 & 1.00 \\ 
    sd phylogeny & 0.11 & 0.06 & 0.03 & 0.07 & 0.10 & 0.14 & 0.23 & 1.12 \\ 
    \hline
  \end{tabular}
  \label{tab:post_sum}
\end{table}
