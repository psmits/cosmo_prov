\documentclass[12pt,letterpaper]{article}

\usepackage{amsmath, amsthm}
\usepackage{graphicx,hyperref}
\usepackage{microtype, parskip}
\usepackage[numbers,sort&compress]{natbib}
\usepackage{lineno}
\usepackage{docmute}
\usepackage[font=small]{caption}
\usepackage{subcaption, multirow, morefloats}
\usepackage{wrapfig}
\usepackage{titlesec}
\usepackage{authblk, attrib, fullpage}
\usepackage{lineno}

\frenchspacing

\captionsetup[subfigure]{position = top, labelfont = bf, textfont = normalfont, singlelinecheck = off, justification = raggedright}

\begin{document}
\section{Discussion}

The number and structure of the parameters from the fitted model allows for many different hypotheses to be addressed and inferences to be made. First, I interpret the results in terms of hypotheses of trait effects on survival. Second, I address implications for future paleobiological modeling. Finally, I discuss additional model improvements and complexities that can be used to address further hypotheses of species survival.

% interpretation
%   results are consistent with which hypotheses?
As expected, bioprovince occupancy has the largest effect on expected species duration/extinction risk. The other factors all appear to have small effect sizes which is also consistent with expectations.

I found that carnivores have a lower extinction risk than herbivores and insectivores, while ominivores have approximately equal or lower extinction risk than carnivores. Omnivorous taxa are also associated with a lower extinction risk than both herbivores and insectivores. These results sit nicely beside those of \citet{Price2012}.

Given that carnivores and omnivores have approximately equal extinction risk, and that \citet{Price2012} found that carnivores have a greater diversification rate than omnivores, this implies that carnivores have a greater origination rate than omnivores. In comparison, given the results of this study and \citet{Price2012}, this also implies that herbivores which have the greatest extinction risk must also have a very high origination rate in order to have the greatest diversification rate of these three categories. 

The large difference in time-invariant extinction risk between omnivores and both herbivores and insectivores is most likely related to the concept of ``survival of the unspecialized'' where less specialized taxa have lower extinction risks \citep{Liow2004a,Simpson1944}. Because larger effects are easier to identify in coarse or small data sets, the magnitude of this effect also explains both the early identification and origin of this hypothesis of time-invariant effects on survival \citep{Simpson1944}.

The comparisons between the effects of locomotor category on expected species extinction risk are consistent with the hypotheses that arboreality is associated with a greater expected extinction risk than either with scansoriality and ground dwelling taxa. Importantly, scansoriality appears to not influence any difference in extinction risk when compared with ground dwelling taxa. This can be interpreted that arboreal taxa, which require a specific kind of environment, may be more prone to extinction because the lack of permanency of those environments may prevent species persistence. 

The very weak if non-existent effect of body size on extinction risk, which can be considered not ``significant'', is consistent with \citet{Tomiya2013}. The direction/sign of the modal estimate of effect is not consistent with the prediction of increase in extinction risk associated with increase in body size \citep{Liow2008}. Importantly, however, the other studies were performed at the generic-level which may or may not involve different processes that are not included in this species-level model \citep{Liow2008,Tomiya2013}.

%   VPC and what that means about cohort and shared evolutionary history
While explicit phylogenetic relations between taxa are frequently not modeled in paleobiological studies of diversity \citep{Alroy2009,Foote2013,Jablonski2006a,Hunt2007a,Liow2008,Payne2007,Alroy2000g,Jernvall2002,Jernvall2004,Marcot2014}, there have been more recent studies which analyze fossil diversification in an explicitly phylogenetic context \citep{Slater2012,Slater2013a,Tomiya2013,Harnik2014,Simpson2011a}. The partitioning of the different sources of variance involved in this model indicate that phylogeny or shared evolutionary history accounts for approximately 10-15\% of the unexplained variance each. Because VPC of phylogeny is greater than 0, it is not appropriate to ignore phylogeny when modeling survival in paleontological studies \citep{Housworth2004}. An addition 10-15\% of unexplained variance was due to shared origination cohort. Between these two sources of variance, it is clear that a shared evolutionary history and temporal occurrence are non-ignorable in paleontological studies or survival. Modeling phylogeny as a hierarchical effect, as done here, is a very simple and interpretable means of integrating phylgoenetic information into any and all regression models \citep{Lynch1991,Housworth2004} and is most likely ideal for most paleontological studies.
% This is where WAIC might make some people believe me.

% cohort effect
While the observed pattern of older cohorts having increased extinction risk compared to younger cohorts is weak, it is notable because the shift from older, higher risk cohorts to younger, lower risk cohorts coincides approximately with the Paleogene--Neogene boundary. The shift from the Paleogene to the Neogene is marked by the transition from a principally closed, forested environment to a principally open, grassland environment \citep{Stromberg2005,Janis1993a,Janis2000}.  % MORE

% increasing extinction risk with age
% two potential reasons
%   older cohorts getting aged out
%     see possible tipping point/state shift at Paleogene-Neogene
%   minimum resolution
% may be a balance between the two 
%   need to model paleogene vs neogene effect on species-level traits


% implications
It is possible to think of the observed time-invariant effects modeled here as the signature of the Cenozoic background extinction macroevolutionary regime \citep{Jablonski1986}. One of the open questions in paleobiology and macroecology is whether the current biodiversity crisis qualified as a mass extinction \citep{Alroy2010,Barnosky2011,Barnosky2012a}. Because change in the magnitude of extinction risk is not necessarily the best indicator of a shift from background to mass extinction \citep{Wang2003}, it is more fruitful to look for changes in the direction selection, loss of a selective pressure, or appearance of novel selective pressures. Comparison of the estimated effects of organismal- and species-level traits analyzed here with previous studies demonstrates a mixture of congruence and incongruence. 

As expected, large range size is always associated with lower extinction risk in the Recent \citep{Fritz2009,Fritz2010b,Liow2009,Purvis2000a}. The primacy of geographic range as a time-invariant factor influencing extinction risk has been found for marine invertebrates across the Phanerozoic and in particular at the Cretaceous mass extinction event \citep{Jablonski1986,Payne2007}.

While I found that body size has no time-invariant effect on extinction risk, large body size is associated with increased extinction risk in the Recent, though this is variable across environments and clades \citep{Liow2009,Fritz2009,Purvis2000a}. 

A higher trophic level (e.g. carnivory versus herbivory) is associated with greater extinction risk in Primates and Carnivora \citep{Purvis2000a} which is not congruous with the results found here that carnivores have lower extinction risk than herbivores. 

Finally, phylogeny has been found to be a factor underlying current mammal species extinction risk, though this effect seems much greater in the Recent than for the whole Cenozoic \citep{Fritz2010b}. Note that the phylogeny of Recent mammals is much better than the primarily taxonomy based phylogeny used here, which may partially account for the difference in effect.

How much of these incongruities are due to normal time-variant effects is unknown, though these comparisons across multiple different factors do point to the potential shift in macroevolutionary regime \citep{Jablonski1986} or arrival at a tipping point \citep{Barnosky2012a,Barnosky2011}.

There are a few data quality concerns in this study which are also inherent to almost any paleontological study.

Almost all of the body mass estimates were obtained using published regression equations that estimate mass from some other body part (e.g. tooth). These estimates are known with error, which was not included in the model. If the standard deviation of the residuals from each of these regression equations was known, it would be possible to directly model this as measurement error \citep{Gelman2013d}. 

A similar situation occurs with species bioprovince occupancy. Depending on the structure of the biogeographic network, there can be a range in the number of emergent bioprovinces. By estimating the standard deviation of both the number of bioprovinces and the number of occupied bioprovinces, it might be possible in propegate this measurement error correctly through the model \citep{Gelman2013d}. 

These model improvements were not done here for a variety of reasons: adequacy of current model fit, lack of residual standard deviation information from regression equations, and for improved tractability. Both of these measurement error models involve estimating the actual value given the observed and some known amout of error. Because of this, by allowing two covariates to be known with error approximately 4000 more (nuisance) parameter values would need to be estimated. Because of the combination of these factors, no measurement error in body size or bioprovince occupancy was included.

The phylogeny used here is only a coarse, baseline estimate of the actual species relationships. Because of this, the analysis of phylogenetic effect on survival represents a minimum or rough estimate. With improved topology and resolution, it would be possible to more accurately estimate the effect of shared evolutionary history. As it stand, given the rough minimum estimate of phylogenetic heritability, these results point to the importance of including shared evolutionary history in diversification models.

The minimum resolution of the fossil record might cause an upward bias in estimates of the Weibull shape parameter \(\alpha\) \citep{Sepkoski1975}. This effect can be observed by the initial plateau in the K-M estimate of \(S(t)\) for the observed (Fig. \ref{fig:ppc_surv}). This plateau is a hallmark of the survival analyses by \citet{VanValen1973} which \citet{Sepkoski1975} identified as partially a product of minimum resolution of the fossil records of the different studied groups.

% relative importance of time-invariant and time-variant effects?
% not dealt with in this model, potential avenue of future model improvement

% take home point
%   background extinction is highly variable
%   non-random with respect to a lot of biology
%     this is partially heritable
%   extinction is found to be partially age dependent (accelerating)
%     without modeling fossilization process hard to be sure
%     this result may be due to minimum resolution of fossil record (Sepkoski)

There are many processes encompassed by background extinction and identifying the exact cause of any one species' reason for extinction is extremely difficult. By focusing on estimating the effects of different ecologies and historical factors on average extinction risk, it is possible to better understand what processes may have driven the resulting pattern of selection (i.e. diversity). Here, I focused on time-invariant factors and their relation to biological selectivity of extinction, possible reasons for the observed time-invariant effects, and the effects of taxon-age on extinction risk. I found that some organismal- and species-level traits such as omnivory and large geographic range size have time-invariant effects on mammal species extinction risk. I also found that there are small but non-ignorable effects of cohort and phylogeny. Finally, I found putative evidence of increasing extinction risk with species age, though this result may be partially due to the minimum resolution of the fossil record itself \citep{Sepkoski1975}.


\end{document}
