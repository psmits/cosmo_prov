\documentclass[12pt,letterpaper]{article}

\usepackage{amsmath, amsthm}
\usepackage{graphicx,hyperref}
\usepackage{microtype, parskip}
\usepackage[numbers,sort&compress]{natbib}
\usepackage{lineno}
\usepackage{docmute}
\usepackage[font=small]{caption}
\usepackage{subcaption, multirow, morefloats}
\usepackage{wrapfig}
\usepackage{titlesec}
\usepackage{authblk, attrib, fullpage}
\usepackage{lineno}

\frenchspacing

\captionsetup[subfigure]{position = top, labelfont = bf, textfont = normalfont, singlelinecheck = off, justification = raggedright}

\begin{document}
\section{Discussion}

% interpretation
%   results are consistent with which hypotheses?
As expected based on previous work, bioprovince occupancy has the single largest effect of expected species duration. The other factors all appear to have small effect sizes which is also consistent with expectations.

The comparisons between the effects of locomotor category on expected species duration are consistent with my hypotheses that arboreality has the lowest effect on expected duration when compared with scansoriality and ground dwelling taxa. Importantly, scansoriality appears to not influence any difference in duration when compared with ground dwelling taxa. This can be interpreted that arboreal taxa, which require a specific kind of environment, may be more prone to extinction because the lack of permanency of those environments may prevent species persistence. 

I found that carnivores are expected to have a greater average duration than herbivores and insectivores, while ominivores have approximately equal or greater expected durations than carnivores. Omnivorous taxa are also associated with a greater expected duration than both herbivores and insectivores. These results sit nicely beside those of \citet{Price2012}.

Given that carnivores and omnivores have approximately equal expected durations, and that \citet{Price2012} found that carnivores have a greater diversification rate than omnivores, this implies that carnivores have a greater origination rate than omnivores. In comparison, given the results of this study and \citet{Price2012}, this also implies that herbivores which have the lowest expected duration must have a very high origination rate in order to have the greatest diversification rate of these three categories.

The very weak, if non-existent, effect of body size on expected duration, which can be considered not ``significant'', is consistent with \citet{Tomiya2013}. The direction/sign of the effect, however, is not consistent with the prediction of decrease of duration associated with increase in body size \citep{Liow2008}. Importantly, however, the other studies were performed at the generic level which may or may not involve different processes that are not included in this species level model \citep{Liow2008,Tomiya2013}.

%   VPC and what that means about cohort and shared evolutionary history
While explicit phylogenetic relations between taxa are frequently not included in paleobiological studies of diversity \citep{Alroy2009,Foote2013,Jablonski2006a,Hunt2007a,Liow2008,Payne2007,Alroy2000g,Jernvall2002,Jernvall2004,Marcot2014}, there have been more recent studies which analyze fossil diversification in an explicitly phylogenetic context \citep{Slater2012,Slater2013a,Tomiya2013,Harnik2014,Simpson2011a}. The partitioning of the different sources of variance involved in this model shows that phylogeny or shared evolutionary history accounts for approximately 10-15\% of the unexplained variance. Because VPC of phylogeny is greater than 0, it is not appropriate to ignore phylogeny when modeling survival in paleontological studies \citep{Housworth2004}. An addition 10-15\% of unexplained variance was due to shared origination cohort. Between these two sources of variance, it is clear that a shared evolutionary history and temporal occurrence are non-ignorable in paleontological studies or survival.


% implications
Comparison of time-invariant effects from the Cenozoic, the ``signature'' of a regime, to the predictions of selection for current biodiversity crisis. This presuposes that the current biodiversity crisis may involve fundamentally different selection pressures than experienced over the Cenozoic.
%   conservation
%     compare with other results
%   ``macroevolutionary regime''
% future
%   interaction between traits and/or hierarchical effects
%   time transient effects


\end{document}
