\documentclass{article}
%\documentclass{pnastwo}

\usepackage{amsmath, amsthm, amsfonts, amssymb}
\usepackage{graphicx,hyperref}
\usepackage{microtype, parskip}
\usepackage[comma,sort&compress]{natbib}
\usepackage{docmute}
\usepackage{subcaption, multirow, morefloats}
\usepackage{wrapfig, rotating}


\captionsetup[subfigure]{position = top, labelfont = bf, textfont = normalfont, singlelinecheck = off, justification = raggedright}

\title{Corrections to ``Expected time-invariant effects of biological traits on mammal species durations''}
\author{Peter D Smits\\Committee on Evolutionary Biology, University of Chicago}
%\author{Peter D Smits\affil{1}{Committee on Evolutionary Biology, University of Chicago, Chicago, Illinois, USA}}

\begin{document}

\maketitle

A series of errors in the body mass data recorded in the Supplementary Dataset 1 of \citep{Smits2015} was recently brought to my attention. These errors, which were included in the analyses presented, took three forms: 1) improperly back-transforming body mass data from \citet{Smith2004}, 2) incorrectly inserting the measurement of the part used to estimate body mass into the body mass column for the ``PBDB + regression'' observations, and 3) improperly transforming rodent log mass estimated as a part of this study. The first error affects 24 of 1936 observations, the second affects approximately 1000 of the observations, and the third error affected 333 of the estimates already affected by the second error.

All three errors were caused by mistakes in the code processing the collected body size data. The first error was due to the body mass data recorded by \citet{Smith2004} was log base 10 transformed and not natural log transformed; I made the mistake and back-transformed by raising \(e\) to that power instead of 10. The second error stems from the code used from processing body mass data from the different sources considered in this study. These columns which are processed as ``PBDB + regression'' have an intermediate step with four columns; I used the incorrect column call which caused this error. The third error was involved with predicting rodent mass from lower first molar area. The regression formula used to estimate mass uses log of area of lower first molar and gives log mass. Instead of exponentiating that log mass to put it in grams, I accidentally log transformed the log mass value again. This produces a number of ``mass'' estimates below 2 grams, which is unrealistic given that this is the approximate minimum for mammal body size \citep{Smith2004,Smith2011f}. 

The original analyses of \citep{Smits2015} were re-run using the updated and corrected body mass estimates. Quantitatively, the subtle aspects of the estimates have moved slightly but not substantially enough to change the original results. Qualitatively, there is no difference in the conclusions that can be made from these new results. The changes to the parameter estimates that occurred are most apparent in Table \ref{tab:new_res}, which is a corrected version of table 1 from \citet{Smits2015}; the mean and median of the posterior distributions for each parameter have changed by approximately 0.01. For the sake of brevity, the newly made figures from this re-analysis are presented in the new Supplementary Information for this correction.

An archive of the original, uncorrected code is available at http://dx.doi.org/10.5281/zenodo.44365. An archive of the new, corrected code is available at https://github.com/psmits/cosmo\_prov. The first two errors take place in file R/body\_size.r on lines 48-49 and 160-161, respectively. The third error is found in file R/predict\_mass.r on line 207. No other code changes were necessary to correct these errors.

\textbf{Acknowledgements:} The author would like to thank Chrstine Janis for bringing these inconsistencies to his attention. The author would also like to thank Kenneth Angielczyk and Michael Foote for advice.

\begin{table}[c]
  \centering
  \caption{Marginal posterior estimates for the parameters of interested based on 1000 posterior samples. The intercept \(\beta_{0}\) can also be interpreted as the estimate for the mean observed species. The remaining \(\beta\) values can be interpreted as the effect of a trait on the expected species duration as expressed as deviation from the mean. The categorical variables are binary index variables where an observation is of that category or not. See \citet{Smits2015} for explainations of the parameters and their estimates.}
  \begin{tabular}{ l l r r r r r r r r }
    parameter & effect & mean & sd & 2.5\% & 25\% & 50\% & 75\% & 97.5\% & \(\hat{R}\) \\ 
    \hline
    \(\alpha\) & ``age'' &1.29 & 0.03 & 1.23 & 1.27 & 1.29 & 1.31 & 1.36 & 1.02 \\ 
    \hline
    \(\beta_{0}\) & arboreal/carnivore & -0.78 & 0.14 & -1.06 & -0.88 & -0.78 & -0.69 & -0.50 & 1.01 \\ 
    \(\beta_{o}\) & occupancy & -0.57 & 0.08 & -0.73 & -0.63 & -0.57 & -0.52 & -0.41 & 1.00 \\ 
    \(\beta_{size}\) & body size & 0.01 & 0.05 & -0.08 & -0.02 & 0.02 & 0.05 & 0.11 & 1.00 \\ 
    \(\beta_{g}\) & ground dwelling & -0.27 & 0.09 & -0.45 & -0.33 & -0.27 & -0.21 & -0.08 & 1.00 \\ 
    \(\beta_{s}\) & scansorial & -0.21 & 0.11 & -0.43 & -0.28 & -0.21 & -0.14 & -0.01 & 1.00 \\ 
    \(\beta_{h}\) & herbivore & 0.08 & 0.09 & -0.09 & 0.02 & 0.08 & 0.14 & 0.26 & 1.00 \\ 
    \(\beta_{i}\) & insectivore & 0.08 & 0.11 & -0.13 & 0.01 & 0.08 & 0.15 & 0.29 & 1.00 \\ 
    \(\beta_{o}\) & omnivore & -0.13 & 0.10 & -0.34 & -0.20 & -0.13 & -0.06 & 0.07 & 1.00 \\ 
    \hline
    \(\sigma_{c}\) & sd cohort & 0.34 & 0.07 & 0.24 & 0.30 & 0.33 & 0.38 & 0.50 & 1.00 \\ 
    \(\sigma_{p}\) & sd phylogeny & 0.11 & 0.06 & 0.02 & 0.07 & 0.10 & 0.15 & 0.25 & 1.07 \\ 
    \hline
  \end{tabular}
  \label{tab:new_res}
\end{table}



\pagebreak 

\bibliographystyle{abbrvnat}
%\bibliographystyle{pnas}
\bibliography{biblio}

\end{document}
