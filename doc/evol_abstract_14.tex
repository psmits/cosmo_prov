\documentclass{article}

\usepackage{parskip, microtype}

\frenchspacing

\title{Cenozoic mammals and the biology of extinction}

\author{Peter D. Smits\\ psmits@uchicago.edu}

\begin{document}

\maketitle

Extinction is expected to be non-random with respect to biology. Determining how different traits, both alone or in concert, influence extinction risk is extremely important for understanding the differential diversification of taxa. Traits relating to environmental preference are good candidates for modeling differential extinction because it is expected that, based purely on stochastic grounds, taxa which prefer rare environments would have a greater extinction risk than taxa which prefer abundant environments. Importantly, the Law of Constant Extinction posits that extinction risk is random with respect to taxon age. This statement has come increasingly under fire and its generality is possibly suspect. By fitting different theoretical distributions of survival to empirically observed durations, the generality of this statement can be tested. 

Trait mediated survival was studied at both the generic and specific level for the record of Cenozoic mammals using dietary and locomotor categories, as well as body size. These traits were selected because they relate directly to environmental preference. Preliminary analysis of specific level survival as predicted from dietary and locomotor categories indicated that North American and European mammals had fundamentally different patterns of taxon duration. North American species survival was best predicted by locomotor category while European species survival was best predicted by dietary category. Also, the distribution of North American species durations was best modeled as exponentially distributed (constant extinction risk) while European species durations were better modeled by a Weibull distribution (monotonic nonconstant extinction risk). Even with further refinements to this analysis, these results do highlight how regional level processes may shape taxonomic patterns in fundamentally different ways.  

\end{document}
