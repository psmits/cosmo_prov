\documentclass{article}

\usepackage{parskip, microtype}

\frenchspacing

\title{Cenozoic mammals and the biology of extinction}

\author{Peter D. Smits\\ psmits@uchicago.edu}

\begin{document}

\maketitle

Extinction is expected to be non-random with respect to biology. Determining how different traits, both alone or in concert, influence extinction risk is extremely important for understanding the differential diversification of taxa over the Phanerozoic. Traits relating to environmental preference are good candidates for modeling differential extinction. A simple expectation based purely on stochastic grounds is that taxa with a preference for rare environments will have a greater extinction risk than taxa which prefer abundant environments. Importantly, the Law of Constant Extinction posits that extinction risk is random with respect to taxon age. This statement has become increasingly under fire and its generality is possibly suspect. By fitting different theoretical distributions of survival to empirically observed durations, the generality of this statement can be tested. 

Trait mediated survival was studied using the fossil record of Cenozoic mammals using dietary category, locomotor category, and body size. These traits were selected because they relate directly with environmental preference. Importantly, these traits can be estimated from fossil remains. Analysis was conducted at both the generic and specific level. Preliminary results from analysis of specific level survival as predicted from dietary and locomotor categories indicates that North American and European mammals have fundamentally different patterns of trait mediated survival. North American species survival is best best modeled via locomotor category while European species survival is best predicted via dietary category. Also, the distribution of North American species durations is best modeled as exponentially distributed (constant extinction risk) while European species durations are better modeled by a Weibull distribution (monotonic nonconstant extinction risk). While there are many further refinements necessary for this analysis, these preliminary results do highlight how regional level processes may shape taxonomic patterns in fundamentally different ways.  
\end{document}
