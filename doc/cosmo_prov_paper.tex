% where to send?
%   AmNat 21 pages, 6 figures/tables
%   Paleobiology 50 pages

\documentclass[12pt,letterpaper]{article}
\usepackage{amsmath, amsthm}
\usepackage{parskip, graphicx, microtype}
\usepackage{caption, subcaption, multirow}
\usepackage{morefloats, hyperref}
\usepackage{rotating, longtable}
\usepackage{authblk}
\usepackage[sort&compress]{natbib}
\usepackage{fullpage}

\usepackage{lineno}

\usepackage{lipsum}

\frenchspacing


\title{Cosmopolitan and provincialism dynamics of North American terrestrial mammals across the Cenozoic}
\author[1]{Peter D Smits}
\author[2]{John Alroy}
\affil[1]{Committee on Evolutionary Biology, University of Chicago}
\affil[2]{Department of Biological Sciences, Macquarie University}

\begin{document}

\maketitle

\linenumbers
\modulolinenumbers[2]

\begin{abstract}
  % This is where the abstract would go.
  % Paleobiology: 250-300 words
  %
  \noindent (Keywords: macroevolution, paleoecology, herbivory, trophic structure, biogeography)
\end{abstract}

\section{Introduction}
Evolutionary paleoecology is the study of how ecological traits expressed at any level affect the macroevolutionary process, or long term patterns in speciation and extinction \citep{Kitchell1985a}. Community structure plays a fundamental roll in determining which biotic interactions are possible at a given time. The importance of biotic interactions in understanding evolutionary dynamics
% evolutionary paleoecology
%   Kitchell1985
%   what organisms interact?
%     macroecology 
%       mammal work focuses on body size and range size
%     landscape dynamics limit the types of biotic--biotic interactions that are possible
%   how does community structure change over time?
%     consitituent taxa
%       paleo record complicates sampling
%         use site occupancy/distribution Sidor2013,Jerval2004
%       preservation of structure Damuth1982
%     site similarity
%       common herbivores drive trends Jernvall2002
%         cosmopolitan taxa drive trends
%       stability/stationary relationship between trophic groups Jernvall2004
%         herbivore group is largest
%         resilience and locality coverage 
%           comparison of three dietary categories
%     network analysis
%       food webs
%         community stability 
%           Permo-Triassic boundary Angielczyk2005,Roopnarine2007
%           late Cretaceous terrestrial Mitchell2012
%         methods Roopnarine2006,Roopnarine2010
%       biogeographic networks
%         Sidor2013,Vilhena2013

% questions
%   what is the pattern of organismal distribution over the Cenozoic?
%     how does taxonomic abundance affect this pattern?
%     how does life history affect this pattern?
%       diet Jernvall2004
%       locomotor category
%     how does climate affect this pattern?
%       abiotic factors Alroy2000g,Zachos2001,Zachos2008
%     do the most common taxa underly this pattern?
%     does the dominate category underly this pattern?
%   are these patterns different across continents

% study system
%   locality presence
%   continents
%     North America
%       Alroy
%       Lyons, Felisa Smith
%     Europe
%       Jernvall2002,Jervall2004
%     South America
%       Flynn1998a,Macfadden2006,Macfadden1997
%       unique degree of endemism
In order to investigate how cosmopolitan--endemism dynamics vary across the Cenozoic, the terrestrial mammal records of North America, Europe and South America. The mammalian taxonomic dynamics of each of these three continents have been studied to different degrees \citep{Jernvall2002,Jernvall2004,Flynn1998a,Macfadden2006,Macfadden1997,Alroy1996a,Blois2009,Eronen2009,Gunnell1995,Raia2006}. While all of these continents transitioned from a principally closed and forested environment to a open and savanna like environment over Cenozoic, the timing of the start of this transition was different on each continent \citep{Blois2009,Eronen2009,Janis1993b}. Additionally, the mammal lineages present on each continent are divergent especially in the case on the isolated continent of South America \citep{Flynn1998a,Macfadden2006,Macfadden1997}. Because of this it can be expected that there will be a mix of similarities and striking differences between the continents in how taxonomic distributions change over the Cenozoic.


% predictions
%   general
%     locomotion
%       opening up of landscape over Cenozoic
%         habitat availability 
%           increase in land dwelling taxa
%             increased connectedness, decreased endemism
%           decrease in arboreal taxa
%             decrease connectedness, increased endemism
%     diet
%       shift to grassland dominated environments
%       herbivore
%         link between herbivory and both arboreality and land dwelling
%           increased connectedness, decreased in endemism
%       omnivore
%         stationary
%         highest variance in biogeographic network summaries
%         unremarkable endemism
%       carnivore
%         stationary
%         high endemism
%     major pattern patterns driven by common taxa
%       common taxa drive trend Jernvall2002,Jernvall2004
As the environment on each continent moved from a closed, forested environment to an open, savanna-like environment there are a few general predictions of shifts in cosmopolitan--endemism dynamics. It is expected that as the environment shifted, the ratio of arboreal to land dwelling mammals would shift to include an even greater proportion of land welling mammals. Additionally, it is expected there would be an increased similarity between localities, as reflected by decreased endemism in land dwelling taxa. Arboreal taxa, however, are expected to trend towards increasing endemism and decreased site similarity as driven by increasingly fragmented forests.

The shift to grassland environments is frequently connected with the shift from browser-dominated communities to grazer-dominated communities CITATIONS, however in this study the focus is instead on coarser trophic categories such as herbivore and carnivore. There is an expected correlation between herbivore dynamics and land dwelling mammal dynamics because of the expansion of terrestrial herbivores (e.g. horses) CITATIONS.

%   continents
%     SA
%       extreme biome provincialism between north and south
%         increasing endemism over Cenozoic
%         effect of record?
South America is expected to have experienced rather different cosmopolitan--endemism dynamics than both North America and Europe because of both its long-term isolation and extremely biome provincialism \citep{Flynn1998a,Macfadden2006} OTHER CITATIONS. The Amazonian lowland and Andean highlands are radically different biomes that diverged in similarity approximately during the Miocene \citep{Pascual1990,Ortiz-Jaureguizar2006} and are now a rainforest and an arid desert, respectively. Because of this, it is expected that there may be increasing and extreme endemism over the Cenozoic. However, the relative paucity of the South American record in comparison to North America and Europe may be a potential confounding factor for interpreting these results.

%     SA versus NA versus Eurasia
%       early grasslands in SA?
%       largest savanna in NA
%       Eurasia is intermediate between NA and SA for endemism



\section{Materials and Methods}
Mammalian occurrence information was obtained from the Paleobiology Database (PBDB; \url{http://www.paleodb.org}). Occurrence information was restricted to terrestrial mammals from the North American fossil record of the Cenozoic. For each occurrence, the locality information was also recorded, most importantly formation name and estimated age. Hierarchical taxonomic information was recorded for each taxon. All partially, ambiguously, or incompletely identified genera or species (e.g. aff., cf., ?) were excluded from analysis. 

% European fossils
%   PBDB or NOW
% South American fossils
%   field museum

For each taxon, dietary and life habit information recorded in the PBDB was gathered. Dietary information was then simplified into three categories: herbivore, omnivore, and carnivore. Herbivorous taxa were the amalgam of the PBDB classifications herbivore, grazer, browser, folivore, and granivore. Omnivorous taxa were the amalgam of frugivores and omnivores. Finally, carnivores were the amalgam of all carnivores and insectivores. These three categories were chosen because they represent coarse groups which are identifiable from most mammalian teeth which are the primarily mammalian fossil material. Additionally, these groupings have been used in prior analysis of the effect of ecology on site similarity \citep{Jernvall2004}.

Temperature information was estimated using the \(\delta O^{18}\) isotope information from the benthic foram record for the entire Cenozoic \citep{Zachos2008}. An increase in \(\delta O^{18}\) levels are associated with a decrease in atmospheric temperature \citep{Zachos2001,Zachos2008}. Benthic foram \(\delta O^{18}\) information has been used previously as climatic information in studies of mammalian macroevolutionary patterns \citep{Alroy2000g,Figueirido2012,Rose2011}.
%   taxonomic information
%     all incompletely/partially identified speciems were excluded
%   d18o data from Zachos2008
%
% temporal bins
%   stage
%   fixed bin (2 My)
%     Alroy2000g,Alroy2000a,Alroy1996a,Alroy1998a,Alroy1998 1 My bins
%     wanted to include ~2+ formations per bin
%

Taxonomic presence-absence was recorded for each formation. Abundance information was not included in this analysis, though it is possible \citep{Sidor2013}, because the highly variable preservation conditions not just between formations but across the Cenozoic may not accurately record abundance information \citep{Damuth1982}.

% abundance information
%   relative raw
%     tabulated at number of unique genus not abundance
%   relative subsampled
%     subsampled using sqs following Alroy2010,Alroy2010b,Alroy2010c
%

Bipartite biogeographic networks were constructed, following \citet{Sidor2013} and \citet{Vilhena2013}, with species as the taxonomic occurrence and formation as the locality of interest. As explained in \citet{Sidor2013}, biogeographic networks have many advantages over ordination based methods that are frequently used in numerical ecology \citep{Legendre2012}. Principally, both taxonomic and locality information are preserved in analysis which allows for a more complete understanding of community structure. For each temporal bin biogeographic networks were constructed for all taxonomic information, individually for each dietary category and each locomotor category.

Four measures of biogeographic network structure were used to asses community change: code length as measured via the map equation \citep{Rosvall2007a,Rosvall2008,Rosvall2010b}, biogeographic connectedness, average number of locality occurrences per taxon, and average number of endemics per site. Biogeographic connectedness is defined 
\begin{equation}
  BC = \frac{O - N}{LN - N}
  \label{eq:bc}
\end{equation}
where \(O\) is the number of edges or number of occurrences in the biogeographic network, \(N\) is the number of taxa, and \(L\) is the number of localities \citep{Sidor2013}. All four of these measures have previously been used to assess cosmopolitan and endemism dynamics \citep{Sidor2013}. The four measures of biogeographic network were then calculated for each temporal bin for the total network, each of the dietary networks, and each of the locomotor category networks.

Network analysis, including calculation of network code length, was done using the \texttt{igraph} package \citep{Csardi2006} for the R language \citep{R2013}.

% resampling procedure
%   similar to Sidor2013
%     probability of removing an taxon
%       number of occurences of taxon divided by total number of occurences
%     removed taxa are replaced from random sample within same family
%       only from those taxa present in any of the localities
%   repeated X times
%
% time series analysis
%   correlation
%     first differences of biogeographic network structure series
%     first differences of Zachos2008 curve
%       between the total curves and Zachos
%       between the diet/locomotor categories and Zachos
%       between the diet and locomotor categories
%     the time series presented here are too small to accurately infer directionality
%       Hannisdal2011,Hannisdall2011d,Hannisdall2011c
%   modeling
%     Hunt2006a
%     three basic models of time series
%       random walk
%       random walk with a trend
%       stasis
%     alternative methods? 
%


\section{Results}


\section{Discussion}


\section*{Acknowledgements}
Kenneth Angielczyk, David Bapst, Michael Foote, Benjamin Frable, Dallas Krentzel, Carl Simpson. Money? What money?

\bibliographystyle{amnatnat}
\bibliography{cosmo_prov,packages}

\end{document}
