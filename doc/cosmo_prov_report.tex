% where to send?
%   AmNat 21 pages, 6 figures/tables
%   Paleobiology 50 pages

\documentclass[12pt,letterpaper]{article}
\usepackage{amsmath, amsthm}
\usepackage{parskip, graphicx, microtype}
\usepackage{caption, subcaption, multirow}
\usepackage{morefloats, hyperref}
\usepackage{rotating, longtable}
\usepackage{authblk}
\usepackage[sort&compress]{natbib}
\usepackage{fullpage}

\usepackage{lineno}

\usepackage{lipsum}

\frenchspacing


\title{Cosmopolitan and provincialism dynamics of North American terrestrial mammals across the Cenozoic}
\author[1]{Peter D Smits}
\author[2]{John Alroy}
\affil[1]{Committee on Evolutionary Biology, University of Chicago}
\affil[2]{Department of Biological Sciences, Macquarie University}

\begin{document}

\maketitle

\linenumbers
\modulolinenumbers[2]

\begin{abstract}
  % This is where the abstract would go.
  % Paleobiology: 250-300 words
  %
  \noindent (Keywords: )
\end{abstract}

\section{Introduction}
% distribution of organisms across the landscape
% how does life history affect distribution?
% how does taxonomic distribution change over time
%   life history characters
%     diet
%     life habit/locomotor category
%   abiotic factors
%     climate Zachos2001,Zachos2008
%
% previous work on contribution of different taxa to site similarity
%   Jernvall and Fortelius 2004 AmNat Jernvall2004
\lipsum[1-3]


\section{Materials and Methods}
% taxonomic occurence information from PBDB
%   mammals from North America
%   dietary information
%   life habit information
%   geological information (including formation)
%   taxonomic information
%     all incompletely/partially identified speciems were excluded
% d18o data from Zachos2008
%
% temporal bins
%   stage
%   fixed bin (2 My)
%     Alroy2000g,Alroy2000a,Alroy1996a,Alroy1998a,Alroy1998 1 My bins
%     wanted to include ~2+ formations per bin
%
% abundance information
%   relative raw
%     tabulated at number of unique genus not abundance
%   relative subsampled
%     subsampled using sqs following Alroy2010,Alroy2010b,Alroy2010c
%
% following Sidor et al. 2013 and Vilhena et al. 2013
%   bipartite occurence network
%     taxon defined as species
%     locality defined as formation
%   constructed for total data, diet data, and locomotor data
%   Sidor2013,Vilhena2013
%   igraph
%
% biogeographic analysis
%   following Sidor2013, four measures of biogeographic network structure
%     code length
%       map equation Rosvall2007a,Rosvall2008,Rosvall2010b
%       see Vilhena2013 for proof with bipartite networks
%     biogeographic connectedness
%     average number of locality occurences
%     average number of endemics
%   stage, bin
%
% resampling procedure
%   similar to Sidor2013
%     probability of removing an taxon
%       number of occurences of taxon divided by total number of occurences
%     removed taxa are replaced from random sample within same family
%       only from those taxa present in any of the localities
%   repeated X times
%
% time series analysis
%   correlation
%     first differences of biogeographic network structure series
%     first differences of Zachos2008 curve
%       between the total curves and Zachos
%       between the diet/locomotor categories and Zachos
%       between the diet and locomotor categories
%     the time series presented here are too small to accurately infer directionality
%       Hannisdal2011,Hannisdall2011d,Hannisdall2011c
%   modeling
%     Hunt2006a
%     three basic models of time series
%       random walk
%       random walk with a trend
%       stasis
%     alternative methods? 
%
\lipsum[1-3]


\section{Results}
\lipsum[1-3]


\section{Discussion}
\lipsum[1-3]


\section*{Acknowledgements}
Kenneth Angielczyk, David Bapst, Michael Foote, Benjamin Frable, Dallas Krentzel, Carl Simpson. Money? What money?

\bibliographystyle{amnatnat}
\bibliography{cosmo_prov,packages}

\end{document}
